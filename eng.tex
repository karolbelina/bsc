\documentclass[english,engineering]{wizthesis}

\usepackage[utf8]{inputenc}
\usepackage{float} % H float positioning
\usepackage{xcolor}

% Set up the thesis
\author{Karol Belina}
\title{Formal grammar\par production rule parsing tool}
\supervisor{dr inż. Zdzisław Spławski}
\fieldofstudy{Computer Science}
\keywords{Parser combinators, context-free grammars, Extended Backus-Naur Form}
\summary{}

% Set up the style of code listings (optional)
\setminted{frame=single,breaklines,linenos}
% Set up the bibliography style
\bibliographystyle{acm}

\newcommand{\todo}[1]{{\color{red}[\textbf{TODO} \textit{#1}]}}

\begin{document}

\frontmatter % Disable page and chapter numbering for this section

\maketitle

% '\chapter*' removes the abstract from the table of contents
\chapter*{Abstract}

The thesis presents the design and implementation of an EBNF-based context-free
grammar parsing tool with real-time explanations and error detection. For this
purpose, the official specification of the Extended Backus-Naur Form from the
ISO/IEC 14977 standard has been examined and transformed to an unambiguous and
ready for implementation form. The thesis proposes a definition of a grammar in
the form of an abstract syntax tree. It describes the process of tokenization
--- the act of dividing the grammar in a textual form into a sequence of tokens
--- while taking into account proper interpretation of Unicode graphemes. The
whitespace-agnostic tokens are then being combined together to form a
previously-defined AST with a technique called \textit{parser combination}. A
number of smaller helper parsers are defined, all of which are then combined
into more sophisticated parsers capable of parsing entire terms, productions and
grammars. \todo{coś o regexach w specjalnych sekwencjach?} The paper defines an
algorithm for handling left recursion in the resulting grammar defined by an
AST, as well as a dependency graph reduction algorithm for determining the
starting rule of a grammar. Up to this stage, any errors encountered in the
textual form of a grammar are reported to the user in a user-friendly format
with exact locations of the errors in the input. The paper thus compares several
techniques of storing the locations of individual tokens and AST nodes for the
purposes of error reporting. Further, the thesis describes a method of testing
an arbitrary input against the constructed grammar to determine if it belongs to
the language generated by that grammar.
\todo{tutaj prawdopodobnie coś o wyjaśnieniach zwracanych przez checker}
The thesis describes the process of creating a simple command line REPL program
to act as a basic tool for interfacing with the grammar parser and checker, but
in order to efficiently use the library, a web-based application is designed on
top of that to serve as a more visual, user-friendly and easily accessible tool.
\todo{tutaj coś o wizualizacjach, edytorze tesktowym i highlightowaniu} The
paper describes the deployment of the application on a static site hosting
service, as well as a cross-platform desktop application with the use of
Electron. The designed and implemented system gives the opportunity to extend it
with other grammar specifications.
\todo{poparafrazować ``The thesis describes...''}

\tableofcontents

\mainmatter % Re-enable page and chapter numbering

{\backmatter % Disable this chapter number
\chapter{Introduction}}

\chapter{Problem analysis}

\section{Description}

\section{Motivation}

\section{Goal}

\section{Scope}

\chapter{Analysis of similar solutions}

\chapter{Theoretical preliminaries}

\section{Context-free grammars}

\section{Specification}

\section{Grammar definition}

\section{Tokenization}

\section{Parsing}

\subsection{Methods}

\subsection{Parser combination}

\subsection{Parser definitions}

\section{Grammar preprocessing}

\subsection{Left recursion handling}

\subsection{Dependency graph reduction}

\section{Grammar processing}

\chapter{Design}

\section{Requirements}

\subsection{Functional requirements}

\subsection{Non-functional requirements}

\section{Use cases}

\section{The architecture}

\section{Interface prototype}

\chapter{Implementation}

\section{Environment}

\section{Business logic}

\subsection{Lexer}

\subsection{Parser}

\subsection{Preprocessor}

\subsection{Checker}

\section{Command line application}

\section{Web-based application}

\subsection{Linking the business logic}

\subsection{Text editor}

\subsection{Visualizations}

\chapter{Testing}

\section{Automated testing}

\subsection{Business logic testing}

\subsection{UI testing}

\section{Manual testing}

\chapter{Deployment}

\section{GitHub Pages}

\section{Electron}

{\backmatter % Disable this chapter number
\chapter{Summary}}

\bibliography{bibliography.bib}

\listoffigures

\listoftables

\listoflistings

\begin{appendices}

\chapter{Modified specification}

\begin{listing}[H]
  \inputminted[fontsize=\small]{lexers/ebnf_lexer.py:EbnfLexer -x}
  {listings/specification.ebnf}
  \caption{Modified version of the EBNF language specification defined in
  \cite{iso-14977}}
  \label{lst:specification}
\end{listing}

\end{appendices}

\end{document}
