\documentclass[english,bachelors,forcepolishlogotype]{wizthesis}

\usepackage[utf8]{inputenc}
\usepackage{float} % H float positioning
\usepackage{xcolor}
\usepackage{enumitem} % enumerate
\usepackage{amsmath, bm}
\usepackage{mathtools}
\usepackage[ruled,vlined,algochapter]{algorithm2e}
\usepackage{wrapfig}
\usepackage{tikz}
\usetikzlibrary{automata,positioning,arrows,trees,shapes}
\usepackage{booktabs}
\usepackage{xltabular}
\usepackage{array} % \raggedright, \arraybackslash
\usepackage[polish,english,main=english]{babel}
\usepackage{siunitx}
\usepackage{subcaption}
% Please load this as the very last package for footnotes to link correctly
\usepackage{hyperref} % Hyperlinks

% enumerate spacings in xltabular environment
\AtBeginEnvironment{xltabular}{\setlist[enumerate, 1]{wide, leftmargin=*,
  itemsep=0pt, before=\vspace{-\dimexpr\baselineskip +2 \partopsep},
  after=\vspace{-\baselineskip}}}

% Set up the thesis
\author{Karol Belina}
\title{Formal grammar\par production rule parsing tool}
\fieldofstudy{Computer Science}
\supervisor{dr inż.\ Zdzisław Spławski}
\keywords{combinatory parsing, context-free grammars, Extended Backus-Naur Form}
\summary{The thesis documents the process of designing and implementing a tool
for parsing the production rules of context-free grammars in a textual form. It
discusses the choice of Extended Backus-Naur Form notation over the alternatives
and provides a model for parsing such a notation. The implemented parser can
turn a high-level specification of a grammar into a parser itself, which in turn
is capable of constructing a parse tree from arbitrary input provided to the
program with the use of parser combinators.}

% Set up the bibliography style
\bibliographystyle{acm}
% Set up the column types
\newcolumntype{L}[1]{%
  >{\raggedright\let\newline\\\arraybackslash\hspace{0pt}}p{#1}%
}
% Set up the graphics path
% This is done to fix the paths inside the .pdf_tex files generated by inkscape.
\graphicspath{{images/}}
% Set up the comment style for algorithm2e
\newcommand\mycommfont[1]{\footnotesize\ttfamily#1}
\SetCommentSty{mycommfont}
% booktabs and xltabular bug fix
\makeatletter
\def\@BTrule[#1]{%
  \ifx\longtable\undefined
    \let\@BTswitch\@BTnormal
  \else\ifx\hline\LT@hline
    \nobreak
    \let\@BTswitch\@BLTrule
  \else
     \let\@BTswitch\@BTnormal
  \fi\fi
  \global\@thisrulewidth=#1\relax
  \ifnum\@thisruleclass=\tw@\vskip\@aboverulesep\else
  \ifnum\@lastruleclass=\z@\vskip\@aboverulesep\else
  \ifnum\@lastruleclass=\@ne\vskip\doublerulesep\fi\fi\fi
  \@BTswitch}
\makeatother

\setlist[description]{font=\normalfont}

\newcommand{\todo}[1]{%
  \textcolor{red}{[\textbf{TODO}\ifx&#1&{}\else{ }\fi\emph{#1}]}%
}

\newcommand{\thisproject}{Parser-parser}

\begin{document}

\frontmatter % Disable page and chapter numbering for this section

\maketitle

% '\chapter*' removes both abstracts from the table of contents
\chapter*{Abstract}

The thesis presents the design and implementation of a context-free grammar
parsing tool with real-time explanations and error detection. It discusses the
choice of Extended Backus-Naur Form notation over the alternatives and provides
a model for parsing such a notation. For this purpose, the official
specification of the EBNF from the ISO/IEC 14977 standard has been examined and
transformed into an unambiguous form. A definition of an abstract syntax tree is
proposed to act as a result of the syntactic analysis phase formed with a
technique called \emph{combinatory parsing}. A method of testing an arbitrary
input against the language generated by the constructed grammar is described.
The thesis shows the process of creating a simple command-line program to act as
a basic tool for interfacing with the grammar parser and checker, but in order
to efficiently use the library, a web-based application is designed on top of
that to serve as a more visual, user-friendly and easily accessible tool. It
describes the deployment of the application on a static site hosting service.
The designed and implemented system gives the opportunity to extend it with
other grammar specifications.

{\let\clearpage\relax % Keep the polish abstract on the same page
\begin{otherlanguage}{polish}

\chapter*{Streszczenie}

Praca przedstawia proces projektowania i~implementacji narzędzia służącego~do
analizy~syntaktycznej gramatyk~bezkontekstowych z~naciskiem na~obsługę błędów
i~wyjaśnień w czasie rzeczywistym. Omawia wybór rozszerzonej~notacji
Backusa-Naura i~przestawia model do~analizy takiej notacji. W~tym celu
przeprowadzono analizę i~przekształcenie w jednoznaczną formę oficjalnej jej
specyfikacji zdefiniowanej w standardzie ISO/IEC~14977. Zaproponowano definicję
abstrakcyjnego drzewa składniowego dla tej notacji, która tworzona jest w wyniku
analizy syntaktycznej za pomocą techniki zwanej \emph{kombinacją~parserów}.
Opisano metodę sprawdzania dowolnego ciągu znaków pod kątem języka generowanego
przez analizowaną gramatykę. Praca przedstawia stworzenie prostego programu
działającego z~poziomu wiersza poleceń, który jest podstawowym narzędziem do
analizy gramatyk, jednak by móc efektywnie korzystać ze~stworzonej biblioteki,
zaprojektowano aplikację webową, która służy za bardziej wizualne, przyjazne i
łatwo dostępne dla użytkownika narzędzie. Praca opisuje wdrażanie aplikacji
na~usługę hostingową dla statycznych stron. Zaprojektowany i~wdrożony system
daje możliwość rozszerzenia go o~inne specyfikacje gramatyk.

\end{otherlanguage}
}

\tableofcontents

\chapter*{Thesis structure}

\begin{description}
  \item[Chapter~\ref{ch:problem-analysis}] describes the aim of the thesis and
  the problem it is solving. It states the scope of the project and provides a
  glossary of common terms used throughout the thesis.
  \item[Chapter~\ref{ch:theoretical-preliminaries}] introduces the reader to
  theoretical concepts regarding the project.
  \item[Chapter~\ref{ch:analysis-of-similar-solutions}] analyses various
  solutions available on the market, which are similar to the project stated in
  the thesis.
  \item[Chapter~\ref{ch:design-of-the-project}] introduces a specification of
  the project in terms of requirements, use cases, user stories, and various UML
  diagrams representing the architecture, use case scenarios, or the flow of the
  business logic.
  \item[Chapter~\ref{ch:implementation-of-the-project}] talks about the
  implementation phase of the project. It describes the technologies used in the
  implementation, as well as the project structure. Finally, it presents various
  intricacies encountered during the implementation process.
  \item[Chapter~\ref{ch:project-quality-study}] studies the quality of the
  implemented project by proposing unit and integration tests of the business
  logic, defining benchmarks, as well as describing various auditing techniques.
  \item[Chapter~\ref{ch:deployment}] describes the process of deploying the
  final product into production environment, as well as techniques related to
  continuous integration.
  \item[Chapter~\ref{ch:software-artifacts}] presents the results of the project
  in form of software artifacts.
  \item[Chapter~\ref{ch:user-manual}] describes the process of installing,
  building, and running the web application on a local machine, as well as lists
  its system and software requirements. It presents screenshots of the web
  application along with a user guide and an example use case scenario of the
  product.
  \item[Chapter~\ref{ch:summary}] sums up the thesis and proposes next steps
  in the evolution of the project.
\end{description}

\mainmatter % Re-enable page and chapter numbering

\chapter{Problem analysis} \label{ch:problem-analysis}

\section{Description and motivation} \label{sec:description-and-motivation}

Programming language theory has become a well-recognized branch of computer
science that deals with the study of programming languages and their
characteristics. It is an active research field, with findings published in
various journals, as well as general publications in computer science and
engineering. But besides the formal nature of Programming language theory, many
amateur programming language creators try their hand at the challenge of
creating a programming language of their own as a personal project. It is
certainly relevant for a person to write their own language for educational
purposes, and to learn about programming language and compiler design. However,
the language creator must first of all make some fundamental decisions about the
paradigms to be used, as well as the syntax of the language.

The tools for aiding the design and implementation of the syntax of a language
are generally called \emph{compiler-compilers}. These programs generate parsers,
interpreters, or compilers from some formal description of a programming
language (usually a grammar). The most commonly used types of compiler-compilers
are \emph{parser generators}, which handle only the syntactic analysis of the
language --- they do not handle the semantic analysis, nor the code generation
aspect. The parser generators most generally transform a grammar of the syntax
of a given programming language into a source code of a parser for that
language. The language of the source code for such a parser is dependent on the
parser generator.

Most such tools, however, suffer from too much complexity and generally have a
steep learning curve for people inexperienced with the topic. Limited
availability makes them less fitted for prototyping a syntax of a language ---
they often require a complex setup for simple tasks, which is not welcoming for
new users. The lack of visualization capabilities shipped with these tools makes
them less desirable for teachers in the theory of formal languages, who often
require such features for educative purposes in order to present the
formulations of context-free grammars in a more visual format.

\section{Goal of the thesis}

The main goal of this thesis is to design and implement a specialized tool, that
serves teachers, programmers and other kinds of enthusiasts of the theory of
formal languages in the field of discrete mathematics and computer science, in
order to formulate and visualize context-free grammars in the Extended
Backus-Naur Form. To make it more approachable to users, the tool must provide a
graphical user interface. Additionally, to ensure the hightest degree of
approachability, the tool must be available in the form of an easily accessible
web-based application that is deployed on a web page and can run in a browser
without the need of installation on the user's device. The thesis itself will
document the entire process of creating such a project.

The final product will be much more welcoming to new users than other similar
tools. Users will be able to access the website and use the tool right from
their browser without needing to install any programs on their device or carry
out a complex setup process. The tool will be able to visualize the parse tree
in an interactive manner, which will provide additional benefits to the entire
process of working with this tool.

In order to achieve the general goal, several sub-goals have been
distinguished, all of which contribute to the main objective as a whole
\begin{itemize}
  \item analysis of existing solutions and applications,
  \item presentation of the theoretical preliminaries of the project,
  \item definition of the outline of the project, including a description of the
  functional and non-functional requirements, the use case diagram, use case
  scenarios, and user stories,
  \item description of technologies used in the implementation,
  \item implementation of the project,
  \item description of the testing and deployment environments.
\end{itemize}

\section{Scope of the project}

The thesis will propose a definition of a grammar in the form of an abstract
syntax tree of the Extended Backus-Naur Form. It will describe the process of
implementing the business logic of the application in the Rust programming
language~\cite{rust} compiled to WebAssembly~\cite{webassembly}. The compiled
code is then ran inside the web-based application made with the Svelte
framework, which incorporates the markup, CSS styles, and JavaScript scripts in
the superset of the HyperText Markup Language (HTML).

The implementation phase will include the process of tokenization --- the act of
dividing the grammar in a textual form into a sequence of tokens --- while
taking into account proper interpretation of Unicode graphemes. The
whitespace-agnostic tokens will be then combined together to form an instance of
a previously-defined abstract syntax tree with a technique called
\emph{combinatory parsing}. Several smaller helper parsers will be defined, all
of which then will be combined into more sophisticated parsers capable of
parsing entire terms, productions, and grammars. The implementation phase will
also include the definition of an algorithm for detecting left recursion in the
resulting grammar, as well as catching any undefined production rules referenced
in non-terminals. Up to this stage, any errors encountered in the textual form
of a grammar are going to be reported to the user in a friendly format with
exact locations of the errors in the input. After providing a valid definition
of a grammar, the user will be able to supply an arbitrary input string to check
if it belongs to the language generated by that grammar. This process will
produce a parse tree, which will be displayed to the user as an interactive
component. The web application will provide a basic code editor for inputting
the grammar, with autocompletions, syntax highlighting, and other user-friendly
features. The scope of the thesis includes the implementation of a simple
command-line program for basic interfacing with the grammar parser and checker.

The web application will be deployed on the GitHub Pages hosting service for
static sites.

\newpage

\section{Glossary}

Please note that all terms below are explained in detail further in the thesis
in their respective sections. This section provides simplified definitions of
these terms.

\begin{description}
  \item[AST --- Abstract syntax tree] \hfill \\ A tree representation of the
  syntactic structure of computer source code written in some language.
  \item[BNF --- Backus-Naur form] \hfill \\ A notation for context-free grammars
  used to describe the syntax of various languages.
  \item[CFG --- Context-free grammar] \hfill \\ A type of a formal grammar, in
  which the production rules can be applied regardless of the context (the
  surrounding symbols) of a non-terminal.
  \item[DFA --- Deterministic finite automaton] \hfill \\ A finite-state machine
  that accepts or rejects a given string of symbols, by running through a state
  sequence uniquely determined by the string.
  \item[EBNF --- Extended Backus-Naur form] \hfill \\ A family of languages used
  to describe other languages. An extension of Backus-Naur Form; allows to more
  easily define optionals, repetitions, and other constructs.
  \item[Formal grammar] \hfill \\ Formally defines the construction of strings
  of symbols from the language's alphabet according to the language's syntax.
  \item[LBA --- Linear bounded automaton] \hfill \\ A restricted form of Turing
  machine that is able to recognize context-sensitive languages described by
  type-1 grammars in the Chomsky hierarchy.
  \item[Parser] \hfill \\ A software component that takes input in the form of a
  sequence of characters or tokens and produces a structured output.
  \item[\thisproject{}] \hfill \\ The codename of the project.
  \item[PDA --- Pushdown automaton] \hfill \\ A type of automaton that employs a
  stack and is able to recognize context-free~languages.
  \item[PEG --- Parsing expression grammar] \hfill \\ A type of analytic formal
  grammar introduced by Bryan Ford in 2004, which is an unambiguous version of
  the context-free grammar more suitable for machine parsing.
  \item[RE --- Regular expression] \hfill \\ The most restrictive type of a
  formal grammar often used for lexical analysis of programming languages and
  for defining search patterns for text.
  \item[REPL --- Read-Eval-Print loop] \hfill \\ A simple interactive computer
  programming environment that takes single user inputs, executes them, and
  returns the result to the user.
\end{description}

\chapter{Theoretical preliminaries} \label{ch:theoretical-preliminaries}

\section{Formal grammars}

\subsection{Introduction to formal grammars}

\emph{Formal grammar} of a language defines the construction of strings of
symbols from the language's \emph{alphabet} according to the language's
\emph{syntax}. It is a set of so-called \emph{production~rules} for
rewriting certain strings of symbols with other strings of symbols --- it can
therefore generate any string belonging to that language by repeatedly applying
these rules to a given starting symbol~\cite{meduna-2014}. Furthermore, a
grammar can also be applied in reverse: it can be determined if a string of
symbols belongs to a given language by breaking it down into its constituents
and analyzing them in the process known as \emph{parsing}.

For now, let's consider a simple example of a formal grammar. It consists of two
sets of symbols: (1) set $N = \{\,S, B\,\}$, whose symbols are
\emph{non-terminal} and must be rewritten into other, possibly non-terminal,
symbols, and (2) set $\Sigma = \{\,a, b, c\,\}$, whose symbols are
\emph{terminal} and cannot be rewritten further. Let $S$ be the start symbol
and set $P$ be the set of the following production rules:
\begin{enumerate}[noitemsep]
  \item $S \rightarrow aBSc$
  \item $S \rightarrow abc$
  \item $Ba \rightarrow aB$
  \item $Bb \rightarrow bb$
\end{enumerate}
To generate a string in this language, one must apply these rules (starting with
the start symbol) until a string consisting only of terminal symbols is
produced. A production rule is applied to a string by replacing an occurrence
of the production rule's left-hand side in the string by that production rule's
right-hand side. The simplest example of generating such a string would be
\begin{equation*}
  S \xRightarrow[2]{} \underline{abc}
\end{equation*}
where $P \xRightarrow[i]{} Q$ means that string $P$ generates the string $Q$
according to the production rule $i$, and the generated part of the string
is underlined.

By choosing a different sequence of production rules we can generate a different
string in that language
\begin{equation*}
\begin{split}
  S & \xRightarrow[1]{} \underline{aBSc} \\
    & \xRightarrow[2]{} aB\underline{abc}c \\
    & \xRightarrow[3]{} a\underline{aB}bcc \\
    & \xRightarrow[4]{} aa\underline{bb}cc
\end{split}
\end{equation*}

After examining further examples of strings generated by these production rules
we may come into a conclusion that this grammar generates the language
$\{\,a^nb^nc^n \mid n \ge 1\,\}$, where $x^n$ is a string of $n$ consecutive $x$'s.
It means that the language is the set of strings consisting of one or more
$a$'s, followed by the exact same number of $b$'s, then followed by the exact
same number of $c$'s.

Such a system provides us with a notation for describing a given
language formally. Such a language is a usually infinite set of finite-length
sequences of terminal symbols from that language.

\subsection{The Chomsky Hierarchy}

\begin{wrapfigure}{R}{0.45\textwidth}
  \centering
  \begin{tikzpicture}[scale=2.5]
    \draw[color=black](0,0) ellipse (1.25 and 1)
      node at (0, 0.68) {\small recursively enumerable};
    \draw[color=black](0,-0.2) ellipse (1 and 0.75)
      node at (0, 0.25) {\small context-sensitive};
    \draw[color=black](0,-0.4) ellipse (0.75 and 0.5)
      node at (0, -0.15) {\small context-free};
    \draw[color=black](0,-0.6) ellipse (0.5 and 0.25)
      node {\small regular};
  \end{tikzpicture}
  \caption{The~Chomsky~Hierarchy visualized.}
  \label{fig:chomsky-hierarchy}
\end{wrapfigure}

In~\cite{chomsky-1956}, Chomsky divides formal grammars into four classes and
classifies them in the now called \emph{Chomsky~Hierarchy}. Each class is a
subset of another, distinguished by the complexity. The hierarchy is visualized
in Figure~\ref{fig:chomsky-hierarchy}.

Type-3 grammars generate the so-called \emph{regular~languages}. As described
in~\cite{aho-1990}, regular languages can be matched by
\emph{regular~expressions} and decided by a \emph{finite~state~automaton}. They
are the most restricting kinds of grammars, with its production rules consisting
of a single non-terminal on the left-hand side and a single terminal, possibly
followed by a single non-terminal on the right-hand side. Because of their
simplicity, regular languages are used for lexical analysis of programming
languages~\cite{johnson-1968}.

Type-2 grammars produce \emph{context-free~languages} and can be represented as
a \emph{pushdown~automaton} which is an automaton that can maintain its state
with the use of a stack. A pushdown automaton (PDA) varies from a finite state
machine, as it is able to utilize the top of the stack in order to decide on the
next transition, as well as manipulate the stack as part of performing a
transition. A pushdown automaton reads the input sequentially from left to
right. For every step, it decides which transition to take by taking into
account the input symbol, the current state, and the current symbol on the top
of the stack. A pushdown automaton can also mutate the stack by, for example,
pushing a particular symbol onto the stack, or popping off the element from the
top of the stack. A pushdown automaton can also just ignore the stack, and leave
it in the same state as before. Context-free languages are defined by rules of
the form of a single non-terminal on the left-hand side with a string of
terminals or non-terminals on the right-hand side. They are the theoretical
basis for the syntax and structure of most programming languages of today, even
though their syntax usually comes with the problem of context-sensitive name
resolution \cite{hopcroft-2005}.

Type-1 grammars generate context-sensitive languages. The languages described by
these grammars are exactly all languages that can be recognized by a
\emph{linear bounded automaton} (LBA) --- a Turing machine whose tape size is
finite and can be predicted based on the length of the input. This feature makes
an LBA a more realistic representation of a real-world computer than a Turing
machine, which by definition has a tape of infinite length. Because of the
ability to perform random access on its memory, LBAs can generate languages
based on an arbitrary context, hence the name of languages these grammars
generate. Context-sensitive languages have less restrictive rules than type-2
and type-3 grammars --- these come in the form of $\alpha A \beta \rightarrow
\alpha \gamma \beta$, where $A$ is a non-terminal, $\alpha$, $\beta$, and
$\gamma$ are strings of terminals, and $\alpha$ and $\beta$ may be empty.

Type-0 grammars include all formal grammars. They generate all languages that
can be recognized by a Turing machine. These languages are also known as the
\emph{Turing-recognizable} or \emph{recursively enumerable} languages. They have
no constraints on the structure of the production rules.

\subsection{Parsing expression grammars} \label{sbs:pegs}

The syntax of various programming languages and protocols has been expressed
with the use of Chomsky hierarchy of grammars --- context-free grammars and
regular expressions, in particular. Their original purpose is modeling natural
languages, and the ability to express ambiguous constructs reflects that. This,
however, makes it difficult to express deterministic, machine-oriented languages
using context-free grammars. \emph{Parsing expression grammars} (PEGs),
introduced by Ford in \cite{ford-2004} are an alternative to context-free
grammars for specifying syntax formally. From the syntactic point of view, PEGs
look similar to context-free grammars. They, however, have a different
interpretation: most notably, the choice operator in PEGs chooses the first
match, while in CFGs it happens ambiguously. As a result, in PEGs, if a string
parses successfully, it must have exactly one valid parse tree.

PEGs address limitations of expressiveness of context-free grammars and regular
expressions by simplifying their syntax definitions. Each parsing rule in a PEG
has the form $A \leftarrow e$, where A is a non-terminal symbol, and $e$ is a
\emph{parsing expression}. An \emph{atomic} parsing expression is either a
terminal symbol, a non-terminal symbol, or the empty string $\varepsilon$. A
parsing expression which consists of a single terminal yields a \emph{success}
only if the beginning of the input string matches that terminal, and then
consumes an appropriate number of input characters. In any other case, the
expression yields a \emph{failure} result. A parsing expression consisting of
the empty string always succeeds without consuming any input. A parsing
expression consisting of a non-terminal $A$ depicts a recursive call to the
production rule $A$.

Terminals, non-terminals, and empty strings may be combined to construct more
complex parsing expression using the following operators:
\begin{itemize}[noitemsep]
  \item Sequence $e_1\ e_2$, which first invokes $e_1$, and in case $e_1$
  succeeds, subsequently invokes $e_2$ on the unconsumed part of the input
  string, and returns the result. In case either $e_1$ or $e_2$ fails, the
  entire sequence expression $e_1\ e_2$ fails consuming no input.
  \item Ordered choice $e_1\ /\ e_2$, which first invokes $e_1$, and in case
  $e_1$ succeeds, returns the result. Otherwise, in case $e_1$ fails, the choice
  operator calls $e_2$ with the original input it invoked $e_1$ with, returning
  $e_2$'s result.
  \item Zero-or-more $e*$, which consumes zero or more consecutive repetitions
  of the sub-expression $e$.
  \item One-or-more $e+$, which consumes one or more consecutive repetitions of
  the sub-expression $e$.
  \item Optional $e?$, which consumes zero or one occurrence of the
  sub-expression $e$.
  \item And-predicate $\&e$, which invokes the sub-expression $e$, and then
  succeeds if $e$ succeeds and fails if $e$ fails, but never consumes any input.
  \item Not-predicate $!e$, which succeeds if $e$ fails and fails if $e$
  succeeds, consuming no input.
\end{itemize}
where $e$, $e_1$, and $e_2$ are parsing expressions. 

\section{Why EBNF?} \label{sec:why-ebnf}

\emph{Backus-Naur form} or \emph{Backus normal form} (BNF) is a metasyntax
notation for context-free grammars, often used similarly to Parsing expression
grammars (discussed in Section~\ref{sbs:pegs}) to describe the syntax of various
languages, such as computer programming languages, instruction sets, document
formats and communication protocols. They are applied in official language
specifications, in manuals, and in textbooks on programming language theory ---
anywhere the precise descriptions of languages are required.

\begin{wraplisting}{l}{0.5\textwidth}
  \begin{minted}[fontsize=\small,frame=lines,breaklines]{bnf}
<expr> ::= <term> "+" <expr> | <term>
<term> ::= <factor> "*" <term> | <factor>
<factor> ::= "(" <expr> ")" | <const>
<const> ::= "0" | "1" | ... | "9"
  \end{minted}
  \caption{Example expression grammar in BNF.}
  \label{lst:bnf}
\end{wraplisting}

BNF was proposed by John Backus, a programming language designer at IBM. He
introduced a metalanguage of ``metalinguistic formulas'' to define the grammar
of the new programming language IAL, recognized today as ALGOL 58. His notation
was used for the first time in the ALGOL 60 report. BNF is a notation for
context-free grammars defined by Chomsky.

A BNF specification is a set of production rules, written as \texttt{A ::= e},
where \texttt{A} is a non-terminal, and \texttt{e} consists of one or more
sequences of symbols, where each sequence is separated by the vertical bar
``\texttt{|}'', which represents a choice. The entire expression \texttt{e} is a
potential substitution for the symbol on the left-hand side. Symbols that do not
appear on a left side of any production rule are terminals. On the other hand,
symbols that do appear on a left side are non-terminals, which must be always
surrounded by angle brackets ``\texttt{<>}'', as seen in Listing~\ref{lst:bnf}.

Extended Backus-Naur form (EBNF) is a family of extensions of the original
Backus-Naur form. These are not, by any means, a superset of BNF, and there's
not a single official EBNF --- each author defines their own variant of EBNF
that is different from others. Any grammar described with EBNF can likewise be
expressed in BNF, however, the representations of BNF are usually longer and
more verbose. Constructs, such as options or repetitions are impossible to
express directly in BNF. These require an intermediate production rule or, for
options, an alternative operator of either nothing or the optional production,
and, for repetitions, an alternative operator of nothing or the repeated
production of itself, recursively. The BNF uses the symbols ``\texttt{<}'',
``\texttt{>}'', ``\texttt{|}'', and ``\texttt{::=}'' for its grammar, but does
not originally include quotes around terminals. As a result, these characters
cannot be used in languages defined in BNF, and require a unique symbol for the
empty string. In EBNF, terminal symbols must be surrounded by quotation marks
(\texttt{"}...\texttt{"} or \texttt{'}...\texttt{'}). The angle brackets
(``\texttt{<}...\texttt{>}'') used normally for non-terminals in BNF can be
ignored in EBNF. BNF syntax can only represent a single rule in a single line,
whereas in EBNF the end of a production rule is marked with the semicolon
character ``\texttt{;}''. Moreover, EBNF provides mechanisms for extensions,
specifying the number of repetitions, excluding alternatives, comments, and so
on.

The first EBNF notation was developed by Niklaus Wirth, who combined concepts of
\emph{Wirth syntax notation} with a different syntax. Many of the variants of
EBNF are currently still in use. The International Organization for
Standardization adopted an EBNF standard (ISO/IEC 14977) in 1996
\cite{iso-14977}. \thisproject{} uses EBNF as specified by that ISO. Other EBNF
variants use different syntactic conventions.

\section{Modifying the ISO/IEC 14977 specification}
\label{sec:modified-specification}

There are many concerns regarding the modern use of the ISO/IEC 14977 standard
\cite{wheeler-2019}. The use of this standard in \thisproject{} comes mostly
from its popularity, however, some considerations regarding the official
specification of the notation should be taken into account. Although
\thisproject{} conforms to the majority of the EBNF specification, there are
some deviations introduced to facilitate the use of the specification in a
modern environment.

First of all, ISO/IEC 14977 does not support International/Unicode characters,
code points, or byte values. Instead, it is defined through ISO/IEC 646
characters. Even though it provides a special notation for describing a
character in an informal way, it does not have official support for it,
consequently, having no ability to directly represent the full range of
characters allowed by ISO/IEC 10646 or Unicode when processing text.
Additionally to text, ISO/IEC 14977 cannot describe binary formats or provide
characters by their value. It cannot include Unicode characters as terminal
symbols in the grammar by surrounding them with single or double-quotes.
\thisproject{} extends the ISO/IEC 14977 specification by taking into account
Unicode characters in place of \texttt{letter} and \texttt{decimal digit}
definitions in the ISO/IEC 14977 EBNF specification \cite{iso-14977}.

Secondly, the ISO/IEC 14977 specification may be difficult to understand, and
many of the key concepts are not clearly defined. The specification is abstract,
which may be somewhat inevitable given the subject matter. The \texttt{syntax}
production rule in the specification is defined ambiguously in multiple sections
of the specification.

This led to an eventual modification of the ISO/IEC 14977 specification for the
final implementation of the EBNF parser. The modified specification can be seen
in Appendix~\ref{ch:modified-spec}. It addresses the problem of Unicode
characters and unambiguously defines the grammar.

\section{Lexical analysis} \label{sec:lexing}

Prior to parsing, the input string provided by the user should be divided into
individual \emph{tokens} in the process of \emph{tokenization}, or the
\emph{lexical analysis}. Thanks to tokenization, all symbols provided in a
textual form get separated and become independent from whitespace characters or
other meaningless constructs of the language. This makes the parsing phase
easier to reason about --- the designers do not need to concern themselves with
whitespace on every occasion.

A program that performs lexical analysis may be termed a \emph{lexer},
\emph{tokenizer}, or \emph{scanner}. A lexer is generally combined with a
parser, which are used together to analyze the syntax of programming languages,
web pages, and other similar concepts. A lexer outlines the first phase of a
parser front-end in language processing. The lexical analysis commonly happens
in a single pass. Lexers tend to be uncomplicated, with most of the complexity
deferred to the parsing phase. However, lexers can sometimes involve more
complex behavior to make the input easier and simplify the parser. Lexers may be
written by hand, to either support more features or simply for performance
reasons \cite{sipser-2009}.

A \emph{lexical token} or simply \emph{token} is a string with an assigned
meaning. It is defined as a pair of a \emph{token name} and an optional
\emph{token value}. The token name is a category of the token. The EBNF
specification consists of several, clearly identifiable token types, which are
listed in Table~\ref{tab:tokens}.

\begin{longtable}[c]{@{}lm{9cm}@{}}
  \caption{Tokens types from the modified EBNF notation.}
  \label{tab:tokens}\\
  \toprule
  Token name           & Normal representation \\* \midrule
  \endfirsthead
  %
  \endhead
  %
  \endfoot
  %
  \endlastfoot
  %
  Non-terminal         & a string of alphanumeric characters or whitespace
  characters beginning with an alphabetic character \\
  Terminal             & a string of at least one character surrounded by either
  ``\texttt{'}''s or ``\texttt{"}''s \\
  Special              & a string of characters surrounded by question marks
  (``\texttt{?}'') \\
  Integer              & a string of at least one decimal digit with optional
  whitespace in-between \\
  Concatenation        & ``\texttt{,}'' \\
  Definition           & ``\texttt{=}'' \\
  Definition separator & ``\texttt{|}'', ``\texttt{/}'', or ``\texttt{!}'' \\
  End group            & ``\texttt{)}'' \\
  End option           & ``\texttt{]}'' or ``\texttt{/)}'' \\
  End repeat           & ``\texttt{\}}'' or ``\texttt{:)}'' \\
  Exception            & ``\texttt{-}'' \\
  Repetition           & ``\texttt{*}'' \\
  Start group          & ``\texttt{(}'' \\
  Start option         & ``\texttt{[}'' or ``\texttt{(/}'' \\
  Start repeat         & ``\texttt{\{}'' or ``\texttt{(:}'' \\
  Terminator           & ``\texttt{;}'' \\* \bottomrule
\end{longtable}

Several of these tokens must carry an additional token value to distinguish it
from other tokens of the same name. These include: non-terminals, terminals,
specials, and integers.

\section{Syntactic analysis} \label{sec:parsing}

\emph{Parsing}, \emph{syntax analysis}, or \emph{syntactic analysis} is the
process of analyzing a string of symbols, either in natural languages, computer
languages, or data structures, adhering to the rules of a formal grammar. A
\emph{parser} is a program that takes input data and produces a data structure
--- usually, a parse tree, abstract syntax tree, or other hierarchical
structure, providing a structural description of the input while simultaneously
ensuring the correct syntax \cite{aho-2019,sipser-2009}. This is usually
accomplished with a context-free grammar which defines elements that make up an
expression and the order in which they appear, in a recursive manner.
Determining if and how the input string can be produced from the initial symbol
of the grammar is the main responsibility of the parser. This usually can be
accomplished in two ways:
\begin{description}
  \item[Top-down parsing] --- This method of parsing can be portrayed as finding
  left-most derivations of an input string by searching for parse trees using a
  top-down expansion of the given production rules of the formal grammar. Tokens
  of the input string are consumed sequentially from left to right. The
  inclusive choice is used to support ambiguity by expanding all alternative
  right-hand-sides of production rules.
  \item[Bottom-up parsing] --- This method of parsing starts with the input string
  and makes an attempt to rewrite it back to the initial symbol. The parser
  tries to locate the most basic components, then the elements containing these,
  etc.
\end{description}

\subsection{Combinatory parsing}

\emph{Combinatory parsing} is a top-down parsing technique that involves
\emph{parser combinators}. A parser combinator is a higher-order function that
receives several parsers as inputs and produces an entirely new parser as its
output. In this context, a parser is a function accepting a string of tokens as
input and returning some structured output, usually a parse tree. Parser
combinators enable a strategy of recursive descent parsing that simultaneously
facilitates modular construction of parsers \cite{swierstra-2009}.

Parsers formulated with the use of parser combinators are straightforward to
construct, readable, modular, well-structured, and easily maintainable
\cite{swierstra-2009}. They have been widely used in situations where complex
and diverse semantic actions are closely integrated with syntactic analysis,
such as compiler prototyping and processors for domain-specific languages.

Following examples of parsers will be written in the Haskell \cite{haskell}
programming language. Haskell is purely functional, and one of its features
(heavily utilized in examples below) is higher-order functions, which are
functions that can be passed to other functions as values. Every expression in
Haskell has a type which is determined at compile time. Types become not only a
form of guarantee, but a language for expressing the construction of programs.
At the same time, the programmer does not have to state every type of every
value, as types will be inferred by the compiler automatically. This makes the
syntax of Haskell very concise and ideal for presenting various mathematical
concepts.

The basic idea of a parser is a function from strings to some \emph{things}. It
can be easily defined in a programming language that supports higher-order
functions. The definition of a parser in Haskell would look like
\begin{minted}{haskell}
type Parser t = String -> Maybe (t, String)
\end{minted}
A parser that returns \emph{things} of an arbitrary type \texttt{t} is a
function that takes a string and returns \texttt{Maybe} a pair of \emph{thing}
\texttt{t} and the unconsumed part of the input string, which can in turn be
passed on to some other parser.

The \texttt{Parser} type utilizes the built-in \texttt{Maybe} type --- a type
that represents an optional value. \texttt{Maybe}, as an algebraic data type is
defined as
\begin{minted}{haskell}
data Maybe t = Just t | Nothing
\end{minted}
\texttt{Maybe} is a type constructor of a single argument (here, the \texttt{t}
type), where the value of type \texttt{Maybe t} either contains a value of type
\texttt{t} (represented as \texttt{Just t}, where \texttt{Just} is a
single-argument value constructor of \texttt{Maybe}), or it is empty
(represented as \texttt{Nothing}, which takes no arguments). Using
\texttt{Maybe} is a good way to deal with errors or exceptional cases, as it is
in the \texttt{Parser} type. Further in the thesis, Listing~\ref{lst:token} in
Chapter~5 will illustrate a concrete use of algebraic data types. Please note
that the \texttt{Parser} type is not an algebraic data type --- it is an alias
to an existing function type from \texttt{String}s to \texttt{Maybe (t,
String)}s. A type alias does not introduce any value constructors for that type.

Such a \texttt{Parser} type can be found in every library for combinatory
parsing, and all of these libraries work in the same way: the user can utilize
\emph{primitives} or ``building blocks'' for parsers, and then combine them to
build bigger, more sophisticated parsers.

One of the simple primitive parsers is a parser for parsing a single digit at
the beginning of the string, here called \texttt{digit}.
\begin{minted}{haskell}
parse digit "123"
-- Just ('1', "23")
\end{minted}
It takes a string of characters and returns a pair of the parsed digit and the
remaining characters, which the user may want to subsequently parse with another
parser.

Another parser might be called \texttt{char} and parse a specified character at
the beginning of the input string. Here it is specialized to parse a character
``a'', but since there's no character ``a'' at the beginning of ``bcd'' it fails
and returns \texttt{Nothing}.
\begin{minted}{haskell}
parse (char 'a') "bcd"
-- Nothing
\end{minted}

The code snippet shown below presents a way to combine these primitive parsers
into more complex ones. For example, the \texttt{<|>} operator tries to parse
either a \texttt{digit}, or if it fails it then tries to parse a
\texttt{letter}. Then, the \texttt{multiple} combinator takes a parser and
applies it as many times as it can to the input string.
\begin{minted}{haskell}
parse (multiple (digit <|> letter)) "abc123"
-- Just ("abc123", "")
\end{minted}

Consider the following grammar defined in EBNF:
\begin{minted}[fontsize=\small]
  {lexers/ebnf_lexer.py:EbnfLexer -x}
expression = term, '+', expr | term;
term       = factor, '*', term | factor;
factor     = '(', expression, ')' | digit;
digit      = '0' | '1' | '2' | '3' | '4' | '5' | '6' | '7' | '8' | '9';
\end{minted}
The \texttt{expression} production rule could be defined with parser combinators
like so
\begin{minted}{haskell}
expression = do x <- term
               char '+'
               y <- expression
               return (Addition x y)
             <|> term
\end{minted}
where \texttt{term}, \texttt{expression}, and \texttt{char} are different
parsers.

The \texttt{do} keyword in the example above signifies a \emph{``do'' block},
where several parsers are applied to the input sequentially. As all values
returned by \texttt{term}, \texttt{expression}, and \texttt{char} parsers may
result in an error (the \texttt{Nothing} value), it would be reasonable to not
check subsequent parsers after encountering a \texttt{Nothing} value, as it is
guaranteed the whole parser already failed. ``Do'' blocks are a concise way to
represent just that --- a sequential operation which might result in an error,
without having to check if the \texttt{Maybe} values are \texttt{Just} values or
\texttt{Nothing} values at every step. If any of the values are
\texttt{Nothing}, the whole \texttt{do} expression will result in a
\texttt{Nothing}.

The key observation here is that the production rule looks basically
the same as the parser. With some more of these smaller parsers, a complete
parser for arithmetic expressions could be implemented. This is the idea of
parser combinators --- parsers are, in principle, functions. The user defines a
library with some basic ``building blocks'' or primitives and some combining
forms that let them put these things together. Parsers wrote in this fashion
look very similar to the grammars that are written to describe languages
\cite{fokker-1995,leijen-2001}.

\subsection{Abstract syntax tree} \label{sbs:ast}

\emph{Abstract syntax trees} (AST) are data structures that represent the
structure of program code, and are the result of the syntax analysis phase of a
program. The design of an AST is usually tightly linked with the design of a
parser and its expected features \cite{aho-2019}.

The definition of an abstract syntax tree for EBNF is closely related to the
definition of Parsing expression grammars discussed in Section~\ref{sbs:pegs}.
Like PEGs, EBNF provides such constructs as sequences, alternatives, optionals,
and repeated expressions. Additionally, EBNF can represent \emph{factors}, which
are repeated expressions with a number of repetitions specified, as well as
\emph{exceptions}, which are excluding alternatives that impose certain
restrictions on expressions. All EBNF constructs can be represented as the nodes
of the AST, which are listed in Table~\ref{tab:ast-nodes}.

\begin{longtable}[c]{@{}lm{9cm}@{}}
  \caption{Node types for the abstract syntax tree of EBNF.}
  \label{tab:ast-nodes}\\
  \toprule
  Node name           & Normal representation \\* \midrule
  \endfirsthead
  %
  \endhead
  %
  \endfoot
  %
  \endlastfoot
  %
  Alternative  & ``\texttt{$e_1$ | $e_2$ | $\dots$ | $e_n$}'', where $e_i$ is an AST node \\
  Sequence     & ``\texttt{$e_1$, $e_2$, $\dots$, $e_n$}'', where $e_i$ is an AST node \\
  Optional     & ``\texttt{[ $e$ ]}'', where $e$ is an AST node \\
  Repeated     & ``\texttt{\{ $e$ \}}'', where $e$ is an AST node \\
  Factor       & ``\texttt{$n$ * $e$}'', where $n$ is an integer and $e$ is an AST node \\
  Exception    & ``\texttt{$e_1$ * $e_2$}'', where $e_1$ and $e_2$ are AST nodes \\
  Non-terminal & A non-terminal token \\
  Terminal     & A terminal token \\
  Special      & A special token \\
  Empty        & The empty string $\varepsilon$ \\* \bottomrule
\end{longtable}

\subsection{The parsing process}

Figure~\ref{fig:parsing-process} shows the entire parsing process, starting with
the textual representation of the EBNF grammar. The lexical analyser produces a
string of tokens, which are then provided to the syntactic analyser, which
produces an abstract syntax tree, or some sort of an intermediate representation
of the abstract syntax tree. The semantic analyser analyses the intermediate
representation and produces a final AST, which represents the grammar the user
provided. At any point, the lexer, parser, or the processor may produce an error
in the form it's been given to.

\begin{figure}[H]
  \centering
  \begin{tikzpicture}
    \node (input) [cylinder,shape border rotate=90,draw,minimum height=10mm,minimum width=2cm,label={EBNF grammar string}] {};
    \node (lexer) [rectangle,draw,below=1.5cm of input,text width=2cm,align=center] {Lexical analyser (lexer)};
    \node (parser) [rectangle,draw,right=3cm of lexer,text width=2cm,align=center] {Syntactic analyser (parser)};
    \node (preprocessor) [rectangle,draw,right=3cm of parser,text width=3cm,align=center] {Semantic analyser (preprocessor)};
    \node (output) [cylinder,shape border rotate=90,draw,above=1.5cm of preprocessor,minimum height=10mm,minimum width=2cm,label={Output grammar}] {};
    \node (errors1) [below=1.5cm of lexer] {errors};
    \node (errors2) [below=1.5cm of parser] {errors};
    \node (errors3) [below=1.5cm of preprocessor] {errors};

    \draw [-triangle 45] (input) edge[right] node{characters} (lexer)
      (lexer) edge[below] node {tokens} (parser)
      (parser) edge[below] node[text width=2.5cm,align=center] {intermediate representation} (preprocessor)
      (preprocessor) edge[left] node {AST} (output)
      (lexer) edge (errors1)
      (parser) edge (errors2)
      (preprocessor) edge (errors3);
  \end{tikzpicture}
  \caption{The visualization of the EBNF parsing process.}
  \label{fig:parsing-process}
\end{figure}

\chapter{Analysis of similar solutions} \label{ch:analysis-of-similar-solutions}

\section*{Regex101}

Regex101 \cite{regex101} is a web-based tool that lets the user build regular
expressions and debug them in real-time. Users are able to construct their
expressions and see how it affects a data set all in a single screen at the same
time. The tool was made by Firas Dib, with contributions from many other
developers. It is said to be the largest regex testing service in the world.

\begin{figure}[ht]
  \centering
  \frame{\includegraphics[width=\textwidth]{regex101_matching.png}}
  \caption{Screenshot of the Regex101's matching functionality. The user
  provided the ``\texttt{\textbackslash{}s\textbackslash{}s+}'' regular
  expression, which matched every occurrence of two or more consecutive space
  characters in the test string.}
  \label{fig:regex101-matching}
\end{figure}

The tool is available to users in the form of a web application and can be
accessed from \url{https://regex101.com/}. It lets users build expressions fast
and debug them along the way, for example by pasting in a set of data and then,
through trial and error, building an expression with desired behavior.
Figure~\ref{fig:regex101-matching} shows a typical usage of Regex101 ---
matching a pasted test string to a regular expression. The tool makes it clear
if data is matching the expression or not, it even notifies users when the
expression is broken, and gives some explanation of why it is not working, as
seen in Figure~\ref{fig:regex101-error}.

\begin{figure}[ht]
  \centering
  \frame{\includegraphics[width=\textwidth]{regex101_error.png}}
  \caption{Screenshot of a basic error in the regular expression reported by
  Regex101.}
  \label{fig:regex101-error}
\end{figure}

These two feedback mechanisms are essential if the user is not accustomed to the
regular expression language, or just does not know how to construct the
appropriate expression yet. Being able to follow each step of the expression is
convenient when users are not able to figure out why something is not working,
or even if they are simply interested in discovering more about regular
expressions. Getting this immediate feedback without Regex101 would have
required users to write their expressions in a text editor and then run the code
separately, without getting much feedback about why it is or isn't working ---
Regex101.com eliminates this step.

Not only does Regex101.com make it easy to construct expressions, discover
errors, and even get accustomed to the syntax, it makes looking up a
metacharacter in regular expressions very easy. Always present, unless users
minimize it, the \emph{Quick Reference} tool lets them search for any token or
character they need. Finally, Regex101 lets users switch which \emph{flavor} or
version of regular expressions they want to use, as they may need to integrate a
regular expression into any number of other programming languages such as
Python, JavaScript, Golang, etc. Regex101 has the ability to switch the version
of the testing environment and will generate the code in that language for the
user to use in other projects.

\thisproject{} takes a lot of inspiration from Regex101 when it comes to
availability --- it's a web application, where all the work is done client-side.
The user does not have to install any additional software except the web
browser, the web application is accessed through a web page. In spite of its
similar nature, Regex101 cannot be a replacement of \thisproject{} --- it
focuses on various dialects of regular expressions rather than parsing EBNF and
generating parse trees --- it does, however, influence it with its accessibility
and functionalities.

\section*{Pest}

Pest \cite{pest} is a general purpose parser for the Rust \cite{rust}
programming language. It uses its own dialect of \emph{parsing expression
grammars} as input, similarly to \thisproject{}. Pest addresses the problem of
hand-written parsers in Rust, which in some circumstances can become hard to
maintain by their developers. Writing a specialized, domain-specific parser for
a language can become tedious, so developers usually gravitate towards using a
grammar-generated parser. This allows the developers to focus on the definition
of the language, rather than on the implementation of the parser. Grammars that
define the language offer better correctness guarantees, and any issues with the
language can be fixed declaratively in the grammar itself. Memory safety of Rust
additionally limits the damage bugs could cause. Static analysis and a precise
low-level implementation form a stable foundation on which major performance
tuning is feasible.

Developers of Pest, in spite of focusing mainly on the functionalities in the
Rust programming language, also provide an online editor available from the
browser on the Pest's homepage (\url{https://pest.rs/#editor}). The online
editor allows potential future users of Pest to experience the syntactic
characteristics of the Pest's dialect of PEGs and its error reporting
capabilities. The editor will inform the user about any syntactic errors, as
well as errors of semantic nature, such as undefined or left-recursive
production rules (seen in Figure~\ref{fig:pest-error}) and highlight them in
their exact locations.

\begin{figure}[ht]
  \centering
  \frame{\includegraphics[width=\textwidth]{pest_error.png}}
  \caption{Screenshot of Pest's online editor example error report.}
  \label{fig:pest-error}
\end{figure}

After parsing the grammar, Pest provides a window, which acts as an input
console, where users can type string that may or may not be parsed by the parser
generated by Pest. Additionally, users can choose the initial production rule
from a dropdown menu, which is an interesting choice, as opposed to
automatically detecting the initial rule based on the dependency graph of
production rules. The output window presents the parse tree, or the errors
encountered in the input string in case there are any. These features can be
seen in Figure~\ref{fig:pest-output}.

\begin{figure}[ht]
  \centering
  \frame{\includegraphics[width=\textwidth]{pest_output.png}}
  \caption{Screenshot of the input and output windows in Pest's online editor.}
  \label{fig:pest-output}
\end{figure}

While Pest focuses mainly on its integration with the Rust programming language,
this is not the case for \thisproject{}, which aims to provide all of its
functionality inside the web application. The online editor of Pest serves
largely as a ``try me'' feature for new users, rather than a reliable tool. The
editor lacks the standard editor features, such as autocompletion, code folding,
search and replace interface, as well as bracket and tag matching, all of which
\thisproject{} does provide. The parse tree in the output window is shown in a
basic textual form, without any interactive capabilities, which the user may
value. Finally, while Pest's grammar is based on PEGs, and is similar in nature
to EBNF, it is, in fact, not EBNF. The whole point of using EBNF and other
notations discussed in Section~\ref{sec:why-ebnf} is that they're standardized,
well-known, and accepted by the community; Pest's syntax is known only to users
of Pest and requires them to learn a new, non-standard language just for the
purpose of parsing grammars, where other, already established languages may have
sufficed.

\chapter{Design of the project} \label{ch:design-of-the-project}

This chapter introduces a specification for the application described in
Chapter~\ref{ch:problem-analysis}. The specification is presented in forms of a
list of functional and non-functional requirements in
Section~\ref{sec:requirements}, and user stories in
Section~\ref{sec:user-stories}. Section~\ref{sec:use-case-specification}
describes use cases and their descriptions structured in the form of a use case
diagram in the Unified Modeling Language, as well as their example scenarios,
also presented as activity and sequence diagrams. The chapter describes the
architecture of the system from the logical and physical perspective as
component and deployment diagrams in Section~\ref{sec:system-architecture}. The
chapter does not cover any class or database diagrams, as the implementation of
this project and its functionally-oriented nature, as opposed to being
object-oriented, does not require them. Finally, the chapter concludes with the
prototype and sketches of the user interface for the web application in
Section~\ref{sec:interface-prototype}.

\section{Requirements} \label{sec:requirements}

\subsection{Functional requirements}

Functional requirements shown in Table~\ref{tab:functional-requirements} define
functionalities and features of the system. Each requirement is associated with
a certain priority.

\begin{xltabular}{\textwidth}{@{}lL{3cm}Xl@{}}
  \caption{The functional requirements of the project, their features, and
  priorities.}
  \label{tab:functional-requirements}\\
  \toprule
  Id & Requirement & Features & Priority \\* \midrule
  \endfirsthead
  %
  \endhead
  %
  \endfoot
  %
  \endlastfoot
  %
  \emph{FR1} & Specifying the grammar & The user can specify the grammar of a
  given language in the EBNF notation by providing it in a textual form in
  a designated editor window. & high \\
  \addlinespace[0.5em] \emph{FR2} & Error reporting & The editor provides
  feedback about any syntactic or semantic\footnote{Such as production
  rule duplication or left recursion.} errors encountered during the parsing by
  highlighting the exact location of the error in the provided grammar. The user
  can then hover the mouse pointer over the highlighted area to read the error
  message. & high \\
  \addlinespace[0.5em] \emph{FR3} & Specifying the input string & The user can
  specify the input string in a designated editor window to check if it
  belongs to the language generated by the previously-defined grammar. & high
  \\
  \addlinespace[0.5em] \emph{FR4} & Visualizing the parse tree & The
  application visualizes the parse tree resulting from parsing the specified
  input string with the parser generated by the grammar defined by the user. &
  high \\
  \addlinespace[0.5em] \emph{FR5} & Syntax highlighting & The editor highlights
  parts of the specified grammar with a different syntactic meaning in a
  different manner with the use of multi-colored fonts. & medium \\
  \addlinespace[0.5em] \emph{FR6} & Autocompletion of non-terminals & The
  editor predicts the identifier of a non-terminal a user is typing by
  providing a list of possible non-terminals, which then can be chosen by the
  user. & low \\
  \addlinespace[0.5em] \emph{FR7} & Production rule folding & The editor
  provides the ability to hide and reveal a production rule of the grammar
  inside the editor window. & low \\
  \addlinespace[0.5em] \emph{FR8} & Search and replace interface & The user can
  search for any occurrences of a phrase in the editor window and possibly
  replace them with a different phrase. The search and replace functionality
  should also support regular expressions. & low \\* \bottomrule
\end{xltabular}

\subsection{Non-functional requirements}

Table~\ref{tab:non-functional-requirements} describes requirements of the
non-functional nature of the system, which focus on aspects of usability,
availability, and compatibility of the system.

\begin{xltabular}{\textwidth}{@{}Xl@{}}
  \caption{The non-functional requirements of the project and their priorities.}
  \label{tab:non-functional-requirements}\\
  \toprule
  Requirement & Priority \\* \midrule
  \endfirsthead
  %
  \endhead
  %
  \endfoot
  %
  \endlastfoot
  %
  The web application should be available 24 hours a day, 7 days a week. &
  medium \\
  \addlinespace[0.5em] Page loading time should be less than 1 second with
  internet download speed of 80\,Mbps. Parsing and checking times should both
  be less than 50 milliseconds. & high \\
  \addlinespace[0.5em] The application must work and display correctly in
  \begin{itemize}[noitemsep,nolistsep]
    \item Chrome version 86 or later,
    \item Safari version 14 or later,
    \item Edge version 86 or later,
    \item Firefox version 82 or later,
    \item Opera version 71 or later. \end{itemize} & high \\
  & medium \\
  \addlinespace[0.5em] The source code of the product should be open source
  and freely available for possible modification and redistribution. & high \\
  \addlinespace[0.5em] The project should include the documentation necessary
  for extension and maintenance of the system. & high \\
  \addlinespace[0.5em] The system should provide high degree of integrability
  with future components which extend the functionalities of the system. & high
  \\* \bottomrule
\end{xltabular}

\section{User stories} \label{sec:user-stories}

Stories in Table~\ref{tab:user-stories} are short descriptions of a feature told
from the perspective of the person who desires a new functionality in the
system.

\begin{xltabular}{\textwidth}{@{}lX@{}}
  \caption{The user stories.}
  \label{tab:user-stories}\\
  \toprule
  Id & User story \\* \midrule
  \endfirsthead
  %
  \endhead
  %
  \endfoot
  %
  \endlastfoot
  %
  \emph{US1} & As the user, I want to be able to paste the contents of my
  clipboard into the editor window in the application. \\
  \addlinespace[0.5em] \emph{US2} & As the user, I want to be able to type in
  the editor window with my keyboard. \\
  \addlinespace[0.5em] \emph{US3} & As the user, I want to be able to appreciate
  the multi-colored appearance of the text that represents the syntax that I
  provided. \\
  \addlinespace[0.5em] \emph{US4} & As the user, I want to be able to select a
  portion of the text in the editor window and copy it to the clipboard using a
  keyboard shortcut. \\
  \addlinespace[0.5em] \emph{US5} & As the user, I want to be able to hold the
  \emph{Alt} key on my keyboard to define multiple cursors in the editor window.
  \\
  \addlinespace[0.5em] \emph{US6} & As the user, I want to have the ability to
  autocomplete the non-terminal I am typing that has already been declared
  elsewhere in the code. \\
  \addlinespace[0.5em] \emph{US7} & As the user, I want to be able to hide any
  existing production rules that might appear too long, to increase the degree
  of clarity and readability of the grammar I'm working on. \\
  \addlinespace[0.5em] \emph{US8} & As the user, I want to be able to show any
  previously hidden production rules of the grammar. \\
  \addlinespace[0.5em] \emph{US9} & As the user, I want to have the ability to
  press a certain key combination on my keyboard that would allow me to type a
  specific phrase in the popup window, which would then find all the occurrences
  of that phrase in the editor window. \\
  \addlinespace[0.5em] \emph{US10} & As the user, I want to be able to provide a
  regular expression for the \emph{find} functionality that would allow me to
  find all occurrences of phrases that pattern match that specific regular
  expression. \\
  \addlinespace[0.5em] \emph{US11} & As the user, I want to be able to replace
  some of the occurrences of phrases found with the \emph{find} functionality
  with another phrase provided in a popup window. \\
  \addlinespace[0.5em] \emph{US12} & As the user, I want to be able to specify
  the initial production rule in the process of checking the input string
  against the grammar I provided. \\
  \addlinespace[0.5em] \emph{US13} & As the user, I want to be able to see
  errors in the syntax of the provided grammar in the form of underlined text in
  the location of where the errors actually occur. \\
  \addlinespace[0.5em] \emph{US14} & As the user, I want to have the ability to
  hover the mouse pointer over the underlined text to read the error message at
  that location. Alternatively, I want to be able to hover over the error
  indicator, which appears next to the line number. \\
  \addlinespace[0.5em] \emph{US15} & As the user, I want to be able to see the
  parse tree of the recognized input string that I provided. \\
  \addlinespace[0.5em] \emph{US16} & As the user, I want to have the ability to
  collapse any nodes in the visualized parse tree that might appear too long.
  \\* \bottomrule
\end{xltabular}

\section{Use case specification} \label{sec:use-case-specification}

\subsection{Use cases} \label{sbs:use-cases}

Figure~\ref{fig:use-case-diagram} shows the use case diagram of the system. Each
use case also presented in Table~\ref{tab:use-cases} along with a short
description.

\begin{figure}[H]
  \centering
  \resizebox{0.75\textwidth}{!}{\fontsize{9}{10}\input{images/use_case_diagram.pdf_tex}}
  \caption{The use case diagram.}
  \label{fig:use-case-diagram}
\end{figure}

\begin{xltabular}{\textwidth}{@{}lL{3cm}Xl@{}}
  \caption{Descriptions of the use cases.}
  \label{tab:use-cases}\\
  \toprule
  Id & Name & Description \\* \midrule
  \endfirsthead
  %
  \endhead
  %
  \endfoot
  %
  \endlastfoot
  %
  \emph{UC1} & Specifying the grammar & Allows the user to specify the grammar
  of a given language in the EBNF notation by providing it in a textual form in
  a designated editor window. \\
  \addlinespace[0.5em] \emph{UC2} & Specifying the input string & Allows the
  user to specify the input string in a designated editor window to check if it
  belongs to the language generated by the previously-defined grammar. \\
  \addlinespace[0.5em] \emph{UC3} & Interacting with the visualization & Allows
  the user to observe the visualized parse tree of the provided input string and
  interact with it by expanding and collapsing the tree nodes. \\
  \addlinespace[0.5em] \emph{UC4} & Changing the initial rule & Allows the user
  to specify the initial production rule used in the process of checking the
  provided input string against the defined grammar. \\* \bottomrule
\end{xltabular}

\subsection{Requirements traceability graph}

Figure~\ref{fig:rtg} presents the relationship between functional requirements,
user stories and use cases in the form of a requirements traceability graph. It
shows that every user story is connected with at least one functional
requirement and vice versa, and that every use case is associated with at least
one user story and vice versa.

\begin{figure}[H]
  \centering
  \pgfdeclarelayer{bg}
  \pgfsetlayers{bg,main}
  \resizebox{\textwidth}{!}{%
    \begin{tikzpicture}[xscale=1.2,yscale=3]
      \tikzstyle{every node}=[draw,circle,minimum size=1cm,inner sep=0pt,fill=white]

      \node at (-1.5,-1) (uc1) {\emph{UC1}};
      \node at (-0.5,-1) (uc2) {\emph{UC2}};
      \node at (0.5,-1)  (uc3) {\emph{UC3}};
      \node at (1.5,-1)  (uc4) {\emph{UC3}};

      \node at (-7.5,0) (us1)  {\emph{US1}};
      \node at (-6.5,0) (us2)  {\emph{US2}};
      \node at (-5.5,0) (us3)  {\emph{US3}};
      \node at (-4.5,0) (us4)  {\emph{US4}};
      \node at (-3.5,0) (us5)  {\emph{US5}};
      \node at (-2.5,0) (us6)  {\emph{US6}};
      \node at (-1.5,0) (us7)  {\emph{US7}};
      \node at (-0.5,0) (us8)  {\emph{US8}};
      \node at (0.5,0) (us9)  {\emph{US9}};
      \node at (1.5,0) (us10) {\emph{US10}};
      \node at (2.5,0) (us11) {\emph{US11}};
      \node at (3.5,0) (us12) {\emph{US12}};
      \node at (4.5,0) (us13) {\emph{US13}};
      \node at (5.5,0) (us14) {\emph{US14}};
      \node at (6.5,0) (us15) {\emph{US15}};
      \node at (7.5,0) (us16) {\emph{US15}};

      \node at (-3.5,1) (fr1) {\emph{FR1}};
      \node at (-2.5,1) (fr2) {\emph{FR2}};
      \node at (-1.5,1) (fr3) {\emph{FR3}};
      \node at (-0.5,1) (fr4) {\emph{FR4}};
      \node at (0.5,1)  (fr5) {\emph{FR5}};
      \node at (1.5,1)  (fr6) {\emph{FR6}};
      \node at (2.5,1)  (fr7) {\emph{FR7}};
      \node at (3.5,1)  (fr8) {\emph{FR8}};

      \begin{pgfonlayer}{bg}
        \draw (fr1) -- (us1);
        \draw (fr1) -- (us2);
        \draw (fr1) -- (us4);
        \draw (fr1) -- (us5);
        \draw (fr2) -- (us13);
        \draw (fr2) -- (us14);
        \draw (fr3) -- (us1);
        \draw (fr3) -- (us2);
        \draw (fr3) -- (us4);
        \draw (fr3) -- (us5);
        \draw (fr4) -- (us15);
        \draw (fr4) -- (us16);
        \draw (fr5) -- (us3);
        \draw (fr6) -- (us6);
        \draw (fr7) -- (us7);
        \draw (fr7) -- (us8);
        \draw (fr8) -- (us9);
        \draw (fr8) -- (us10);
        \draw (fr8) -- (us11);

        \draw (us1) -- (uc1);
        \draw (us2) -- (uc1);
        \draw (us3) -- (uc1);
        \draw (us4) -- (uc1);
        \draw (us5) -- (uc1);
        \draw (us6) -- (uc1);
        \draw (us7) -- (uc1);
        \draw (us8) -- (uc1);
        \draw (us9) -- (uc1);
        \draw (us10) -- (uc1);
        \draw (us11) -- (uc1);
        \draw (us12) -- (uc1);
        \draw (us13) -- (uc1);
        \draw (us1) -- (uc2);
        \draw (us2) -- (uc2);
        \draw (us3) -- (uc2);
        \draw (us4) -- (uc2);
        \draw (us5) -- (uc2);
        \draw (us6) -- (uc2);
        \draw (us7) -- (uc2);
        \draw (us8) -- (uc2);
        \draw (us9) -- (uc2);
        \draw (us10) -- (uc2);
        \draw (us11) -- (uc2);
        \draw (us13) -- (uc2);
        \draw (us14) -- (uc2);
        \draw (us12) -- (uc3);
        \draw (us15) -- (uc4);
        \draw (us16) -- (uc4);
      \end{pgfonlayer}
    \end{tikzpicture}
  }
  \caption{The requirements traceability graph.}
  \label{fig:rtg}
\end{figure}

\subsection{Use case scenarios} \label{sbs:use-case-scenarios}

Tables~\ref{tab:uc1-scenario}, \ref{tab:uc2-scenario}, \ref{tab:uc3-scenario},
and \ref{tab:uc4-scenario} describe the scenarios of each use case in the
system. Every scenario is defined by its pre-conditions, its post-conditions,
and a list of steps made by the system or the user required to complete it.

\begin{xltabular}{\textwidth}{@{}L{3cm}X@{}}
  \caption{Use case scenario of \emph{UC1} Specifying the grammar.}
  \label{tab:uc1-scenario}\\
  \toprule
  \endfirsthead
  %
  \endhead
  %
  \endfoot
  %
  \endlastfoot
  %
  Identifier & \emph{UC1} \\
  \addlinespace[0.5em] Name & Specifying the grammar \\
  \addlinespace[0.5em] Summary & Allows the user to specify the grammar of a
  given language in the EBNF notation by providing it in a textual form in a
  designated editor window. \\
  \addlinespace[0.5em] Pre-conditions & None. \\
  \addlinespace[0.5em] Post-conditions & The grammar has been correctly defined
  by the user with no syntactic errors. \\
  \addlinespace[0.5em] Main scenario &
  \begin{enumerate}[noitemsep,nolistsep,labelindent=0.5cm,align=right]
    \item [1.] The system shows a grammar editor window to the user.
    \item [2.] The user provides a syntactically and semantically correct
    definition of a grammar.
    \item [3.] The system shows an icon indicating no errors detected in the
    grammar.
    \item [] End of scenario.
  \end{enumerate} \\
  \addlinespace[0.5em] Alternative scenario &
  \begin{enumerate}[noitemsep,nolistsep,labelindent=0.5cm,align=right]
    \item [2a.1.] The user provides an invalid definition of a grammar.
    \item [2a.2.] The system highlights the text in the grammar editor window at
    the error location.
    \item [] Return to step 2.
  \end{enumerate} \\* \bottomrule
\end{xltabular}

\begin{xltabular}{\textwidth}{@{}L{3cm}X@{}}
  \caption{Use case scenario of \emph{UC2} Specifying the input string.}
  \label{tab:uc2-scenario}\\
  \toprule
  \endfirsthead
  %
  \endhead
  %
  \endfoot
  %
  \endlastfoot
  %
  Identifier & \emph{UC2} \\
  \addlinespace[0.5em] Name & Specifying the input string \\
  \addlinespace[0.5em] Summary & Allows the user to specify the input string in
  a designated editor window to check if it belongs to the language generated by
  the previously-defined grammar. \\
  \addlinespace[0.5em] Pre-conditions & None. \\
  \addlinespace[0.5em] Post-conditions & The input string has been correctly
  entered by the user. \\
  \addlinespace[0.5em] Main scenario &
  \begin{enumerate}[noitemsep,nolistsep,labelindent=0.5cm,align=right]
    \item [1.] The system shows a input string editor window to the user.
    \item [2.] The user provides a desired input string.
    \item [3.] A valid grammar has been provided by the user in the grammar
    editor window.
    \item [4.] The system shows the result of the checker in the result window.
    \item [] End of scenario.
  \end{enumerate} \\
  \addlinespace[0.5em] Alternative scenario &
  \begin{enumerate}[noitemsep,nolistsep,labelindent=0.5cm,align=right]
    \item [3a.1.] The user did not provide a valid grammar in the grammar editor
    window.
    \item [3a.2.] The system does not show a result of the checker.
    \item [] End of scenario.
  \end{enumerate} \\* \bottomrule
\end{xltabular}

\begin{xltabular}{\textwidth}{@{}L{3cm}X@{}}
  \caption{Use case scenario of \emph{UC3} Interacting with the visualization.}
  \label{tab:uc3-scenario}\\
  \toprule
  \endfirsthead
  %
  \endhead
  %
  \endfoot
  %
  \endlastfoot
  %
  Identifier & \emph{UC3} \\
  \addlinespace[0.5em] Name & Interacting with the visualization \\
  \addlinespace[0.5em] Summary & Allows the user to observe the visualized parse
  tree of the provided input string and interact with it by expanding and
  collapsing the tree nodes. \\
  \addlinespace[0.5em] Pre-conditions & The user has provided a valid definition
  of a grammar, as well as an input string, that belongs to the language
  generated by that grammar. \\
  \addlinespace[0.5em] Post-conditions & None. \\
  \addlinespace[0.5em] Main scenario &
  \begin{enumerate}[noitemsep,nolistsep,labelindent=0.5cm,align=right]
    \item [1.] The system shows the result of a checker in the form of a
    visualized parse tree.
    \item [2.] The user clicks the nodes of the parse tree to collapse or
    expand them.
    \item [] End of scenario.
  \end{enumerate} \\* \bottomrule
\end{xltabular}

\begin{xltabular}{\textwidth}{@{}L{3cm}X@{}}
  \caption{Use case scenario of \emph{UC2} Specifying the input string.}
  \label{tab:uc4-scenario}\\
  \toprule
  \endfirsthead
  %
  \endhead
  %
  \endfoot
  %
  \endlastfoot
  %
  Identifier & \emph{UC4} \\
  \addlinespace[0.5em] Name & Changing the initial rule \\
  \addlinespace[0.5em] Summary & Allows the user to specify the initial
  production rule used in the process of checking the provided input string
  against the defined grammar. \\
  \addlinespace[0.5em] Pre-conditions & The user has provided a valid definition
  of a grammar. \\
  \addlinespace[0.5em] Post-conditions & The initial production rule has been
  successfully changed to the desired one. \\
  \addlinespace[0.5em] Main scenario &
  \begin{enumerate}[noitemsep,nolistsep,labelindent=0.5cm,align=right]
    \item [1.] The system shows a button the current initial production rule
    written on top.
    \item [2.] The user clicks on the button.
    \item [3.] The system shows a dropdown menu with a list of all production
    rules defined in the provided grammar.
    \item [4.] The user clicks on an item of the list corresponding to the
    desired initial production rule.
    \item [5.] The system changes the identifier of the initial production rule
    on the button.
    \item [] End of scenario.
  \end{enumerate} \\* \bottomrule
\end{xltabular}

\subsection{Activity diagrams}

Figures~\ref{fig:uc1-activity-diagram}, \ref{fig:uc2-activity-diagram},
\ref{fig:uc3-activity-diagram}, and \ref{fig:uc4-activity-diagram} are the
graphical representations of use case scenarios defined in
Section~\ref{sbs:use-case-scenarios}, represented in the form of UML activity
diagrams.

\begin{figure}[H]
  \centering
  \resizebox{!}{5cm}{\footnotesize\input{images/uc1_activity_diagram.pdf_tex}}
  \caption{The activity diagram of \emph{UC1} Specifying the grammar.}
  \label{fig:uc1-activity-diagram}
\end{figure}

\begin{figure}[H]
  \centering
  \resizebox{!}{5cm}{\footnotesize\input{images/uc2_activity_diagram.pdf_tex}}
  \caption{The activity diagram of \emph{UC2} Specifying the input string.}
  \label{fig:uc2-activity-diagram}
\end{figure}

\begin{figure}[H]
  \centering
  \resizebox{!}{5cm}{\footnotesize\input{images/uc3_activity_diagram.pdf_tex}}
  \caption{The activity diagram of \emph{UC3} Interacting with the
  \label{fig:uc3-activity-diagram}
  visualization.}
\end{figure}

\begin{figure}[H]
  \centering
  \resizebox{!}{5cm}{\footnotesize\input{images/uc4_activity_diagram.pdf_tex}}
  \caption{The activity diagram of \emph{UC4} Changing the initial rule.}
  \label{fig:uc4-activity-diagram}
\end{figure}

\subsection{Sequence diagram}

Figure~\ref{fig:sequence-diagram} shows a sequence diagram, that is in essence
an interaction diagram that details how operations in the system are carried out
and visualizes interactions between objects and components. It captures
interactions from every use case, all of which were defined in
Section~\ref{sbs:use-cases}.

\begin{figure}[H]
  \centering
  \resizebox{0.65\textwidth}{!}{\footnotesize\input{images/sequence_diagram_1.pdf_tex}}
  \caption{The sequence diagram representing the specification of the grammar.}
  \label{fig:sequence-diagram}
\end{figure}

\newpage

\section{System architecture} \label{sec:system-architecture}

\subsection{Logical architecture}

Logical architecture of a system can be represented by UML component diagrams,
which focus on a system's components that are often used to model the static
implementation view of a system. A component diagram divides the system into
various high levels of functionality. Each component is responsible for one
distinct aim within the entire system and only communicates with other necessary
elements on the so-called need-to-know basis. In a system with a
functional-oriented approach it is more suitable for modelling interactions
between components. The logical architecture of \thisproject{} is modelled with
such a diagram and can be seen in Figure~\ref{fig:logical-architecture}.

\begin{figure}[H]
  \centering
  \resizebox{0.8\textwidth}{!}{\small\input{images/logical_architecture.pdf_tex}}
  \caption{The logical architecture of the system represented with a UML
  component diagram.}
  \label{fig:logical-architecture}
\end{figure}

\subsection{Physical architecture}

A deployment diagram in the Unified Modeling Language models the physical
deployment of artifacts on nodes and can represent a physical architecture of
a system. Diagram shown on Figure~\ref{fig:physical-architecture} visualizes
the architecture for \thisproject{}.

\begin{figure}[H]
  \centering
  \resizebox{\textwidth}{!}{\footnotesize\input{images/physical_architecture.pdf_tex}}
  \caption{The physical architecture of the system represented with a UML
  deployment diagram.}
  \label{fig:physical-architecture}
\end{figure}

\section{Interface prototype} \label{sec:interface-prototype}

Because of the visual nature of the web application, a prototype of the user
interface design should be established to allow the developer to plan out the
implementation of the front-end aspect of the application. \thisproject{}, being
a rather simple application, will consist of a single view (as seen in
Figure~\ref{fig:interface-prototype}), which is made out of several components.

\begin{figure}[H]
  \centering
  \begin{tikzpicture}
    \draw (0,0) rectangle (15,8);
    \draw[black,fill=black] (14.5,7.5) circle (0.25);
    \draw (0.25,0.25) rectangle (7.375,7) node[pos=.5] {\footnotesize\ttfamily Grammar window};
    \draw (7.625,6.25) rectangle (12,7) node[pos=.5] {\footnotesize\ttfamily Selection box};
    \draw (7.625,3.625) rectangle (14.75,6) node[pos=.5] {\footnotesize\ttfamily Test string window};
    \draw (7.625,0.25) rectangle (14.75,3.375) node[pos=.5] {\footnotesize\ttfamily Visualization window};
    \node[] at (2.4,7.5) {\Large\ttfamily\thisproject{}};
  \end{tikzpicture}
  \caption{The user interface sketch.}
  \label{fig:interface-prototype}
\end{figure}

\chapter{Implementation of the project} \label{ch:implementation-of-the-project}

\section{Software environment}

\subsection{Used technologies} \label{sbs:used-technologies}

\subsubsection*{Visual Studio Code}

Visual Studio Code \cite{vs-code} is a free, open-source text editor made by
Microsoft for Windows, Linux and macOS. It is designed to write code and
features syntax highlighting, code completion, snippets, code refactoring, and
code debugging. The editor can be used with various programming languages, and
supports extensions, which can be installed through a central repository called
VS Code Marketplace available in the editor itself. The extensions may provide
feature additions to the editor, as well as the support for various programming
languages in the form of code linters, static code analysers, and debuggers. The
editor is integrated with various version control systems, including Git and
Subversion

According to the 2019 Developers Survey of Stack Overflow, Visual Studio Code
ranked \#1 among the top popular developer tools, with $50.7\,\%$ of the 87317
respondents using it \cite{stack-overflow-insights-2019}.

The extensions for the editor are made by the members of Visual Studio Code
community. Two main extensions used by the author to develop the project were:
\begin{description}
  \item[rust-analyzer] \cite{rust-analyzer} --- An implementation of the
  Language Server Protocol for the Rust programming language, which provides
  features such as code completion, messages for syntax and semantic errors,
  code actions, diagnostics, ``go to definition'' and other editor actions.
  \item[Svelte for VS Code] \cite{svelte-for-vs-code} --- An implementation of
  the Language Server Protocol for the Svelte framework. The extension provides
  diagnostic messages for warnings and errors, support for Svelte pre-processors
  that provide source maps, as well as the support for Svelte-specific
  formatting (via prettier-plugin-svelte). Besides the Svelte language, the
  extension supports features such as hover info, messages for syntax and lint
  errors, and autocompletions for HTML, CSS/SCSS/LESS, as well as TypeScript and
  JavaScript.
\end{description}
The extensions have not proven to be crucial for the development of the project,
but were an excellent addition to the workflow.

Besides the editor extensions, the terminal integrated with Visual Studio Code
editor has been a valuable feature throughout the development process. The
command line is a substantial factor in the development of modern applications,
so a built-in terminal window allows the user to swiftly switch between the code
editor and the command line.

The support for the Git version control system has also been advantageous when
it comes to code editing. Every added, modified, or removed line of code is
highlighted with an appropriate color in the code editor. This greatly improves
the readability of the code, and allows the users to revert the code to its
previous state right from the editor without any external tools.

\subsubsection*{Git}

Git \cite{git} is a free and open source distributed version control system. It
has been a major part of the development process for the project, and has been
used mainly as a tool for keeping track of the changes made to the source code
and for integrating features in a smooth, non-disruptive manner.

Git supports branching and merging, which means that several project features
may be implemented simultaneously and independently on separate \emph{branches}
and then \emph{merged} into the main project. Every major code change has been
implemented on a designated branch and was merged into the main branch only
after a thorough testing process --- this has made parallel development very
easy, by isolating new development from finished work. This style of a workflow
is known as GitFlow, made popular by Vincent Driessen \cite{git-flow}, it has
shown itself to be very effective for projects of any scale. Efficient switching
between different versions of project files enables developers to work
effectively on the project. Git includes specific tools for visualizing and
navigating a non-linear development history. The author used \cite{chacon-2014}
as a reference for using the tool.

Git is now the most widely used source-code management tool, with $87.2\,\%$ of
the 74298 respondents of the 2018 Developers Survey of Stack Overflow reporting
that they use Git as their primary source-control system
\cite{stack-overflow-insights-2018}.

The main client of Git used in the project was the command-line tool on the
Ubuntu operating system running on Windows Subsystem for Linux.
Figure~\ref{fig:git} shows an example of GitFlow's \emph{feature branches} and
changes in the project repository in the Git version control system.

\begin{figure}[h]
  \centering
  \frame{\includegraphics[width=\textwidth]{git.png}}
  \caption{Screenshot of the command-line interface of the Git version control
  system.}
  \label{fig:git}
\end{figure}

\subsubsection*{GitHub}

GitHub \cite{github} is a for-profit company owned by Microsoft that offers a
cloud-based Git repository hosting service. As a company, GitHub makes money by
selling hosted private code repositories, as well as other business-focused
plans that make it easier for organizations to manage team members and security.
The author used the free GitHub plan as the service for hosting the project's
Git repository. Having the source code on an external server protected the
project against data loss and allowed the developer to work on the project from
any device at any convenient time.

In addition to using GitHub as a hosting service, one can also exploit its
project management features. Developers can define project boards related to the
project's code repository, which are simple kanban \cite{ahmad-2013} boards that
can help organize and prioritize the work. With projects, the developers have
the flexibility to manage boards for an entire project, or just for specific
features. Figure~\ref{fig:github-projects} shows an example project board from
\thisproject{}.

Project boards contain \emph{issues} and \emph{pull requests}, which can be
moved from one kanban column to another, indicating that some work is currently
``to do'', work in progress, or complete. These work ``cards'' contain
information about the author, assignees, the status, as well as simple textual
notes. The \emph{issues} are a way of reporting ideas, bugs, enhancements, or
tasks natively on GitHub. After completing the work on an issue, a developer
might make a \emph{pull request} to allow other developers on the project to
review and discuss the changes made to the code, and then deploy the changes by
``pulling'' the code to the central code repository.

\begin{figure}[h]
  \centering
  \frame{\includegraphics[width=\textwidth]{github_projects.png}}
  \caption{Screenshot of one of the project's kanban boards on GitHub.}
  \label{fig:github-projects}
\end{figure}

GitHub supports Continuous integration and Continuous Delivery functionalities
in form of \emph{Actions} and \emph{Pages}. GitHub Actions are a way to automate
and execute any software development workflow after any change to the code in
the repository. The user may set up many various actions for testing the changes
on many development environments and operating systems at the same time, as well
as building and deploying the code as a package or an arbitrary artifact. An
action consists of jobs, which are defined by a list of steps required to
execute them.

The GitHub Actions are used by the author to automate the testing and build
process on every change made to the code repository. The built application is
then deployed to a static site hosting service called GitHub Pages, which
integrates itself seamlessly with Actions and GitHub repositories. GitHub Pages
allows the user to host a website directly from a GitHub repository by combining
static HTML, CSS, JavaScript, and other files straight from a repository into
a website and publishing it on a \texttt{github.io} domain or a custom one.

\subsubsection*{Rust}

Rust \cite{rust} is the main programming language used in \thisproject{} --- it
powers the business logic part of the project. The language has been the most
loved language for four years in a row in the Stack Overflow's survey
\cite{stack-overflow-insights-2019}. The core idea of the language is memory
safety --- the language enforces certain rules checked at compile time, which
guarantee that the program is safe from bugs like dereferencing null or dangling
pointers, as well as making it difficult for the programmer to leak memory. Rust
does this through a system of ownership and borrowing. The language, besides the
safety, focuses on speed --- its design lets the developer design programs that
have the performance and control of a low-level language, but with the powerful
abstractions of a high-level language.

Rust's design borrows heavily from the one of Haskell \cite{haskell} --- both
languages feature a rich type system, both are immutable-by-default, avoid
mutation of shared references et cetera. Many developers tend to write Rust code
in a functional style and adhere to the principles of functional programming,
even though the language is multi-paradigm.

Without the need of a garbage collector, Rust projects are well-suited to be
used as libraries by other programming languages. The language over the last few
years has manifested itself in several distinct domains, including command-line
tools, networking, and embedded systems. Rust is supported on multiple operating
systems and targets multiple platforms, has notable documentation, a
user-friendly compiler with convenient error messages, and excellent tooling and
ecosystem. For referencing the language, the author used \cite{klabnik-2018},
which covers many features and concepts of Rust.

\subsubsection*{WebAssembly}

WebAssembly \cite{webassembly} (abbreviated \emph{Wasm}) is a safe, portable,
low-level code format designed for efficient execution and compact
representation. Its main goal is to enable high performance applications on the
Web, working alongside JavaScript, but not to be a replacement of it. Its design
makes it portable, compact, and execute at native or near-native speeds.
Although it has currently attracted attention in the JavaScript community and
other Web communities in general, Wasm does not make any assumptions about its
host environment. WebAssembly is supported as a target for many programming
languages, including C$\sharp$ via Blazor, C++ via EmScripten, and the main
language used in \thisproject{} --- Rust. The author compiles Rust code to
WebAssembly to be then used in a web environment for several reasons:
\begin{itemize}
  \item Code size is an important aspect, as the \texttt{.wasm} file must be
  downloaded over the network. Rust does not require a runtime and does not come
  with a garbage collector, allowing for small Wasm sizes, as there is no extra
  code included,
  \item Rust and WebAssembly integrate with existing JavaScript tooling --- the
  developer can continue using the tooling they are already familiar with, like
  npm and Webpack,
  \item JavaScript Web applications struggle to achieve and maintain stable
  performance. The code is required to be ran frequently, so Wasm can solve this
  kind of problem with greater memory and CPU performance at a lower level
  compared to the interpreter of JavaScript.
  \item The Rust language itself, with a strong package manager, high
  performance, memory safety, and zero-cost abstractions.
\end{itemize}

\subsubsection*{Cargo}

Cargo \cite{crates-io} is the Rust's package manager. It downloads the
dependencies of a Rust package and compiles them, making sure that the developer
will always be presented with a repeatable build. To achieve this, Cargo
includes two metadata files with various information about the packages, it
downloads and builds the dependencies, runs the Rust compiler with appropriate
parameters to build the package, and introduces various conventions to make
working with Rust packages more comfortable.

Rust provides first-class support for unit and integration testing, and Cargo
allows the developer to execute all tests with a single command. Additionally,
Cargo allows the developer to install extensions, which enhance the workflow and
the development process. One of extensions useful for the author was Clippy ---
a collection of lints to catch common mistakes and improve the Rust code.

\texttt{crates.io} is the primary package registry of the Rust community that is
used to discover and download packages. It is used by Cargo as a default way to
find requested packages. The project uses several dependencies, the most
important of which include:
\begin{description}
  \item[nom] \cite{nom,couprie-2015} --- A combinatory parsing library for the
  Rust programming language. It provides tools to build memory-safe parsers
  without compromising memory consumption or speed. To achieve this, Nom uses
  Rust's strong typing and memory safety very extensively to deliver performant
  parsers and provides various functions, macros, and traits to hide most of the
  error-prone details.

  Language parsers are often written manually for extra flexibility and
  performance. Nom can be (and has been) successfully used as a parser for
  prototyping a language. The resulting code of the parser is concise and
  resembles the grammar the developer would have written with other parser
  approaches. Parsers written with Nom are small and easy to maintain, as well
  as easy to test separately.
  \item[unicode-segmentation] \cite{unicode-segmentation} --- A library with a
  set of iterators which split strings on \emph{grapheme clusters}, \emph{words}
  or \emph{sentence boundaries}, according to the Unicode Standard Annex \#29
  \cite{unicode-standard-annex-29} rules.
  \item[wasm-bindgen] \cite{wasm-bindgen} --- A Rust library and CLI tool that
  makes high-level interactions between Wasm modules and JavaScript easier.
  Besides integers and floats, this library allows JavaScript and Wasm to
  communicate with strings, JS objects, classes, and other structures. Notable
  features of this project include:
  \begin{itemize}
    \item Importing JS functionality into Rust such as DOM manipulation, console
    logging, or performance monitoring.
    \item Working with rich types like strings, numbers, classes, closures, and
    objects.
    \item Automatically generating TypeScript bindings for Rust code being
    consumed by JS.
  \end{itemize}
  Wasm-bindgen only generates bindings for the JavaScript imports that are
  actually being used, and Rust functionality that is being exposed.
  \item[quickcheck] \cite{quickcheck} --- A property-based testing framework
  inspired by the QuickCheck framework for Haskell. The crate comes with the
  ability to randomly generate and shrink integers, floats, tuples, booleans,
  lists, strings, options and results. All QuickCheck needs is a property
  function --- it will then randomly generate inputs to that function and call
  the property for each set of inputs. If the property fails (whether by a
  runtime error like index out-of-bounds or by not satisfying the property), the
  inputs are "shrunk" to find a smaller counter-example.
  \item[criterion] \cite{criterion} --- Provides a powerful but simple way to
  measure software performance. It provides both a framework for executing and
  analyzing benchmarks and a set of driver functions that makes it easy to build
  and run benchmarks, and to analyse their results.
  \item[structopt] \cite{structopt} --- Parses command-line arguments by defining
  a struct. StructOpt combines the Clap library with a custom derive for marking
  a struct containing values that will be translated into specific command-line
  arguments based on their types and markers. Besides the desired arguments,
  StructOpt automatically generates the \emph{help} string, which can be invoked
  with the \verb|--help| flag, as well as a \verb|--version| flag for
  checking the version of the program.
\end{description}
All of the above dependencies are available under the MIT license.

\subsubsection*{Wasm-pack}

Wasm-pack \cite{wasm-pack} is a tool for building and working with
Rust-generated WebAssembly that the developer would like to interop with
JavaScript, in the browser, or with Node.js. The generated WebAssembly packages
then could be published to the npm registry, or otherwise used alongside any
JavaScript packages in workflows that the developer already uses, such as
Webpack or Rollup. The tool interoperates and utilizes the wasm-bindgen, another
tool, to provide a bridge between the types of JavaScript and Rust. It allows
JavaScript to call a Rust API with a string, or a Rust function to catch a
JavaScript exception. wasm-pack wraps the CLI portion of the wasm-bindgen tool.
This results in wrapping the WebAssembly module in JS wrappers which make it
easier to interact with the module. Wasm-bindgen supports both ES6 modules and
CommonJS and the developer can use wasm-pack to produce either type of package.

\subsubsection*{Svelte}

Svelte \cite{svelte} is a free and open-source front end JavaScript framework.
Svelte has its own compiler for converting app code into client-side JavaScript
at build time. The developer writes the components using HTML, CSS, and
JavaScript, which Svelte compiles during the build process into small standalone
JavaScript modules. Unlike frameworks like React and Vue, which do most of their
computations on the client-side while the app is running, Svelte turns that
process into a compile-time step that happens only when the developer builds the
app, which produces highly-optimized JavaScript. The compiler can ensure the
user's web browser does as little work as possible by statically analyzing the
templates of the components. Thanks to this approach, Svelte does not only
provide smaller application bundles and better performance, but also an
experience that is more friendly for people that are not familiar with the
modern tooling ecosystem. Svelte is particularly appropriate to tackle the
following situations:
\begin{itemize}
  \item Welcoming people with basic web development knowledge: Web developers
  with basic HTML, CSS, and JavaScript knowledge can easily learn Svelte in a
  short time and start building web applications.
  \item Interactive pages or complex state-based visualizations: If the user is
  building data-visualizations that require displaying a large number of
  elements, the performance gains thanks to no runtime overhead will ensure that
  user interactions are responsive.
  \item Web applications intended for low power devices: Applications built with
  Svelte have smaller bundle sizes, which is ideal for devices with slow network
  connections and limited processing power.
\end{itemize}

Svelte, being a compiler, can extend features of HTML, CSS, and JavaScript, and
generate optimal JavaScript code with no runtime overhead. To accomplish this,
Svelte extends core web technologies and only modifies them in very specific
circumstances and only in the context of Svelte components.

\subsubsection*{Rollup}

Rollup \cite{rollup} is a module bundler for JavaScript which is able to compile
smaller code modules into a complex library or application. It uses the
standardized ES module format for code, which lets the developer smoothly and
seamlessly connect individual functions and external modules. Rollup is able to
optimize modules for faster loading in modern browsers, or output a legacy
format of modules.

By splitting the project into smaller distinct pieces, the development process
is oftentimes more straightforward, since that generally removes unexpected
interactions and reduces the complexity of the issues the developer needs to
work on --- simply writing smaller projects isn't necessarily the answer to
that. Historically, JavaScript has not included the feature of modules as a core
feature in the language. This changed with the ES6 revision of JavaScript, which
includes syntax for importing and exporting functions and data so they can be
shared between separate scripts. The specification is now fixed, however, it is
implemented only in modern browsers. Rollup allows the user to write code using
the new module system, and will then compile it back down to existing supported
formats such as CommonJS modules, AMD modules, and IIFE-style scripts. This
means that the developer gets to write future-proof code.

Besides enabling the use of ES modules, Rollup also statically analyzes the
imported code and will exclude anything that isn't used in the code. This allows
the user to build on top of existing tools and modules without adding extra
dependencies or increasing the size of the project. Rollup only includes the
minimum, which results in lighter, faster, and less complicated libraries and
applications. Since this approach can utilise explicit \texttt{import} and
\texttt{export} statements, it is more powerful than running an automated
\emph{minifier} to discover unused variables in the compiled JavaScript code.

\subsubsection*{npm}

Node Package Manager \cite{npm} is a package manager for the JavaScript
programming language. It includes a command-line client, and an online database
of public and paid-for private packages called the npm registry. The registry
can be accessed through the client application, and the available packages can
also be browsed and searched via the npm website.

Npm provides several built-in scripts and allows users to define their own. An
npm script is a convenient way to bundle common shell commands for the project.
They are usually commands, or strings of commands, which would normally be
entered at the command line in order to do something with the application.
Scripts are stored in a project's configuration file, which means they're shared
amongst everyone using the codebase. They aid in automating repetitive tasks and
allow the user to have to learn fewer tools. Scripts also make sure that every
programmer is using the same command with the same parameters. Common use cases
for npm scripts include building the project, starting a development server,
compiling CSS, linting, or minifying.

The project is dependent on several npm packages:
\begin{description}
  \item[CodeMirror] \cite{codemirror} --- A versatile text editor implemented in
  JavaScript for the browser. It is specialized for editing code and includes
  several built-in language modes and addons that implement more advanced
  functionalities. A rich programming API and a CSS theming system are available
  for customizing CodeMirror to fit the needs of the user, as well as making
  extensions with new functionalities. CodeMirror is used in the developer tools
  for Firefox, Chrome, and Safari, in Light Table, Adobe Brackets, Bitbucket,
  and many other projects.

  CodeMirror supports a wide variety of configurations --- the basic version of
  the editor without any addons provides the support for over 100 languages,
  autocompletion, code folding, configurable keybindings, search and replace
  interface, bracket and tag matching, support for split view, linter
  integration, various themes, and many more.
  \item[svelte-tree] \cite{svelte-tree} --- A tree-like view component for
  Svelte. The component has the ability to display a collapsable tree structure
  based on a provided tree of JavaScript objects with custom \texttt{name} and
  \texttt{children} properties. The component provides a slot space to display
  custom nodes, which will give the tree node DOM/components the access to the
  nodes being rendered.
\end{description}
The configuration file also lists dependencies for the development process,
which can be divided into several categories:
\begin{itemize}
  \item Rollup and its plugins
  \begin{itemize}[noitemsep,nolistsep]
    \item \texttt{rollup},
    \item \texttt{@rollup/plugin-alias},
    \item \texttt{@rollup/plugin-commonjs},
    \item \texttt{@rollup/plugin-node-resolve},
    \item \texttt{@wasm-tool/rollup-plugin-rust},
    \item \texttt{rollup-plugin-copy},
    \item \texttt{rollup-plugin-livereload},
    \item \texttt{rollup-plugin-svelte},
    \item \texttt{rollup-plugin-terser}
  \end{itemize}
  \item Svelte and its plugins
  \begin{itemize}[noitemsep,nolistsep]
    \item \texttt{svelte},
    \item \texttt{svelte-check},
    \item \texttt{svelte-loader},
    \item \texttt{svelte-preprocess}
  \end{itemize}
  \item Miscellaneous dependencies
  \begin{itemize}[noitemsep,nolistsep]
    \item \texttt{gh-pages},
    \item \texttt{rimraf},
    \item \texttt{sirv-cli}
  \end{itemize}
\end{itemize}

\newpage

\subsection{Project structure} \label{sbs:project-structure}

\begin{figure}[H]
  \centering
  \begin{tikzpicture}[%
    grow via three points={one child at (0.5,-0.7) and
    two children at (0.5,-0.7) and (0.5,-1.4)},
    edge from parent path={(\tikzparentnode.south) |- (\tikzchildnode.west)}]
    \tikzstyle{every node}=[draw=black,thick,anchor=west]
    \node {parser-parser}
      child { node {app}
        child { node {editor}}
        child { node {wasm}}
      }
      child [missing] {}
      child [missing] {}
      child { node {base}
        child { node {checker}}
        child { node {ast}}
      }
      child [missing] {}
      child [missing] {}
      child { node {cli}}
      child { node {ebnf}
        child { node {lexer}}
        child { node {parser}}
        child { node {preprocessor}}
      }
      child [missing] {}
      child [missing] {}
      child [missing] {};
  \end{tikzpicture}
  \caption{A simplified directory tree of the \thisproject{} project.}
  \label{fig:file-tree}
\end{figure}

The project structure of \thisproject{}, visualized in
Figure~\ref{fig:file-tree}, consists of several components:
\begin{description}
  \item[app] --- An npm package containing the web application written primarily
  with Svelte. The component also defines a simple Rust crate, which acts as a
  link between the \emph{base} and \emph{ebnf} crates and compiles them to
  WebAssembly. The structure of the web application is described in detail in
  Section~\ref{sec:web-based-application},
  \item[base] --- Rust crate containing the definition of the AST along with the
  \emph{checker} module, which will be described in detail in
  Section~\ref{sbs:grammar-processing},
  \item[cli] --- The auxillary command-line application written in Rust,
  explored further in Section~\ref{sec:command-line-application},
  \item[ebnf] --- The core business logic of the application in the form of a
  Rust crate. This crate is further divided into modules: the lexical,
  syntactic, and semantic analysers, all of which will be discussed in
  Sections~\ref{sbs:lexical-analyser}, \ref{sbs:syntactic-analyser}, and
  \ref{sbs:semantic-analyser}.
\end{description}

\section{Business logic} \label{sec:business-logic}

\subsection{Domain modelling} \label{sbs:domain-modelling}

Domain Modeling is a way to describe and model entities and the relationships
between them, which as a whole describe the problem domain space. Types can be
utilized to describe the domain in a detailed way. Oftentimes, types can even be
used to encode business rules so that the developer cannot write invalid code.
Static type checking can be used as an immediate unit test --- making sure that
the code is valid at compile time. Types are the rules that describe what can
happen in the domain and could be utilized to prevent anyone else from putting
the system in a state invalid to the domain. Forcing illegal states to be
unrepresentable is the idea of statically proving that all values in the runtime
correspond to valid objects in the business domain. That makes the code much
easier to reason about and gives the developer confidence that the business
rules are being respected.

If the logic of the system is represented by types, it is automatically
self-documenting, and any changes to the business rules will immediately produce
breaking changes, which is generally a welcome feature. This way the programmer
can encode business requirements by writing code and, in the development
process, produce documentation enforced by the compiler.

Using algebraic data types is a powerful technique for designing with types and
making illegal states unrepresentable. Constructs such as sum types and product
types provide us with an expressive method of modelling the business rules.
This method also allows the developer to utilize property-based testing ---
letting the computer generate test cases.

For modelling the domain, the author will use the Haskell \cite{haskell}
programming language, with its expressive data types and highly reusable
abstractions, as well as a concise syntax. However, modelling based on algebraic
data types is also practical for other languages with complex enough type
systems --- Rust, used as the main language in \thisproject{}, is one example of
such a language. Please note that the author uses Haskell only in the thesis
document --- the project and its implementation uses the Rust programming
language.

\subsubsection*{Token type definition}

\begin{wraplisting}{l}{0.4\textwidth}
  \inputminted[fontsize=\small,frame=lines,breaklines,linenos]{haskell}
  {listings/token.hs}
  \caption{Definition of the \texttt{Token} type in Haskell.}
  \label{lst:token}
  \vspace{-2cm}
\end{wraplisting}

The tokenization process, described in Section~\ref{sec:lexing}, converts a
stream of characters into a stream of tokens. A set of valid tokens can be
represented as a sum type of all individual token types, shown in
Listing~\ref{lst:token}. Several type constructors carry additional information
about the token:
\begin{description}
  \item[\texttt{Non-terminal}] is specified by the textual form of the
  meta-identifier represented by the \texttt{String} type,
  \item[\texttt{Terminal}] is specified by the contents of the terminal in the
  form of a \texttt{String},
  \item[\texttt{Special}] carries with it exact contents of the special
  sequence specified in the grammar, to be processed further,
  \item[\texttt{Integer}] is specified by an actual numeric value encoded as
  Haskell's \texttt{Integer} type.
\end{description}

An equivalent of such a type in Rust could be defined as an \emph{enumeration},
also referred to as an \emph{enum}. Enums allow the developer to define a type
by enumerating its possible variants. Enums variants in Rust, besides the kind
of a variant, can also carry additional data, either in the form of a tuple, or
a record. This way, the \texttt{Nonterminal}, \texttt{Terminal},
\texttt{Special}, and \texttt{Integer} variants can have an associated value
built right into the type system.

\newpage

\subsubsection*{Grammar type definition}

The definition of the abstract Syntax Tree for EBNF in Section~\ref{sbs:ast}
gives an intuition for the definition of types related to grammars. The
definition of the \texttt{Expression} type, shown is Listing~\ref{lst:ast}, is
recursive, meaning it has another instance of the enumeration as the associated
value for one or more of the enumeration variants. In this case, any instance of
\texttt{Alternative}, \texttt{Sequence}, \texttt{Optional}, \texttt{Repeated},
\texttt{Factor}, or \texttt{Exception} is considered a \texttt{branch node} in
the AST, where, on the other hand, any instance of \texttt{Nonterminal},
\texttt{Terminal}, \texttt{Special}, or \texttt{Empty} is a \texttt{leaf node}.

\begin{listing}[H]
  \inputminted[fontsize=\small,frame=lines,breaklines,linenos]{haskell}
  {listings/ast.hs}
  \caption{Definition of the types related to the AST in Haskell.}
  \label{lst:ast}
\end{listing}

The \texttt{Expression Expression [Expression]} construct indicates a list of
\texttt{Expression}s that contains at least two elements at all times. This way,
the information about the minimum size of a list is built right into the type
system. Such a construct could be abstracted away into its own definition, but
in the process the \texttt{Expression} type would lose its pattern matching
capabilities.

\subsection{Lexical analyser} \label{sbs:lexical-analyser}

The lexical analyser, also known as \emph{the lexer} or \emph{the tokenizer},
performs the tokenization described in Section~\ref{sec:lexing}. A simplified
version of the EBNF tokenizer could be modelled as a Deterministic Finite-state
Automaton (DFA) (see Figure~\ref{fig:lexer-dfa}). This version, however, would
not support nested comments defined in the specification --- any nested
structure cannot be tokenized with the use of regular languages, and those are,
in fact, equivalent to DFAs.

The implementation does not follow the pure DFA approach. Instead, the lexer
stores the current state of the tokenization process in various forms. For the
purpose of tokenizing comments, the lexer remembers the \emph{nest level}, which
essentially counts the number of recursively-nested comments, and ends the
comment only when the nest level reaches zero.

The lexer's main control flow is a simple infinite loop followed by a pattern
match, whose cases are the initial characters of each token type, and each
case may be a loop to consume the rest of the token and return its type.
Every token is preceded by the ``whitespace-comment-whitespace'' search, which
skips any whitespace characters and (possibly nested) comments. Any whitespace
inside integers or meta-identifiers is handled in the corresponding token loop.

\begin{figure}[H]
  \centering
  \resizebox{\textwidth}{!}{%
  \begin{tikzpicture}
    \tikzset{
      ->, % makes the edges directed
      node distance=3cm,
      every state/.style={thick, fill=white},
      initial text=$ $,
      every initial by arrow/.style={-triangle 45},
      initial distance=4cm
    }

    \node[state, initial] (1) {$1$};
    \node[state, above of=1, accepting] (2) {$2$};
    \node[state, above left of=2] (3) {$3$};
    \node[state, above left of=1] (5) {$5$};
    \node[state, left of=3] (4) {$4$};
    \node[state, above of=2, accepting] (6) {$6$};
    \node[state, above right of=2, accepting] (7) {$7$};
    \node[state, above right of=6] (8) {$8$};
    \node[state, right of=2] (9) {$9$};
    \node[state, right of=9] (10) {$10$};
    \node[state, above right of=9] (11) {$11$};
    \node[state, above right of=10] (12) {$12$};
    \node[state, above of=12, accepting] (13) {$13$};
    \node[state, below of=10] (14) {$14$};
    \node[state, above right of=14, accepting] (15) {$15$};
    \node[state, below left of=1, accepting] (18) {$18$};
    \node[state, below of=14, accepting] (19) {$19$};
    \node[state, above right of=19, accepting] (20) {$20$};
    \node[state, below of=19, accepting] (21) {$21$};
    \node[state, above right of=21] (22) {$22$};
    \node[state, left of=21, accepting] (16) {$16$};
    \node[state, below right of=18, accepting] (17) {$17$};

    \draw[-triangle 45,every loop/.append style={-triangle 45}] (1) edge[right] node{\texttt{(}} (2)
          (2) edge[right] node{\texttt{*}} (3)
          (3) edge[above] node{$\Sigma - \{\texttt{)}\}$} (4)
          (4) edge[loop above] node{$\Sigma - \{\texttt{*}\}$} (4)
          (4) edge[left, bend right] node{\texttt{*}} (5)
          (5) edge[right, bend right] node{$\Sigma - \{\texttt{)}\}$} (4)
          (5) edge[above] node{\texttt{)}} (1)

          (2) edge[right] node{\texttt{/}} (6)
          (2) edge[above] node{\texttt{:}} (7)
          (6) edge[above] node{\texttt{)}} (8)
          (7) edge[right] node{\texttt{)}} (8)

          (1) edge[above] node{\texttt{"}} (9)
          (1) edge[above, bend right=14] node{\texttt{'}} (10)
          (9) edge[left] node{$\Sigma - \{\texttt{"}\}$} (11)
          (10) edge[left] node{$\Sigma - \{\texttt{'}\}$} (12)
          (11) edge[loop above] node{$\Sigma - \{\texttt{"}\}$} (11)
          (12) edge[loop below] node{$\Sigma - \{\texttt{'}\}$} (12)
          (11) edge[above] node{\texttt{"}} (13)
          (12) edge[right] node{\texttt{'}} (13)

          (1) edge[above, bend right=14] node{\texttt{?}} (14)
          (14) edge[loop below] node{$\Sigma - \{\texttt{?}\}$} (14)
          (14) edge[above] node{\texttt{?}} (15)

          (1) edge[right] node{$D$} (16)
          (16) edge[loop left] node{$D \cup W$} (16)

          (1) edge[right] node{$A$} (17)
          (17) edge[loop left] node{$A \cup N \cup W$} (17)

          (1) edge[left] node{\texttt{\{}, \texttt{\}}, \texttt{[}, \texttt{]},
          \texttt{)}, \texttt{,}, \texttt{;}, \texttt{=}, \texttt{-},
          \texttt{*}, \texttt{|}, \texttt{!}} (18)
          (1) edge[above] node{\texttt{/}} (19)
          (19) edge[right] node{\texttt{)}} (20)
          (1) edge[above] node{\texttt{:}} (21)
          (21) edge[right] node{\texttt{)}} (22);
  \end{tikzpicture}
  }
  \caption{The DFA representation of the lexer. Note that this DFA does not
  support nested comments. $\Sigma$ is the set of all characters, $A$ is the set
  of all alphabetic characters, $N$ is the set of all numeric characters, $D$ is
  the set of all ten decimal digits, and $W$ is the set off all whitespace
  characters. Set notation on the transitions is omitted.}
  \label{fig:lexer-dfa}
\end{figure}

The header of the \texttt{lex} function is
\begin{minted}[fontsize=\small,breaklines]{rust}
fn lex(string: &str) -> Result<Vec<Spanned<Token>>, Spanned<Error>>
\end{minted}
Please note the return type of the function, which is either a \texttt{Vec} of
\texttt{Spanned} \texttt{Token}s in case the tokenizer succeeded, or an
\texttt{Error}, which is defined as
\begin{listing}[H]
  \begin{minted}[fontsize=\small,breaklines]{rust}
enum Error {
  InvalidSymbol(String),
  UnterminatedSpecial,
  UnterminatedComment,
  UnterminatedTerminal,
  EmptyTerminal,
}
  \end{minted}
\end{listing}
\noindent which encodes every possible error the tokenizer can report. The
\texttt{InvalidSymbol} case contains the additional information about the actual
invalid symbol.

The whole tokenization process is preceded by the procedure of splitting the
input string into individual Unicode graphemes according to the Unicode Standard
Annex \#29 \cite{unicode-standard-annex-29} rules with the use of the
unicode-segmentation crate. As each grapheme may consist of several characters,
it is encoded a string, which means the lexer cannot use native functions for
checking if a character is whitespace, alphabetic, alphanumeric, or a digit ---
these have to be defined separately with graphemes in mind.

The code related to the lexer is contained in the \texttt{lexer} module and is
split into several files:
\begin{description}
  \item[\texttt{mod.rs}] is the main module file with the core business logic of
  the lexer and contains the \texttt{scan} and \texttt{lex} functions, as well
  as the utility functions related to graphemes,
  \item[\texttt{token.rs}] contains the definition of the \texttt{Token} type,
  \item[\texttt{error.rs}] contains the definition of the \texttt{lexer::Error}
  type,
  \item[\texttt{tests.rs}] holds unit tests related to lexical analysis, which
  will be later discussed in Section~\ref{sec:testing}.
\end{description}

\subsection{Syntactic analyser} \label{sbs:syntactic-analyser}

The syntactic analysis phase, also known as the parsing phase (described in
Section~\ref{sec:parsing}) is conducted by the parser module. The goal of this
thesis is not implementing a parser combinator library from scratch, so instead
of writing every parser by hand, \thisproject{} uses Nom (see
Section~\ref{sbs:used-technologies}) as the parser combinator library of
choice, which ships with a large number of utility parsers and combinators
already defined. These are in turn combined into more sophisticated parsers that
are capable of parsing certain EBNF-like structures.

\begin{listing}[H]
  \inputminted[fontsize=\small,frame=lines,breaklines,linenos]{rust}
  {listings/parser.rs}
  \caption{The \texttt{sequence} parser.}
  \label{lst:parser}
\end{listing}

For instance, the parser seen in Listing~\ref{lst:parser} parses the sequence of
\texttt{term}s separated by commas with the use of the
\texttt{separated\_list\_1} native to Nom. It also utilizes the \texttt{map}
combinator to transform the result into an appropriate type --- it returns the
\texttt{Sequence} of terms in case it parsed two or more terms, or just the
single term in case it was the only one in the sequence. Note that this parser
uses the \texttt{term} and \texttt{concatenation\_symbol} parsers defined
in a similar fashion.

All the more complicated parsers eventually use the so-called \emph{literals}
--- the most simple parsers capable of parsing single tokens. Every token from
the \texttt{Token} type (defined in Section~\ref{sbs:domain-modelling}) has an
equivalent literal parser.

All parsers in Nom are basically functions, which most generally take a string
as the input, and return an \texttt{IResult} --- either the parsed \emph{thing}
along with the rest of the unconsumed input, or an error of some sort indicating
that the parser failed. In the case of \thisproject{}, however, the input is not
a string of characters, but a string of tokens. Fortunately, Nom, thanks to its
high extensibility, can parse any type as long as that type implements certain
traits. This resulted in the definition of the \texttt{Tokens} data structure,
which encapsulates the \texttt{\&[Spanned<Token>]} type --- a slice of a sequence
of (spanned) tokens. \texttt{Tokens} implements such traits as
\begin{itemize}[noitemsep]
  \item \verb|InputLength|
  \item \verb|InputIter|
  \item \verb|InputTake|
  \item \verb|UnspecializedInput|
  \item \verb|Compare<Tokens>|
  \item \verb|Slice<Range<usize>>|
  \item \verb|FindSubstring<&[Spanned<Token>]>|
\end{itemize}
The \texttt{Tokens} type can then be used by Nom in a very efficient manner ---
these traits are one of the reasons why Nom is so performant.

The parsers generally return a single AST node as their output. These nodes are
then combined into other nodes, which are then combined into
\texttt{Production}s, and finally into a \texttt{Grammar}. In fact, the main
\texttt{parse} function, the header of which is
\begin{minted}[fontsize=\small,breaklines]{rust}
fn parse(tokens: &[Spanned<Token>]) -> Result<Spanned<Grammar>, Spanned<Error>>
\end{minted}
returns a \texttt{Grammar}, which contains the information about the whole
syntax tree parsed from the provided sequence of tokens.

The code related to syntactic analysis can be found in the \texttt{parser}
module. It is divided among several files:
\begin{description}
  \item[\texttt{mod.rs}] defines all major parsers along with the main
  \texttt{parse} function exported from the module,
  \item[\texttt{ast.rs}] contains the definition of the AST,
  \item[\texttt{tokens.rs}] holds the definition of the \texttt{Tokens}
  type and its trait implementations,
  \item[\texttt{error.rs}] contains the definition of the \texttt{parser::Error}
  type,
  \item[\texttt{utils.rs}] defines the utility functions and \emph{literal}
  parsers,
  \item[\texttt{tests.rs}] contains unit tests related to syntactic analysis,
  which are going to be discussed in Section~\ref{sec:testing}.
\end{description}

\newpage

\subsection{Semantic analyser} \label{sbs:semantic-analyser}

Grammar preprocessing is a phase that tries to detect semantic errors in the
AST. This phase does not \emph{produce} anything from the AST, and serves only
as a \emph{guard} for any semantic inaccuracies, which include undefined
production rules, or direct or indirect left recursion.

Before checking for left recursion, the algorithm checks for any non-terminals
in the AST that do not have a definition. These non-terminals are invalid and
any occurrence of such a non-terminal should be reported as an error. The basic
principle of detecting undefined non-terminals, and so --- production rules ---
is a recursive walk of the AST in search of any undefined non-terminals. A
non-terminal is undefined if there is no \texttt{Production} in the
\texttt{Grammar} with an appropriate identifier in the \texttt{lhs} (left-hand
side). When such a non-terminal is found, the algorithm reports an
\texttt{UndefinedRule} error at the position of the non-terminal.

In the formal language theory, a production rule is left-recursive if the
leftmost symbol of it is itself (in the case of direct left recursion) or can be
made itself by some sequence of substitutions (in the case of indirect left
recursion). A PEG is called \emph{well-formed} if it contains no left-recursive
rules, i.e., rules that allow a non-terminal to expand to an expression in which
the same non-terminal occurs as the leftmost symbol.

Consider the following example:
\begin{minted}[fontsize=\small]
  {lexers/ebnf_lexer.py:EbnfLexer -x}
integer = ? [0-9.]+ ?;
value   = integer | '(', expr, ')';
product = expr, {('*' | '/'), expr};
sum     = expr, {('+' | '-'), expr};
expr    = product | sum | value;
\end{minted}
In this grammar, matching an \texttt{expr} first requires testing if a
\texttt{product} matches while, at the same time, matching a \texttt{product}
requires testing if an \texttt{expr} matches. As the term appears in the
leftmost position, these rules make up a circular definition and cannot be
resolved.

Left recursion regularly poses problems for parsers as it leads them into the
state of infinite recursion. As left-recursive rules can always be rewritten,
grammars are often preprocessed to eliminate them. \thisproject{}, however, does
not attempt to do any rewrites. Instead, it only detects the presence of direct
or indirect left recursion in the provided grammar. The process of detecting
left recursion can be seen in Algorithm~\ref{alg:detecting-left-recursion}.

The code related to preprocessing, or the syntactic analysis, can be found in
the \texttt{preprocessor} module of the crate. The module is divided into
several separate files:
\begin{description}
  \item[\texttt{mod.rs}] contains the algorithms for undefined rule and left
  recursion detection, as well as the main \texttt{preprocess} function,
  \item[\texttt{error.rs}] defines the \texttt{preprocessor::Error}
  type,
  \item[\texttt{tests.rs}] contains unit tests related to preprocessing,
  which are further mentioned in Section~\ref{sec:testing}.
\end{description}

\subsection{Grammar processing} \label{sbs:grammar-processing}

After creating the AST of a grammar, it can be analysed along with an input
string to check if that input string belongs to the language generated by the
grammar. For this purpose, the program recursively checks nodes of the AST to
see if they match the currently scanned part of the input string. First, the
user must provide the initial production rule, from which the process will
begin. The recursive function must either return \emph{success} or
\emph{failure} to indicate

\newpage

\begin{algorithm}[H]
  \DontPrintSemicolon
  \LinesNumbered
  \SetKwFunction{FCheckExpr}{CheckExpr}
  \SetKwFunction{FExit}{exit}
  \SetKwFunction{FPush}{push}
  \SetKwFunction{FPop}{pop}
  \SetKwProg{Fn}{function}{ begin}{}

  \SetKwInput{Input}{input}
  \Input{Dictionary of production rules $R$}
  \Fn{\FCheckExpr{e, t}}{
    \SetKwInput{Input}{inputs}
    \Input{The current expression $e$; stack of identifiers $t$}
    \Switch{e}{
      \Case{Alternative$(S)$}{
        \ForEach{$s \in S$}{
          \FCheckExpr{s, t}
        }
      }
      \Case{Sequence$((s_1, s_2, \dots, s_n))$}{
        \tcc{Skip the last expression}
        \ForEach{$s \in (s_1, s_2, \dots, s_{n-1})$} {
          \tcc{check if $s$ can be an empty expression (e.g. Optional)}
          \If{$s \neq \varepsilon$}{
            \FCheckExpr{s, t}
          }
        }
        \FCheckExpr{$s_n$, t} \tcp*{Check the last expression}
      }
      \Case{Optional$(s)$}{
        \FCheckExpr{s, t}
      }
      \Case{Repeated$(s)$}{
        \FCheckExpr{s, t}
      }
      \Case{Factor$(n, s)$}{
        \If{$n > 0$}{
          \FCheckExpr{s, t}
        }
      }
      \Case{Exception$(s, r)$}{
        \FCheckExpr{s, t}
        \FCheckExpr{r, t}
      }
      \Case{Nonterminal$(i)$}{
        \If{$i = t[0]$}{
          t.\FPush{i}\;
          \FExit{} \tcp*{Report the left recursion error with trace $t$}
        }
        \If{$i \notin t$}{
          t.\FPush{i}\;
          \FCheckExpr{R$[i]$, t}\;
          t.\FPop{}\;
        }
      }
    }
  }
  \ForEach{$(i, e) \in R$}{
    \FCheckExpr{e, R}\;
  }
  \caption{Detecting left recursion in the set of production rules of a
  grammar.}
  \label{alg:detecting-left-recursion}
\end{algorithm}

\noindent whether the parsing has succeeded. The program
processes the input differently depending on the type of the AST node:
\begin{description}
  \item[Alternative] --- The program processes each case of the
  \emph{Alternative} sequentially and returns the first one to return
  \emph{success}. If no case returned \emph{success}, it returns \emph{failure},
  \item[Sequence] --- The program processes each expression of the
  \emph{Sequence} sequentially and returns \emph{success} if and only if every
  processed case returned \emph{success}, otherwise it returns \emph{failure},
  \item[Optional] --- The program processes the expression inside of the
  \emph{Optional} and returns \emph{success} regardless of the result,
  \item[Repeated] --- The program repeatedly processes the expression inside of
  the \emph{Repeated} and returns \emph{success} regardless of any results,
  \item[Factor] --- The program processes the expression inside of the
  \emph{Factor} $N$ times, where $N$ is the number of repetitions of the
  \emph{Factor}. It returns \emph{success} if the processing succeeds all $N$
  times, otherwise it returns \emph{failure},
  \item[Exception] --- The program processes the \emph{subject} of the
  \emph{Exception} and in case of \emph{success}, it stores the processed input
  string to be then processed by the \emph{restriction} of the \emph{Exception}.
  If the restriction succeeds and the resulting processed input string is the
  same as the input string in case of the subject, it returns \emph{failure}. In
  every other case it returns \emph{success},
  \item[Non-terminal] --- The program recursively processes the production rule
  specified by the identifier inside of the \emph{Non-terminal},
  \item[Terminal] --- The program checks if the processed input string starts with
  the input specified by the \emph{Terminal} and returns \emph{success} if it
  does,
  \item[Special] --- Currently, the program always returns \emph{failure} for any
  special sequence,
  \item[Empty] --- The program always returns \emph{success} without processing
  the input string.
\end{description}

\begin{wraplisting}{r}{0.5\textwidth}
  \begin{minted}[fontsize=\small,frame=lines,breaklines]{rust}
pub enum Node {
  Nonterminal(String, Vec<Node>),
  Terminal(String),
}
  \end{minted}
  \caption{The definition of the \texttt{Node} type.}
  \label{lst:checker-node}
\end{wraplisting}

The recursive function produces a parse tree along the way, which is represented
by the \texttt{Node} type defined in Listing~\ref{lst:checker-node}. Each
\texttt{Node} represents a tree node, where \texttt{Nonterminal} variants are
\emph{branches} (containing multiple children), and \texttt{Terminal} variants
are \emph{leaves} of the tree. Additionally, each \texttt{Nonterminal} has an
associated identifier, and each \texttt{Terminal} carries information about the
input parsed by that terminal. The construction of \texttt{Node}s follows a
similar behavior to the checking:
\begin{itemize}
  \item \texttt{Alternative}s return the node of the first successful parse,
  \item In the case of \texttt{Sequence}s , all nodes parsed from that sequence
  are appended to the \texttt{Vec} of children nodes,
  \item \texttt{Optional}s may not return a node at all,
  \item \texttt{Repeated}s append the repeated node to the \texttt{Vec} of
  children as many times as the parser is successful,
  \item \texttt{Factor}s append the repeated node to the \texttt{Vec} of
  children as many times as the \texttt{Factor} defines,
  \item In the case of \texttt{Exception}s, the node of the \emph{subject} is
  returned as long as the exception is valid,
  \item \texttt{Nonterminal}s return the \texttt{Nonterminal} node with an
  appropriate identifier and a recursive call,
  \item \texttt{Terminal}s return the \texttt{Terminal} node along with the
  parsed input.
\end{itemize}

\newpage

\section{Command-line application} \label{sec:command-line-application}

The command-line application serves as a most basic tool for interfacing with
the grammar parser and checker. It is intended for people who want to, for
example, automate the parsing and checking process, do it locally, or just plan
to use it without the graphical user interface.

\begin{wraplisting}{l}{0.5\textwidth}
  \begin{minted}[fontsize=\scriptsize,frame=lines,breaklines]{text}
$ parser-parser --help
parser-parser 0.1.0

USAGE:
    parser-parser [OPTIONS] <GRAMMAR FILE>

FLAGS:
    -h, --help       Prints help information
    -V, --version    Prints version information

OPTIONS:
    -t, --test <TEST STRING FILE>    Test string file

ARGS:
    <GRAMMAR FILE>    Grammar file path
  \end{minted}
  \caption{The output of the program ran with the \texttt{-{}-help} flag.}
  \label{lst:cli-help}
\end{wraplisting}

The tool reads the arguments to the program from the command line. The user must
provide a path to the file containing a grammar in the EBNF format as an
obligatory positional argument. The program provides an additional optional
argument for passing a file containing the test input string denoted with the
\texttt{-t}, or the \texttt{--test} flag. In case the test string is provided,
the program parses the grammar, and then checks the provided input from a file.
However, if the test string is not provided, the program, after parsing the
grammar, enters the REPL mode. The program provides the \verb|--help| and
\verb|--version| flags for displaying the help menu
(Figure~\ref{lst:cli-help}) and the current version of the program respectively.

For parsing command-line arguments, in its core \thisproject{} uses the
\emph{Clap} crate for Rust --- a Command Line Argument Parser. It is a
simple-to-use, efficient, and full-featured library for parsing command-line
arguments and subcommands when writing console/terminal applications. Clap is
used to parse and validate the string of command-line arguments provided by a
user at runtime. The developer provides the list of valid possibilities, and
Clap handles the rest. This means the developer can focus on the application's
functionality, and less on the parsing and validating of arguments.

Clap provides many things with no configuration, including the traditional
version and help switches (or flags) along with associated messages. If the user
is using subcommands, Clap will also auto-generate a help subcommand and
separate associated help messages. Once Clap parses the user provided string of
arguments, it returns the matches along with any applicable values. If the user
made an error or typo, Clap informs them with a friendly message and exits
the program.

Besides Clap, \thisproject{} also uses the \emph{StructOpt} crate, which parses
command line arguments by defining a struct. It combines Clap with a custom
derive. By defining a regular struct and marking it with a specific derive,
StructOpt can automatically generate a command-line argument parser based on the
values in the struct and their types. StructOpt is easy to use and is a
convenient method of parsing the arguments into a single structure, which then
can be used in the actual program. The StructOpt struct for \thisproject{} is
defined in Listing~\ref{lst:cli-structopt}, where it takes two file paths: one
obligatory and positional, and one optional and marked with an appropriate flag.

\begin{listing}[H]
  \begin{minted}[fontsize=\small,frame=lines,breaklines,linenos]{rust}
#[derive(Debug, StructOpt)]
pub struct Config {
  /// Grammar file path
  #[structopt(name = "GRAMMAR FILE", parse(from_os_str))]
  pub grammar_path: PathBuf,
  /// Test string file path
  #[structopt(short = "t", long = "test", name = "TEST STRING FILE", parse(from_os_str))]
  pub test_string_path: Option<PathBuf>,
}
  \end{minted}
  \caption{The StructOpt struct defining the command-line arguments for the
  program.}
  \label{lst:cli-structopt}
\end{listing}

\begin{wraplisting}{r}{0.5\textwidth}
  \begin{minted}[fontsize=\scriptsize,frame=lines,breaklines]{text}
$ parser-parser ../grammar.ebnf
parser-parser 0.1.0

Successfully parsed the provided grammar

Entering REPL mode...
> 4+12*6
true
> x*3f
false
>
  \end{minted}
  \caption{Sample of the output of the program ran in REPL mode.}
  \label{lst:cli-repl}
\end{wraplisting}

In computing, a \emph{REPL}, or the \emph{read–eval–print loop}, is a simple
interactive computer programming environment that takes single user inputs,
executes them, and returns the result to the user. A program written in a REPL
environment is executed piecewise. The term is usually used to refer to
programming interfaces similar to the classic Lisp machine interactive
environment. Common examples include command line shells and similar
environments for programming languages, and the technique is very characteristic
of scripting languages.

The name \emph{read–eval–print loop} comes from the steps the program executes:
\begin{itemize}
  \item The \emph{read} step accepts an input from the user,
  \item The \emph{eval} step takes the provided input and ``evaluates'' it,
  or in this case it checks the input against the provided grammar,
  \item The \emph{print} step takes the result yielded by \texttt{eval} and
  prints it out to the user.
\end{itemize}
The environment then returns to the \emph{read} state, creating a loop, which
terminates when the program is closed. An example terminal output of the REPL
mode can be seen in Figure~\ref{lst:cli-repl}.

\section{Web-based application} \label{sec:web-based-application}

\subsection{Linking the business logic}

To use the business logic (described in Section~\ref{sec:business-logic}) in a
web environment, the Rust code must be compiled to a WebAssembly module. This is
done through Wasm-pack --- a~tool for building and working with WebAssembly
generated from Rust code. The generated WebAssembly module can then be used like
any other JavaScript module. Wasm-pack tool interoperates with wasm-bindgen,
which has the ability to generate bindings, that allows JavaScript and Rust
interact with each other through numbers, strings, or even JavaScript objects
and arrays.

Wasm-pack automatically ensures that Rust 1.30 or newer and the
\texttt{wasm32-unknown-unknown} target is installed via \emph{rustup}, compiles
the Rust sources into a WebAssembly \texttt{.wasm} binary via Cargo, and finally
uses wasm-bindgen to generate the JavaScript API for using the Rust-generated
WebAssembly. To do all of that, the developer needs to run the \texttt{wasm-pack
build} command inside the project directory. When the build completes, its
artifacts can be found in the \texttt{pkg} directory. The directory contains
several files:
\begin{itemize}
  \item The \texttt{.wasm} file is the WebAssembly binary that is generated by
  the Rust compiler from the Rust sources. It contains the compiled-to-wasm
  versions of all of the Rust functions and data.
  \item The \texttt{.js} file is generated by wasm-bindgen and contains
  JavaScript ``glue'' for importing DOM and JavaScript functions into Rust and
  exposing an API to the WebAssembly functions to JavaScript. The code will
  verify parameters passed across the boundary of WebAssembly and JavaScript and
  invoke appropriate functions.
  \item The \texttt{.d.ts} file contains TypeScript type declarations for the
  JavaScript ``glue''. If the developer is using TypeScript, they can have their
  calls into WebAssembly functions type checked, and the IDE can provide
  autocompletions and suggestions. \thisproject{} does not use TypeScript, so
  the file can be ignored.
  \item The \texttt{package.json} file contains metadata about the generated
  JavaScript and WebAssembly package. This is used by npm and JavaScript
  bundlers to determine dependencies across packages, package names, and their
  versions. It helps us integrate with JavaScript tooling and allows the
  developer to publish the package to npm.
\end{itemize}

\begin{wraplisting}{l}{0.5\textwidth}
  \begin{minted}[fontsize=\scriptsize,frame=lines,breaklines,linenos]{rust}
#[wasm_bindgen]
pub struct EbnfParserParser {
  grammar: base::Grammar,
}

#[wasm_bindgen]
impl EbnfParserParser {
  #[wasm_bindgen(constructor)]
  pub fn new(input: &str) -> Result<EbnfParserParser, JsValue> {
    // ...
  }

  #[wasm_bindgen(getter = productionRules)]
  pub fn get_production_rules(&self) -> Array {
    // ...
  }

  pub fn check(&self, input: &str, initial_rule: &str) -> Option<Object> {
    // ...
  }
}
  \end{minted}
  \caption{The definition of the EBNF parser struct that encapsulates the
  grammar.}
  \label{lst:wasm-glue-parser}
\end{wraplisting}

Since the \texttt{Grammar} struct does not to be exported directly to
JavaScript, Wasm-bindgen exposes the \texttt{EbnfParserParser} struct
encapsulating the \texttt{Grammar} (Listing~\ref{lst:wasm-glue-parser}), which
JavaScript treats as a regular class. Wasm-bindgen allows to mark the
\texttt{new} function as a class constructor, so the construction of the
\texttt{EbnfParserParser} object can be done via the \texttt{new} keyword.
Besides the constructor, Wasm-bindgen marks the \texttt{get\_production\_rules}
method as a regular getter, which will be exposed as a JavaScript property,
instead of a method. Finally, after generation, the class will also contain the
\texttt{check} method, which will check the provided \texttt{input} against the
encapsulated grammar, starting with the \texttt{initial\_rule}.

The return types of these functions get exposed to JavaScript in the following
way:
\begin{itemize}
  \item \texttt{Object} and \texttt{Array} types are translated directly into
  JavaScript objects and arrays respectively. These types come from the
  \texttt{js\_sys} crate. The \texttt{Array} type may be produced by coercing a
  \texttt{Vec} of \texttt{JsValue}s. The \texttt{Object} type, on the other
  hand, can be produced with \texttt{Object::new()} and have its properties set
  using the \texttt{Reflect} module and the \texttt{Reflect::set} function.
\end{itemize}

\begin{itemize}
  \item The \texttt{Option<T>} type translates into a \texttt{null}able value,
  so the \texttt{Some(T)} variant becomes \texttt{T}, and the \texttt{None}
  variant becomes \texttt{null}.
  \item A function returning \texttt{Result<T, E>}, after exposing the function
  to JavaScript, will return \texttt{T} in case of the \texttt{Ok(T)} variant of
  the result, and throw the value of \texttt{E} in case of the \texttt{Err(E)}
  variant.
\end{itemize}

\subsection{Text editor}

\thisproject{} utilizes CodeMirror as its text and code editor. CodeMirror is a
code editor component that can be embedded in web pages. The core library
provides only the editor component, no accompanying buttons, auto-completion, or
other IDE functionality. It does provide a rich API on top of which such
functionality can be straightforwardly implemented. The distribution includes
addons, which are reusable implementations of extra features. CodeMirror works
with language-specific modes, which are JavaScript programs that help color (and
optionally indent) text written in a given language. The distribution comes with
a number of modes, including the EBNF mode, which is used in \thisproject{}.

To integrate CodeMirror with Svelte, the author implemented a wrapper Svelte
component on top of the native CodeMirror component. The implemented component
utilizes the state-driven approach of Svelte, and exposes callbacks for
associating functionality with events of the editor. CodeMirror works on top of
a \texttt{<textarea>} HTML element of the web document:
\begin{minted}[fontsize=\small]{html}
<textarea bind:this="{textAreaRef}" readonly></textarea>
\end{minted}
The \texttt{<textarea>} is bound to a \texttt{textAreaRef} variable with the use
of the \texttt{bind:this} construct. Through JavaScript, the editor is
constructed in the \texttt{onMount} event of the component by invoking the
\texttt{fromTextArea} function with an appropriate binding. This will, among
other things, ensure that the \texttt{textarea}'s value is updated with the
editor's contents when the form is submitted.
\begin{minted}[fontsize=\small]{javascript}
editor = CodeMirror.fromTextArea(textAreaRef, opts);
\end{minted}
where \texttt{opts} is a dictionary of configuration options provided by Svelte.
Any option not supplied like this will be taken from
\texttt{CodeMirror.defaults}, an object containing the default options. Options
are not checked in any way, so setting undefined option values is bound to lead
to odd errors. Some more notable options include:
\begin{description}
  \item[\texttt{mode}] --- The mode to use. When not given, this will default to
  the first mode that was loaded. It may be a string, which either simply names
  the mode or is a MIME type associated with the mode. Alternatively, it may be
  an object containing configuration options for the mode, with a name property
  that names the mode (for example \texttt{{name: "javascript", json: true}}).
  The value \texttt{null} indicates no highlighting should be applied.
  \item[\texttt{theme}] --- The theme to style the editor with. The developer
  must make sure the CSS file defining the corresponding \texttt{.cm-s-[name]}
  styles is loaded from the theme directory in the distribution. The default is
  ``\texttt{default}'', for which colors are included in
  \texttt{codemirror.css}.
  \item[\texttt{tabSize}] --- The width of a tab character. Defaults to 4.
  \item[\texttt{extraKeys}] --- Can be used to specify extra key bindings for
  the editor, alongside the ones defined by \texttt{keyMap}. Should be either
  null, or a dictionary of key bindings.
  \item[\texttt{lint}] --- Available after loading the Lint addon, which defines
  an interface component for showing linting warnings, with pluggable warning
  sources. The \texttt{lint} option can be set to a warning source, which is a
  validator function that returns a list of errors from the provided string.
\end{description}

\subsection{Parse tree visualizations}

\begin{wrapfigure}{l}{0.3\textwidth}
  \centering
  \includegraphics[width=5cm]{parse_tree.png}
  \caption{A screenshot of the visualized parse tree.}
  \label{fig:front-end-parse-tree}
\end{wrapfigure}

Section~\ref{sbs:grammar-processing} described the \texttt{Node} structure,
which represents the parse tree returned from the checker. This structure is
utilized in the visualization of the parse tree on the front-end. Each
\texttt{Node} converted to JavaScript objects is composed of a \texttt{name}
parameter, and, in the case of branch nodes, a \texttt{children} parameter with
an array of all children nodes. This structure is compatible with
\emph{Svelte-tree} package --- a tree-like view component for Svelte. The
component is used on the front-end to display a collapsable tree based on the
structure described above. An example of the visualized parse tree is presented
in Figure~\ref{fig:front-end-parse-tree}.

Svelte-tree allows the user to associate a node with an identifier via the
\texttt{let:node} construct, and this way access the names of nodes being
rendered at appropriate DOM elements.

The Svelte-tree is abstracted away into its own Svelte component responsible for
visualizing the parse tree, which can be seen in
Listing~\ref{lst:svelte-parse-tree}. In the \texttt{<script>} section it imports
the \texttt{Tree} component from the Svelte-tree package, and exposes the
\texttt{tree} parameter as its input. Each node in the tree is marked with a
\texttt{node} class, which, in the \texttt{<style>} section, is getting an
appropriate styling with CSS.

\begin{listing}[H]
  \begin{minted}[fontsize=\footnotesize,frame=lines,breaklines,linenos]{html}
<script>
  import Tree from "svelte-tree";
  export let tree;
</script>

<main>
  <Tree tree={[tree]} let:node>
    <div class="node">{node.name}</div>
  </Tree>
</main>

<style>
  .node {
    float: left;
    font-family: "JetBrains Mono", Consolas, monospace;
    color: #928374;
    padding-left: 24px;
  }
</style>
  \end{minted}
  \caption{Contents of \texttt{ParseTree.svelte}, the parse tree visualization
  component.}
  \label{lst:svelte-parse-tree}
\end{listing}

\chapter{Project quality study} \label{ch:project-quality-study}

\section{Business logic testing} \label{sec:testing}

\subsection{Unit testing}

Unit testing is a type of software testing where individual units or components
of the software are being tested. The purpose of unit testing is to validate
that each unit of the software code performs in an expected way. Unit tests
assist in fixing bugs early in the development cycle, help to understand the
code base and make changes to the code quicker, and can also serve as project
documentation. Unit testing allows the programmer to refactor code, and ensure
the refactored module still works correctly. The process is to write test cases
for all functionalities so that whenever a change causes an error, it can be
quickly identified and be dealt with. Unit testing, however, can't be expected
to catch every error in a program. It is not possible to evaluate all execution
paths even in the most trivial programs. Unit testing by its definition focuses
on a unit of code --- it can't catch integration errors or broad system-level
errors.

A programmer usually writes a section of code in the application just to test a
function. Isolating the tested code helps in disconnecting unnecessary
dependencies from the code being tested and other units in the project ---
testing should be focused on only one piece of code at a time, and individual
unit test cases should be independent. In case of any changes in requirements,
unit test cases should not be affected. If units are made less interdependent to
make unit testing possible, the unintended impact of changes to any code is less
noticeable.

Unit tests in \thisproject{} focus on testing the individual components of the
EBNF parser, that is: the lexer, the parser, and the preprocessor. If the
developer is sure that all of these components work correctly in isolation, the
same thing can be said about the whole system, because there are no hidden
dependencies between these components.

Tests related to the lexer test the functionality of the lexer, that is, test
the ability of the lexer to turn the textual representation of EBNF into
individual tokens. An typical unit test related to the lexer can be seen in
Listing~\ref{lst:lexer-unit-test}, where it makes an assertion that the result
of the \texttt{lex} function on input ``\verb*| abc \n = 'def' |'' is a success
(the \texttt{Ok} case), and it's a vector of three certain tokens.

\begin{listing}[H]
  \begin{minted}[fontsize=\scriptsize,frame=lines,breaklines,linenos]{rust}
#[test]
fn test_multiline() {
  assert_eq!(
    lex(" abc \n = 'def' "),
    Ok(vec![
      Token::Nonterminal("abc".to_owned()).spanning(Span::from(((1, 0), (4, 0)))),
      Token::Definition.spanning(Span::from(((1, 1), (2, 1)))),
      Token::Terminal("def".to_owned()).spanning(Span::from(((3, 1), (8, 1))))
    ])
  );
}
  \end{minted}
  \caption{A unit test related to the lexer testing the proper tokenization of
  the input string.}
  \label{lst:lexer-unit-test}
\end{listing}

All test cases related to the lexer are listed in
Table~\ref{tab:lexer-test-cases} in a simplified format, where the result can
either be a success case or a failure case, but does not provide information
about the lexed tokens for simplicity. The actual test cases for the lexer
are located in the file \texttt{ebnf/src/lexer/tests.rs}.

\begin{table}[H]
  \centering
  \caption{Tests cases related to the lexer along with the expected results.}
  \label{tab:lexer-test-cases}
  \begin{tabular}{@{}ll@{}}
    \toprule
    Test case & Result \\ \midrule
    \verb*@,@                    & success \\
    \verb*@| /!@                 & success \\
    \verb*@abc = b;@             & success \\
    \verb*@ (/ [ /) ]@           & success \\
    \verb*@ (/) @                & failure \\
    \verb*@ /@                   & success \\
    \verb*@(::) { } @            & success \\
    \verb*@ (:) @                & failure \\
    \verb*@ "ab c " @            & success \\
    \verb*@  '"aba' @            & success \\
    \verb*@ ' a "@               & failure \\
    \verb*@"bbb'   @             & failure \\ \bottomrule
  \end{tabular}
  \hspace{0.5cm}
  \begin{tabular}{@{}ll@{}}
    \toprule
    Test case & Result \\ \midrule
    \verb*@""@                   & failure \\
    \verb*@''@                   & failure \\
    \verb*@ ? test ?@            & success \\
    \verb*@?a\nbc?  @            & success \\
    \verb*@ ?bbb  @              & failure \\
    \verb*@??@                   & success \\
    \verb*@ 123 @                & success \\
    \verb*@ 1 2  3  @            & success \\
    \verb*@ 01234 5@"            & success \\
    \verb*@ 0 @                  & success \\
    \verb*@ abc @                & success \\
    \verb*@a  bc @               & success \\ \bottomrule
  \end{tabular}
  \hspace{0.5cm}
  \begin{tabular}{@{}ll@{}}
    \toprule
    Test case & Result \\ \midrule
    \verb*@abc12 3 @             & success \\
    \verb*@ x @                  & success \\
    \verb*@ + @                  & failure \\
    \verb*@  , \n,@              & success \\
    \verb*@ (* test *) @         & success \\
    \verb*@ (* test * @          & failure \\
    \verb*@ (* (@                & failure \\
    \verb*@, (*, *) , @          & success \\
    \verb*@ ,(*, (* ,*) ,*) , ,@ & success \\
    \verb*@ (* (* *) @           & failure \\
    \verb*@ (*) @                & failure \\
    \verb*@ abc \n = 'def' @     & success \\ \bottomrule
  \end{tabular}
\end{table}

Tests related to the parser test the functionality of the parser, that is, test
the ability to transform a list of tokens into an AST. To make the unit tests of
the parser independent from the implementation of the lexer, the input to the
\texttt{parse} function is a vector of tokens, rather than a result of the
\texttt{lex} function. An example test case related to the parser is seen in
Listing~\ref{lst:parser-unit-test}, where the \texttt{ok\_case} macro represents
a test case that, after testing the \texttt{factor} parser, should result in a
success, parse 3 tokens, and return a \texttt{Factor} AST node. In the same
file, that is \texttt{ebnf/src/parser/tests.rs} one can also find the
\texttt{error\_case} macro, which represents a test case that should fail with a
specific error.

\begin{listing}[H]
  \begin{minted}[fontsize=\scriptsize,frame=lines,breaklines,linenos]{rust}
ok_case!(
  factor,
  &vec![
    Token::Integer(2).spanning(Span::from(((0, 0), (1, 0)))),
    Token::Repetition.spanning(Span::from(((2, 0), (3, 0)))),
    Token::Terminal("terminal".to_owned()).spanning(Span::from(((4, 0), (14, 0))))
  ],
  3,
  Expression::Factor {
    count: 2.spanning(Span::from(((0, 0), (1, 0)))),
    primary: Box::new(
      Expression::Terminal("terminal".to_owned()).spanning(Span::from(((4, 0), (14, 0))))
    )
  }
  .spanning(Span::from(((0, 0), (14, 0))))
);
  \end{minted}
  \caption{A unit test related to the parser, where the ability to turn a string
  of tokens into an AST is tested.}
  \label{lst:parser-unit-test}
\end{listing}

Tests related to the preprocessor test the functionality of the preprocessor,
that is, test the ability to detect undefined rules and left recursion. Test
case in Listing~\ref{lst:preprocessor-unit-test} tests the grammar
\begin{minted}[fontsize=\small]{lexers/ebnf_lexer.py:EbnfLexer -x}
a = b;
b = a;
\end{minted}
and the detection of indirect left recursion $b \rightarrow a \rightarrow b$.

\begin{listing}[H]
  \begin{minted}[fontsize=\scriptsize,frame=lines,breaklines,linenos]{rust}
#[test]
fn test_indirect_left_recursion() {
  assert_eq!(
    preprocess(
      Grammar {
        productions: vec![
          Production {
            lhs: "a".to_owned().spanning(Span::from(((0, 0), (1, 0)))),
            rhs: Expression::Nonterminal("b".to_owned())
              .spanning(Span::from(((4, 0), (5, 0))))
          }
          .spanning(Span::from(((0, 0), (6, 0)))),
          Production {
            lhs: "b".to_owned().spanning(Span::from(((0, 1), (1, 1)))),
            rhs: Expression::Nonterminal("a".to_owned())
              .spanning(Span::from(((4, 1), (5, 1))))
          }
          .spanning(Span::from(((0, 1), (6, 1))))
        ]
      }
      .spanning(Span::from(((0, 0), (6, 1))))
    ),
    Err(
      Error::LeftRecursion(vec!["b".to_owned(), "a".to_owned(), "b".to_owned()])
        .spanning(Span::from(((4, 0), (5, 0))))
    )
  );
}
  \end{minted}
  \caption{A preprocessor unit test testing the left recursion detection in an
  AST.}
  \label{lst:preprocessor-unit-test}
\end{listing}

All tests related to business logic are written in Rust, as the business logic
is also written in Rust. All unit tests in the project can be ran with the use
of Cargo with the \texttt{cargo test} command in the terminal.

\subsection{Property-based testing}

Test engineers write mostly example-based tests where only one input scenario is
tested. Property-based testing is a useful addition to a test suite because it
runs one statement hundreds of times with different inputs. Property-based
testing frameworks use almost every conceivable input that could break the
code, such as empty lists, negative numbers, and really long lists or strings.
Property based testing has become quite famous in the functional world. Mainly
introduced by QuickCheck framework in Haskell, it suggests another way to test
software.

Property-based tests are designed to test the aspects of a property that should
always be true. They allow for a range of inputs to be programmed and tested
within a single test, rather than having to write a different test for every
value that the programmer wants to test. These tests are useful when a range of
inputs needs to be tested on a given aspect of a software property, or if the
developer is concerned about finding missed edge cases.

\begin{listing}[H]
  \begin{minted}[fontsize=\small,frame=lines,breaklines,linenos]{rust}
use quickcheck_macros::quickcheck;

#[quickcheck]
fn test_arbitrary_input(input: String) {
  let _ = lex(&input);
}
  \end{minted}
  \caption{A QuickCheck test testing arbitrary inputs on the lexer.}
  \label{lst:lexer-quickcheck-test}
\end{listing}

\thisproject{} uses the \texttt{quickcheck} crate for Rust, which is a
property-based testing framework inspired by Haskell's QuickCheck and has been
described in Section~\ref{sbs:used-technologies}. \texttt{quickcheck} tests the
``correctness'' of \texttt{lex} and \texttt{parse}, i.e. it checks if these
functions do not emit a \emph{panic} for an arbitrary input, whether it's a
random string of characters for the case of \texttt{lex}, or a random string of
tokens for \texttt{parse}. Example \texttt{quickcheck} tests can be seen in
Listings~\ref{lst:lexer-quickcheck-test} and \ref{lst:parser-quickcheck-test},
where the \texttt{quickcheck} macro is used to streamline the usage of the
library. \texttt{quickcheck} catches any panics that may occur in \texttt{lex}
and \texttt{parse} functions and reports them to the user along with the input
provided to these functions. In the case of \texttt{parse}, to generate an
arbitrary string of \texttt{Token}s, it was necessary to implement the
\texttt{Arbitrary} trait provided by \texttt{quickcheck} for \texttt{Token}.
\texttt{Spanned<T>}, \texttt{Span}, and \texttt{Location} types (most of this
has been omitted in Listing~\ref{lst:parser-quickcheck-test}).

\begin{listing}[H]
  \begin{minted}[fontsize=\small,frame=lines,breaklines,linenos]{rust}
// ...
impl<T: Arbitrary> Arbitrary for Spanned<T> {
  fn arbitrary<G: Gen>(g: &mut G) -> Spanned<T> {
    Spanned {
      node: T::arbitrary(g),
      span: Span::arbitrary(g),
    }
  }
}

#[quickcheck]
fn test_arbitrary_input(tokens: Vec<Spanned<Token>>) {
    use super::parse;
    let _ = parse(tokens.as_slice());
}
  \end{minted}
  \caption{A QuickCheck test related to the parser, testing strings of arbitrary
  tokens.}
  \label{lst:parser-quickcheck-test}
\end{listing}

\section{Integration testing}

Integration testing is defined as a type of testing where software modules are
integrated logically and tested as a group. A typical software project consists
of multiple software modules, coded by different programmers. The purpose of
this level of testing is to expose defects in the interaction between these
software modules when they are integrated.

\thisproject{} utilizes integration testing thanks to \emph{wasm-pack}. This
tool allows the developer you build rust-generated WebAssembly packages, as well
as test them in a headless web browser via the wasm-pack test command. A tool
used to define the integration tests is the wasm-bindgen-test crate --- an
experimental test harness for Rust programs compiled to wasm using wasm-bindgen
and the wasm32-unknown-unknown target. The \texttt{wasm-pack test} command wraps
the wasm-bindgen-test-runner CLI allowing the developer to run wasm tests in
different browsers without needing to install the different webdrivers. An
example integration test ran with wasm-bindgen-test can be seen in
Listing~\ref{lst:ebnf-integration-tests}.

\begin{listing}[H]
  \begin{minted}[fontsize=\small,frame=lines,breaklines,linenos]{rust}
use ebnf::parse;
use wasm_bindgen_test::*;

#[wasm_bindgen_test]
fn test_ebnf() {
  assert!(parse(" abc = 'def'; ").is_ok());
  assert!(parse(" (* test *) ").is_err());
  assert!(parse(" (* test *").is_err());
  assert!(parse("a = b;").is_err());
  assert!(parse("a = 'a';;").is_err());
  assert!(parse("a = ;").is_ok());
  assert!(parse("a = a;").is_err());
}
  \end{minted}
  \caption{An integration test ran in a headless browser, which tests various
  grammars in a textual form.}
  \label{lst:ebnf-integration-tests}
\end{listing}

\section{Benchmarking}

Benchmarks check the performance of the code. \thisproject{}'s benchmark tool of
choice is Criterion.rs --- a statistics-driven micro-benchmarking tool. It is a
Rust port of Haskell's Criterion library. Criterion.rs benchmarks collect and
store statistical information from run to run and can automatically detect
performance regressions as well as measuring optimizations.
Listing~\ref{lst:benchmark} shows a Criterion.rs benchmark that parses a sample
input grammar multiple times.

\begin{listing}[H]
  \begin{minted}[fontsize=\footnotesize,frame=lines,breaklines,linenos]{rust}
use criterion::{black_box, criterion_group, criterion_main, Criterion};
use ebnf::parse;

const GRAMMAR: &str = "
expression = term, { ('+' | '-'), term };
term       = factor, { ('*' | '/'), factor };
factor     = constant | variable | '(', expression, ')';
variable   = 'x' | 'y' | 'z';
constant   = digit, { digit };
digit      = '0' | '1' | '2' | '3' | '4' | '5' | '6' | '7' | '8' | '9';
";

fn criterion_benchmark(c: &mut Criterion) {
  c.bench_function("parse", |b| b.iter(|| parse(black_box(GRAMMAR))));
}

criterion_group!(benches, criterion_benchmark);
criterion_main!(benches);
  \end{minted}
  \caption{A benchmark testing the speed of parsing a sample grammar.}
  \label{lst:benchmark}
\end{listing}

Criterion.rs can generate a number of useful charts and graphs which the
developer can check to get a better understanding of the behavior of the
benchmark. These charts will be generated with \emph{gnuplot} by default, and
the examples below were generated using the gnuplot backend.

\begin{figure}[H]
  \centering
  \resizebox{\textwidth}{!}{\footnotesize\input{images/pdf_bench.pdf_tex}}
  \caption{Probability Distribution Function chart generated by Criterion.rs.}
  \label{fig:pdf-benchmark}
\end{figure}

The \emph{PDF chart} in Figure~\ref{fig:pdf-benchmark} shows the
\emph{probability distribution function} for the samples. It also shows the
ranges used to classify samples as outliers. In this example we can see that the
performance trend does not change noticeably across the whole benchmark and
stays around \SI{55}{\micro\second}

\begin{figure}[H]
  \centering
  \resizebox{\textwidth}{!}{\footnotesize\input{images/regression_bench.pdf_tex}}
  \caption{Regression plot generated by Criterion.rs.}
  \label{fig:regression-benchmark}
\end{figure}

The \emph{regression plot}, shown in Figure~\ref{fig:regression-benchmark},
shows each data point plotted on an X-Y plane showing the number of iterations
versus the time taken. It also shows the line representing Criterion.rs' best
guess at the time per iteration. A good benchmark will show the data points all
closely following the line. If the data points are scattered widely, this
indicates that there is a lot of noise in the data and that the benchmark may
not be reliable. If the data points follow a consistent trend but don't match
the line (eg. if they follow a curved pattern or show several discrete line
segments) this indicates that the benchmark is doing different amounts of work
depending on the number of iterations, which prevents Criterion.rs from
generating accurate statistics and means that the benchmark may need to be
reworked.

The graphics that Criterion.rs generates are perfect for contributors of the
project as there is no dearth of info. Criterion generates graphics that break
down mean, median, standard deviation, MAD, etc., which are invaluable when
trying to pinpoint areas of improvement.

\section{Auditing}

A software audit is an internal or external review of a software program to
check its quality, progress or adherence to plans, standards and regulations.
\thisproject{} utilizes Google Lighthouse --- an open-source, automated tool for
improving the quality of web pages. Developers can run it against any web page,
public or requiring authentication. It has audits for performance,
accessibility, progressive web apps, SEO and more. When the tool finishes
analyzing a web page, it returns a report with the calculated scores for each
metric, a list of problems with the page, and general, or sometimes specific,
recommendations regarding solving those problems.
Figure~\ref{fig:lighthouse-summary} shows a short summary of Lighthouse ran
against the deployed web application. For most metrics, Lighthouse calculates a
score by comparing the page with the FCP data present in HTTP Archive. The tool
uses a color-coding system to display how well a page performs according to a
particular metric.

\begin{figure}[h]
  \centering
  \includegraphics[width=\textwidth]{lighthouse_summary.png}
  \caption{Google Lighthouse's scores of the deployed web application.}
  \label{fig:lighthouse-summary}
\end{figure}

\thisproject{} scores 100 points in the performance category. This category has
metrics that together reflect how fast the page is perceptually:
\begin{description}
  \item[First Contentful Paint] --- Shows how long it takes for a browser to
  render
  DOM content,
  \item[Speed Index] --- Shows how quickly the contents of a page load visually.
  To do this, Lighthouse records a video of your page loading and then computes
  a visual progression between frames,
  \item[Largest Contentful Paint] --- Reports the render time of the largest
  image or text block within the viewport (i.e., the visible part of the page),
  \item[Time To Interactive] --- Measures how long it takes a page to become
  \emph{fully} interactive, i.e. useful content (measured by FCP) is displayed,
  JavaScript event handlers are bound to visible elements' events, and the page
  responds to user interaction within 50 milliseconds,
  \item[Total Blocking Time] --- The sum of all time records between FCP and TTI
  when a page is blocked from user interaction for more than 50 milliseconds,
  \item[Cumulative Layout Shift] --- Metric for showing how aggressively
  elements shift each other during the loading.
\end{description}
All values for performance metrics generated by Lighthouse can be seen in
Table~\ref{tab:lighthouse-performance-metrics}. Besides the metrics, the website
passed additional 29 audits, such as \emph{Minify CSS}, \emph{Minify
JavaScript}, \emph{Enable text compression}, \emph{Initial server response time
was short}, \emph{Remove duplicate modules in JavaScript bundles}, \emph{Avoids
an excessive DOM size} and more.
\begin{table}[H]
  \centering
  \caption{Performance metrics generated by Lighthouse.}
  \label{tab:lighthouse-performance-metrics}
  \begin{tabular}{@{}lr@{}}
    \toprule
    Metric                   & Value                 \\ \midrule
    First Contentful Paint   & \SI{0.2}{\second}     \\
    Speed Index              & \SI{0.4}{\second}     \\
    Largest Contentful Paint & \SI{0.4}{\second}     \\
    Time to Interactive      & \SI{0.4}{\second}     \\
    Total Blocking Time      & \SI{0}{\milli\second} \\
    Cumulative Layout Shift  & 0                     \\ \bottomrule
  \end{tabular}
\end{table}

In terms of best practices, the web application scores 100 points, thanks to
HTTPS usage provided by GitHub Pages, safe links to cross-origin destinations,
avoiding requesting geolocation and notification permissions on page load,
and avoiding front-end JavaScript libraries with known security vulnerabilities.

Besides performance audits, Lighthouse also provides audits regarding
accessibility, SEO (Search Engine Optimization), and PWA (Progressive Web App).
These audits were not the main focus of the project, however they provide useful
information about possible future improvements. Lighthouse informs that
background and foreground colors do not have a sufficient contrast ratio and
that low-contrast text is difficult or impossible for many users to read, which
is a valid concern that should be taken into consideration. The accessibility
audit also suggests providing labels for associated form elements, which most
likely relates to \texttt{textarea} grammar and input string fields. In terms of
Progressive Web App audits, Lighthouse suggests using a service worker, which
enables the web app to be reliable in unpredictable network conditions and use
many Progressive Web App features, such as \emph{offline}, \emph{add to
homescreen}, and \emph{push notifications}.

\newpage

\section{Complexity analysis}

Software complexity is a way to describe a specific set of characteristics of
the code. These characteristics all focus on how the code interacts with other
pieces of code.

A popular metric for measuring code complexity is \emph{cyclomatic complexity}.
Cyclomatic complexity uses a mathematical model to assess methods, producing
accurate measurements of the effort required to test them, but inaccurate
measurements of the effort required to understand them.

\emph{Cognitive complexity}, on the other hand, breaks from the practice of
using mathematical models to assess software maintainability. It starts from the
precedents set by cyclomatic complexity, but uses human judgement to assess how
structures should be counted, and to decide what should be added to the model as
a whole. As a result, it yields method complexity scores which strike
programmers as fairer relative assessments of maintainability than have been
available with previous models.

Calculating cognitive complexity in Rust can be achieved with the
\emph{rust-clippy} tool. Clippy is a collection of lints to catch common
mistakes and improve Rust code. There are over 400 lints included in the tool,
and are divided into categories, each with a default lint level. One way to use
Clippy is by installing Clippy through \texttt{rustup} as a Cargo subcommand.
Clippy can then be invoked with the \texttt{cargo clippy} command.

\begin{figure}[H]
  \centering
  %% Creator: Matplotlib, PGF backend
%%
%% To include the figure in your LaTeX document, write
%%   \input{<filename>.pgf}
%%
%% Make sure the required packages are loaded in your preamble
%%   \usepackage{pgf}
%%
%% and, on pdftex
%%   \usepackage[utf8]{inputenc}\DeclareUnicodeCharacter{2212}{-}
%%
%% or, on luatex and xetex
%%   \usepackage{unicode-math}
%%
%% Figures using additional raster images can only be included by \input if
%% they are in the same directory as the main LaTeX file. For loading figures
%% from other directories you can use the `import` package
%%   \usepackage{import}
%%
%% and then include the figures with
%%   \import{<path to file>}{<filename>.pgf}
%%
%% Matplotlib used the following preamble
%%
\begingroup%
\makeatletter%
\begin{pgfpicture}%
\pgfpathrectangle{\pgfpointorigin}{\pgfqpoint{6.000000in}{2.500000in}}%
\pgfusepath{use as bounding box, clip}%
\begin{pgfscope}%
\pgfsetbuttcap%
\pgfsetmiterjoin%
\pgfsetlinewidth{0.000000pt}%
\definecolor{currentstroke}{rgb}{1.000000,1.000000,1.000000}%
\pgfsetstrokecolor{currentstroke}%
\pgfsetstrokeopacity{0.000000}%
\pgfsetdash{}{0pt}%
\pgfpathmoveto{\pgfqpoint{0.000000in}{0.000000in}}%
\pgfpathlineto{\pgfqpoint{6.000000in}{0.000000in}}%
\pgfpathlineto{\pgfqpoint{6.000000in}{2.500000in}}%
\pgfpathlineto{\pgfqpoint{0.000000in}{2.500000in}}%
\pgfpathclose%
\pgfusepath{}%
\end{pgfscope}%
\begin{pgfscope}%
\pgfsetbuttcap%
\pgfsetmiterjoin%
\definecolor{currentfill}{rgb}{1.000000,1.000000,1.000000}%
\pgfsetfillcolor{currentfill}%
\pgfsetlinewidth{0.000000pt}%
\definecolor{currentstroke}{rgb}{0.000000,0.000000,0.000000}%
\pgfsetstrokecolor{currentstroke}%
\pgfsetstrokeopacity{0.000000}%
\pgfsetdash{}{0pt}%
\pgfpathmoveto{\pgfqpoint{0.600000in}{0.450000in}}%
\pgfpathlineto{\pgfqpoint{5.400000in}{0.450000in}}%
\pgfpathlineto{\pgfqpoint{5.400000in}{2.487500in}}%
\pgfpathlineto{\pgfqpoint{0.600000in}{2.487500in}}%
\pgfpathclose%
\pgfusepath{fill}%
\end{pgfscope}%
\begin{pgfscope}%
\pgfpathrectangle{\pgfqpoint{0.600000in}{0.450000in}}{\pgfqpoint{4.800000in}{2.037500in}}%
\pgfusepath{clip}%
\pgfsetbuttcap%
\pgfsetmiterjoin%
\definecolor{currentfill}{rgb}{0.121569,0.466667,0.705882}%
\pgfsetfillcolor{currentfill}%
\pgfsetlinewidth{0.000000pt}%
\definecolor{currentstroke}{rgb}{0.000000,0.000000,0.000000}%
\pgfsetstrokecolor{currentstroke}%
\pgfsetstrokeopacity{0.000000}%
\pgfsetdash{}{0pt}%
\pgfpathmoveto{\pgfqpoint{0.818182in}{0.450000in}}%
\pgfpathlineto{\pgfqpoint{1.003868in}{0.450000in}}%
\pgfpathlineto{\pgfqpoint{1.003868in}{2.390476in}}%
\pgfpathlineto{\pgfqpoint{0.818182in}{2.390476in}}%
\pgfpathclose%
\pgfusepath{fill}%
\end{pgfscope}%
\begin{pgfscope}%
\pgfpathrectangle{\pgfqpoint{0.600000in}{0.450000in}}{\pgfqpoint{4.800000in}{2.037500in}}%
\pgfusepath{clip}%
\pgfsetbuttcap%
\pgfsetmiterjoin%
\definecolor{currentfill}{rgb}{0.121569,0.466667,0.705882}%
\pgfsetfillcolor{currentfill}%
\pgfsetlinewidth{0.000000pt}%
\definecolor{currentstroke}{rgb}{0.000000,0.000000,0.000000}%
\pgfsetstrokecolor{currentstroke}%
\pgfsetstrokeopacity{0.000000}%
\pgfsetdash{}{0pt}%
\pgfpathmoveto{\pgfqpoint{1.050290in}{0.450000in}}%
\pgfpathlineto{\pgfqpoint{1.235977in}{0.450000in}}%
\pgfpathlineto{\pgfqpoint{1.235977in}{1.454889in}}%
\pgfpathlineto{\pgfqpoint{1.050290in}{1.454889in}}%
\pgfpathclose%
\pgfusepath{fill}%
\end{pgfscope}%
\begin{pgfscope}%
\pgfpathrectangle{\pgfqpoint{0.600000in}{0.450000in}}{\pgfqpoint{4.800000in}{2.037500in}}%
\pgfusepath{clip}%
\pgfsetbuttcap%
\pgfsetmiterjoin%
\definecolor{currentfill}{rgb}{0.121569,0.466667,0.705882}%
\pgfsetfillcolor{currentfill}%
\pgfsetlinewidth{0.000000pt}%
\definecolor{currentstroke}{rgb}{0.000000,0.000000,0.000000}%
\pgfsetstrokecolor{currentstroke}%
\pgfsetstrokeopacity{0.000000}%
\pgfsetdash{}{0pt}%
\pgfpathmoveto{\pgfqpoint{1.282398in}{0.450000in}}%
\pgfpathlineto{\pgfqpoint{1.468085in}{0.450000in}}%
\pgfpathlineto{\pgfqpoint{1.468085in}{0.588605in}}%
\pgfpathlineto{\pgfqpoint{1.282398in}{0.588605in}}%
\pgfpathclose%
\pgfusepath{fill}%
\end{pgfscope}%
\begin{pgfscope}%
\pgfpathrectangle{\pgfqpoint{0.600000in}{0.450000in}}{\pgfqpoint{4.800000in}{2.037500in}}%
\pgfusepath{clip}%
\pgfsetbuttcap%
\pgfsetmiterjoin%
\definecolor{currentfill}{rgb}{0.121569,0.466667,0.705882}%
\pgfsetfillcolor{currentfill}%
\pgfsetlinewidth{0.000000pt}%
\definecolor{currentstroke}{rgb}{0.000000,0.000000,0.000000}%
\pgfsetstrokecolor{currentstroke}%
\pgfsetstrokeopacity{0.000000}%
\pgfsetdash{}{0pt}%
\pgfpathmoveto{\pgfqpoint{1.514507in}{0.450000in}}%
\pgfpathlineto{\pgfqpoint{1.700193in}{0.450000in}}%
\pgfpathlineto{\pgfqpoint{1.700193in}{0.484651in}}%
\pgfpathlineto{\pgfqpoint{1.514507in}{0.484651in}}%
\pgfpathclose%
\pgfusepath{fill}%
\end{pgfscope}%
\begin{pgfscope}%
\pgfpathrectangle{\pgfqpoint{0.600000in}{0.450000in}}{\pgfqpoint{4.800000in}{2.037500in}}%
\pgfusepath{clip}%
\pgfsetbuttcap%
\pgfsetmiterjoin%
\definecolor{currentfill}{rgb}{0.121569,0.466667,0.705882}%
\pgfsetfillcolor{currentfill}%
\pgfsetlinewidth{0.000000pt}%
\definecolor{currentstroke}{rgb}{0.000000,0.000000,0.000000}%
\pgfsetstrokecolor{currentstroke}%
\pgfsetstrokeopacity{0.000000}%
\pgfsetdash{}{0pt}%
\pgfpathmoveto{\pgfqpoint{1.746615in}{0.450000in}}%
\pgfpathlineto{\pgfqpoint{1.932302in}{0.450000in}}%
\pgfpathlineto{\pgfqpoint{1.932302in}{0.519303in}}%
\pgfpathlineto{\pgfqpoint{1.746615in}{0.519303in}}%
\pgfpathclose%
\pgfusepath{fill}%
\end{pgfscope}%
\begin{pgfscope}%
\pgfpathrectangle{\pgfqpoint{0.600000in}{0.450000in}}{\pgfqpoint{4.800000in}{2.037500in}}%
\pgfusepath{clip}%
\pgfsetbuttcap%
\pgfsetmiterjoin%
\definecolor{currentfill}{rgb}{0.121569,0.466667,0.705882}%
\pgfsetfillcolor{currentfill}%
\pgfsetlinewidth{0.000000pt}%
\definecolor{currentstroke}{rgb}{0.000000,0.000000,0.000000}%
\pgfsetstrokecolor{currentstroke}%
\pgfsetstrokeopacity{0.000000}%
\pgfsetdash{}{0pt}%
\pgfpathmoveto{\pgfqpoint{1.978723in}{0.450000in}}%
\pgfpathlineto{\pgfqpoint{2.164410in}{0.450000in}}%
\pgfpathlineto{\pgfqpoint{2.164410in}{0.519303in}}%
\pgfpathlineto{\pgfqpoint{1.978723in}{0.519303in}}%
\pgfpathclose%
\pgfusepath{fill}%
\end{pgfscope}%
\begin{pgfscope}%
\pgfpathrectangle{\pgfqpoint{0.600000in}{0.450000in}}{\pgfqpoint{4.800000in}{2.037500in}}%
\pgfusepath{clip}%
\pgfsetbuttcap%
\pgfsetmiterjoin%
\definecolor{currentfill}{rgb}{0.121569,0.466667,0.705882}%
\pgfsetfillcolor{currentfill}%
\pgfsetlinewidth{0.000000pt}%
\definecolor{currentstroke}{rgb}{0.000000,0.000000,0.000000}%
\pgfsetstrokecolor{currentstroke}%
\pgfsetstrokeopacity{0.000000}%
\pgfsetdash{}{0pt}%
\pgfpathmoveto{\pgfqpoint{2.210832in}{0.450000in}}%
\pgfpathlineto{\pgfqpoint{2.396518in}{0.450000in}}%
\pgfpathlineto{\pgfqpoint{2.396518in}{0.450000in}}%
\pgfpathlineto{\pgfqpoint{2.210832in}{0.450000in}}%
\pgfpathclose%
\pgfusepath{fill}%
\end{pgfscope}%
\begin{pgfscope}%
\pgfpathrectangle{\pgfqpoint{0.600000in}{0.450000in}}{\pgfqpoint{4.800000in}{2.037500in}}%
\pgfusepath{clip}%
\pgfsetbuttcap%
\pgfsetmiterjoin%
\definecolor{currentfill}{rgb}{0.121569,0.466667,0.705882}%
\pgfsetfillcolor{currentfill}%
\pgfsetlinewidth{0.000000pt}%
\definecolor{currentstroke}{rgb}{0.000000,0.000000,0.000000}%
\pgfsetstrokecolor{currentstroke}%
\pgfsetstrokeopacity{0.000000}%
\pgfsetdash{}{0pt}%
\pgfpathmoveto{\pgfqpoint{2.442940in}{0.450000in}}%
\pgfpathlineto{\pgfqpoint{2.628627in}{0.450000in}}%
\pgfpathlineto{\pgfqpoint{2.628627in}{0.450000in}}%
\pgfpathlineto{\pgfqpoint{2.442940in}{0.450000in}}%
\pgfpathclose%
\pgfusepath{fill}%
\end{pgfscope}%
\begin{pgfscope}%
\pgfpathrectangle{\pgfqpoint{0.600000in}{0.450000in}}{\pgfqpoint{4.800000in}{2.037500in}}%
\pgfusepath{clip}%
\pgfsetbuttcap%
\pgfsetmiterjoin%
\definecolor{currentfill}{rgb}{0.121569,0.466667,0.705882}%
\pgfsetfillcolor{currentfill}%
\pgfsetlinewidth{0.000000pt}%
\definecolor{currentstroke}{rgb}{0.000000,0.000000,0.000000}%
\pgfsetstrokecolor{currentstroke}%
\pgfsetstrokeopacity{0.000000}%
\pgfsetdash{}{0pt}%
\pgfpathmoveto{\pgfqpoint{2.675048in}{0.450000in}}%
\pgfpathlineto{\pgfqpoint{2.860735in}{0.450000in}}%
\pgfpathlineto{\pgfqpoint{2.860735in}{0.484651in}}%
\pgfpathlineto{\pgfqpoint{2.675048in}{0.484651in}}%
\pgfpathclose%
\pgfusepath{fill}%
\end{pgfscope}%
\begin{pgfscope}%
\pgfpathrectangle{\pgfqpoint{0.600000in}{0.450000in}}{\pgfqpoint{4.800000in}{2.037500in}}%
\pgfusepath{clip}%
\pgfsetbuttcap%
\pgfsetmiterjoin%
\definecolor{currentfill}{rgb}{0.121569,0.466667,0.705882}%
\pgfsetfillcolor{currentfill}%
\pgfsetlinewidth{0.000000pt}%
\definecolor{currentstroke}{rgb}{0.000000,0.000000,0.000000}%
\pgfsetstrokecolor{currentstroke}%
\pgfsetstrokeopacity{0.000000}%
\pgfsetdash{}{0pt}%
\pgfpathmoveto{\pgfqpoint{2.907157in}{0.450000in}}%
\pgfpathlineto{\pgfqpoint{3.092843in}{0.450000in}}%
\pgfpathlineto{\pgfqpoint{3.092843in}{0.450000in}}%
\pgfpathlineto{\pgfqpoint{2.907157in}{0.450000in}}%
\pgfpathclose%
\pgfusepath{fill}%
\end{pgfscope}%
\begin{pgfscope}%
\pgfpathrectangle{\pgfqpoint{0.600000in}{0.450000in}}{\pgfqpoint{4.800000in}{2.037500in}}%
\pgfusepath{clip}%
\pgfsetbuttcap%
\pgfsetmiterjoin%
\definecolor{currentfill}{rgb}{0.121569,0.466667,0.705882}%
\pgfsetfillcolor{currentfill}%
\pgfsetlinewidth{0.000000pt}%
\definecolor{currentstroke}{rgb}{0.000000,0.000000,0.000000}%
\pgfsetstrokecolor{currentstroke}%
\pgfsetstrokeopacity{0.000000}%
\pgfsetdash{}{0pt}%
\pgfpathmoveto{\pgfqpoint{3.139265in}{0.450000in}}%
\pgfpathlineto{\pgfqpoint{3.324952in}{0.450000in}}%
\pgfpathlineto{\pgfqpoint{3.324952in}{0.450000in}}%
\pgfpathlineto{\pgfqpoint{3.139265in}{0.450000in}}%
\pgfpathclose%
\pgfusepath{fill}%
\end{pgfscope}%
\begin{pgfscope}%
\pgfpathrectangle{\pgfqpoint{0.600000in}{0.450000in}}{\pgfqpoint{4.800000in}{2.037500in}}%
\pgfusepath{clip}%
\pgfsetbuttcap%
\pgfsetmiterjoin%
\definecolor{currentfill}{rgb}{0.121569,0.466667,0.705882}%
\pgfsetfillcolor{currentfill}%
\pgfsetlinewidth{0.000000pt}%
\definecolor{currentstroke}{rgb}{0.000000,0.000000,0.000000}%
\pgfsetstrokecolor{currentstroke}%
\pgfsetstrokeopacity{0.000000}%
\pgfsetdash{}{0pt}%
\pgfpathmoveto{\pgfqpoint{3.371373in}{0.450000in}}%
\pgfpathlineto{\pgfqpoint{3.557060in}{0.450000in}}%
\pgfpathlineto{\pgfqpoint{3.557060in}{0.450000in}}%
\pgfpathlineto{\pgfqpoint{3.371373in}{0.450000in}}%
\pgfpathclose%
\pgfusepath{fill}%
\end{pgfscope}%
\begin{pgfscope}%
\pgfpathrectangle{\pgfqpoint{0.600000in}{0.450000in}}{\pgfqpoint{4.800000in}{2.037500in}}%
\pgfusepath{clip}%
\pgfsetbuttcap%
\pgfsetmiterjoin%
\definecolor{currentfill}{rgb}{0.121569,0.466667,0.705882}%
\pgfsetfillcolor{currentfill}%
\pgfsetlinewidth{0.000000pt}%
\definecolor{currentstroke}{rgb}{0.000000,0.000000,0.000000}%
\pgfsetstrokecolor{currentstroke}%
\pgfsetstrokeopacity{0.000000}%
\pgfsetdash{}{0pt}%
\pgfpathmoveto{\pgfqpoint{3.603482in}{0.450000in}}%
\pgfpathlineto{\pgfqpoint{3.789168in}{0.450000in}}%
\pgfpathlineto{\pgfqpoint{3.789168in}{0.450000in}}%
\pgfpathlineto{\pgfqpoint{3.603482in}{0.450000in}}%
\pgfpathclose%
\pgfusepath{fill}%
\end{pgfscope}%
\begin{pgfscope}%
\pgfpathrectangle{\pgfqpoint{0.600000in}{0.450000in}}{\pgfqpoint{4.800000in}{2.037500in}}%
\pgfusepath{clip}%
\pgfsetbuttcap%
\pgfsetmiterjoin%
\definecolor{currentfill}{rgb}{0.121569,0.466667,0.705882}%
\pgfsetfillcolor{currentfill}%
\pgfsetlinewidth{0.000000pt}%
\definecolor{currentstroke}{rgb}{0.000000,0.000000,0.000000}%
\pgfsetstrokecolor{currentstroke}%
\pgfsetstrokeopacity{0.000000}%
\pgfsetdash{}{0pt}%
\pgfpathmoveto{\pgfqpoint{3.835590in}{0.450000in}}%
\pgfpathlineto{\pgfqpoint{4.021277in}{0.450000in}}%
\pgfpathlineto{\pgfqpoint{4.021277in}{0.450000in}}%
\pgfpathlineto{\pgfqpoint{3.835590in}{0.450000in}}%
\pgfpathclose%
\pgfusepath{fill}%
\end{pgfscope}%
\begin{pgfscope}%
\pgfpathrectangle{\pgfqpoint{0.600000in}{0.450000in}}{\pgfqpoint{4.800000in}{2.037500in}}%
\pgfusepath{clip}%
\pgfsetbuttcap%
\pgfsetmiterjoin%
\definecolor{currentfill}{rgb}{0.121569,0.466667,0.705882}%
\pgfsetfillcolor{currentfill}%
\pgfsetlinewidth{0.000000pt}%
\definecolor{currentstroke}{rgb}{0.000000,0.000000,0.000000}%
\pgfsetstrokecolor{currentstroke}%
\pgfsetstrokeopacity{0.000000}%
\pgfsetdash{}{0pt}%
\pgfpathmoveto{\pgfqpoint{4.067698in}{0.450000in}}%
\pgfpathlineto{\pgfqpoint{4.253385in}{0.450000in}}%
\pgfpathlineto{\pgfqpoint{4.253385in}{0.450000in}}%
\pgfpathlineto{\pgfqpoint{4.067698in}{0.450000in}}%
\pgfpathclose%
\pgfusepath{fill}%
\end{pgfscope}%
\begin{pgfscope}%
\pgfpathrectangle{\pgfqpoint{0.600000in}{0.450000in}}{\pgfqpoint{4.800000in}{2.037500in}}%
\pgfusepath{clip}%
\pgfsetbuttcap%
\pgfsetmiterjoin%
\definecolor{currentfill}{rgb}{0.121569,0.466667,0.705882}%
\pgfsetfillcolor{currentfill}%
\pgfsetlinewidth{0.000000pt}%
\definecolor{currentstroke}{rgb}{0.000000,0.000000,0.000000}%
\pgfsetstrokecolor{currentstroke}%
\pgfsetstrokeopacity{0.000000}%
\pgfsetdash{}{0pt}%
\pgfpathmoveto{\pgfqpoint{4.299807in}{0.450000in}}%
\pgfpathlineto{\pgfqpoint{4.485493in}{0.450000in}}%
\pgfpathlineto{\pgfqpoint{4.485493in}{0.450000in}}%
\pgfpathlineto{\pgfqpoint{4.299807in}{0.450000in}}%
\pgfpathclose%
\pgfusepath{fill}%
\end{pgfscope}%
\begin{pgfscope}%
\pgfpathrectangle{\pgfqpoint{0.600000in}{0.450000in}}{\pgfqpoint{4.800000in}{2.037500in}}%
\pgfusepath{clip}%
\pgfsetbuttcap%
\pgfsetmiterjoin%
\definecolor{currentfill}{rgb}{0.121569,0.466667,0.705882}%
\pgfsetfillcolor{currentfill}%
\pgfsetlinewidth{0.000000pt}%
\definecolor{currentstroke}{rgb}{0.000000,0.000000,0.000000}%
\pgfsetstrokecolor{currentstroke}%
\pgfsetstrokeopacity{0.000000}%
\pgfsetdash{}{0pt}%
\pgfpathmoveto{\pgfqpoint{4.531915in}{0.450000in}}%
\pgfpathlineto{\pgfqpoint{4.717602in}{0.450000in}}%
\pgfpathlineto{\pgfqpoint{4.717602in}{0.450000in}}%
\pgfpathlineto{\pgfqpoint{4.531915in}{0.450000in}}%
\pgfpathclose%
\pgfusepath{fill}%
\end{pgfscope}%
\begin{pgfscope}%
\pgfpathrectangle{\pgfqpoint{0.600000in}{0.450000in}}{\pgfqpoint{4.800000in}{2.037500in}}%
\pgfusepath{clip}%
\pgfsetbuttcap%
\pgfsetmiterjoin%
\definecolor{currentfill}{rgb}{0.121569,0.466667,0.705882}%
\pgfsetfillcolor{currentfill}%
\pgfsetlinewidth{0.000000pt}%
\definecolor{currentstroke}{rgb}{0.000000,0.000000,0.000000}%
\pgfsetstrokecolor{currentstroke}%
\pgfsetstrokeopacity{0.000000}%
\pgfsetdash{}{0pt}%
\pgfpathmoveto{\pgfqpoint{4.764023in}{0.450000in}}%
\pgfpathlineto{\pgfqpoint{4.949710in}{0.450000in}}%
\pgfpathlineto{\pgfqpoint{4.949710in}{0.450000in}}%
\pgfpathlineto{\pgfqpoint{4.764023in}{0.450000in}}%
\pgfpathclose%
\pgfusepath{fill}%
\end{pgfscope}%
\begin{pgfscope}%
\pgfpathrectangle{\pgfqpoint{0.600000in}{0.450000in}}{\pgfqpoint{4.800000in}{2.037500in}}%
\pgfusepath{clip}%
\pgfsetbuttcap%
\pgfsetmiterjoin%
\definecolor{currentfill}{rgb}{0.121569,0.466667,0.705882}%
\pgfsetfillcolor{currentfill}%
\pgfsetlinewidth{0.000000pt}%
\definecolor{currentstroke}{rgb}{0.000000,0.000000,0.000000}%
\pgfsetstrokecolor{currentstroke}%
\pgfsetstrokeopacity{0.000000}%
\pgfsetdash{}{0pt}%
\pgfpathmoveto{\pgfqpoint{4.996132in}{0.450000in}}%
\pgfpathlineto{\pgfqpoint{5.181818in}{0.450000in}}%
\pgfpathlineto{\pgfqpoint{5.181818in}{0.484651in}}%
\pgfpathlineto{\pgfqpoint{4.996132in}{0.484651in}}%
\pgfpathclose%
\pgfusepath{fill}%
\end{pgfscope}%
\begin{pgfscope}%
\pgfsetbuttcap%
\pgfsetroundjoin%
\definecolor{currentfill}{rgb}{0.000000,0.000000,0.000000}%
\pgfsetfillcolor{currentfill}%
\pgfsetlinewidth{0.803000pt}%
\definecolor{currentstroke}{rgb}{0.000000,0.000000,0.000000}%
\pgfsetstrokecolor{currentstroke}%
\pgfsetdash{}{0pt}%
\pgfsys@defobject{currentmarker}{\pgfqpoint{0.000000in}{-0.048611in}}{\pgfqpoint{0.000000in}{0.000000in}}{%
\pgfpathmoveto{\pgfqpoint{0.000000in}{0.000000in}}%
\pgfpathlineto{\pgfqpoint{0.000000in}{-0.048611in}}%
\pgfusepath{stroke,fill}%
}%
\begin{pgfscope}%
\pgfsys@transformshift{0.911025in}{0.450000in}%
\pgfsys@useobject{currentmarker}{}%
\end{pgfscope}%
\end{pgfscope}%
\begin{pgfscope}%
\definecolor{textcolor}{rgb}{0.000000,0.000000,0.000000}%
\pgfsetstrokecolor{textcolor}%
\pgfsetfillcolor{textcolor}%
\pgftext[x=0.911025in,y=0.352778in,,top]{\color{textcolor}\rmfamily\fontsize{10.000000}{12.000000}\selectfont \(\displaystyle {1}\)}%
\end{pgfscope}%
\begin{pgfscope}%
\pgfsetbuttcap%
\pgfsetroundjoin%
\definecolor{currentfill}{rgb}{0.000000,0.000000,0.000000}%
\pgfsetfillcolor{currentfill}%
\pgfsetlinewidth{0.803000pt}%
\definecolor{currentstroke}{rgb}{0.000000,0.000000,0.000000}%
\pgfsetstrokecolor{currentstroke}%
\pgfsetdash{}{0pt}%
\pgfsys@defobject{currentmarker}{\pgfqpoint{0.000000in}{-0.048611in}}{\pgfqpoint{0.000000in}{0.000000in}}{%
\pgfpathmoveto{\pgfqpoint{0.000000in}{0.000000in}}%
\pgfpathlineto{\pgfqpoint{0.000000in}{-0.048611in}}%
\pgfusepath{stroke,fill}%
}%
\begin{pgfscope}%
\pgfsys@transformshift{1.143133in}{0.450000in}%
\pgfsys@useobject{currentmarker}{}%
\end{pgfscope}%
\end{pgfscope}%
\begin{pgfscope}%
\definecolor{textcolor}{rgb}{0.000000,0.000000,0.000000}%
\pgfsetstrokecolor{textcolor}%
\pgfsetfillcolor{textcolor}%
\pgftext[x=1.143133in,y=0.352778in,,top]{\color{textcolor}\rmfamily\fontsize{10.000000}{12.000000}\selectfont \(\displaystyle {2}\)}%
\end{pgfscope}%
\begin{pgfscope}%
\pgfsetbuttcap%
\pgfsetroundjoin%
\definecolor{currentfill}{rgb}{0.000000,0.000000,0.000000}%
\pgfsetfillcolor{currentfill}%
\pgfsetlinewidth{0.803000pt}%
\definecolor{currentstroke}{rgb}{0.000000,0.000000,0.000000}%
\pgfsetstrokecolor{currentstroke}%
\pgfsetdash{}{0pt}%
\pgfsys@defobject{currentmarker}{\pgfqpoint{0.000000in}{-0.048611in}}{\pgfqpoint{0.000000in}{0.000000in}}{%
\pgfpathmoveto{\pgfqpoint{0.000000in}{0.000000in}}%
\pgfpathlineto{\pgfqpoint{0.000000in}{-0.048611in}}%
\pgfusepath{stroke,fill}%
}%
\begin{pgfscope}%
\pgfsys@transformshift{1.375242in}{0.450000in}%
\pgfsys@useobject{currentmarker}{}%
\end{pgfscope}%
\end{pgfscope}%
\begin{pgfscope}%
\definecolor{textcolor}{rgb}{0.000000,0.000000,0.000000}%
\pgfsetstrokecolor{textcolor}%
\pgfsetfillcolor{textcolor}%
\pgftext[x=1.375242in,y=0.352778in,,top]{\color{textcolor}\rmfamily\fontsize{10.000000}{12.000000}\selectfont \(\displaystyle {3}\)}%
\end{pgfscope}%
\begin{pgfscope}%
\pgfsetbuttcap%
\pgfsetroundjoin%
\definecolor{currentfill}{rgb}{0.000000,0.000000,0.000000}%
\pgfsetfillcolor{currentfill}%
\pgfsetlinewidth{0.803000pt}%
\definecolor{currentstroke}{rgb}{0.000000,0.000000,0.000000}%
\pgfsetstrokecolor{currentstroke}%
\pgfsetdash{}{0pt}%
\pgfsys@defobject{currentmarker}{\pgfqpoint{0.000000in}{-0.048611in}}{\pgfqpoint{0.000000in}{0.000000in}}{%
\pgfpathmoveto{\pgfqpoint{0.000000in}{0.000000in}}%
\pgfpathlineto{\pgfqpoint{0.000000in}{-0.048611in}}%
\pgfusepath{stroke,fill}%
}%
\begin{pgfscope}%
\pgfsys@transformshift{1.607350in}{0.450000in}%
\pgfsys@useobject{currentmarker}{}%
\end{pgfscope}%
\end{pgfscope}%
\begin{pgfscope}%
\definecolor{textcolor}{rgb}{0.000000,0.000000,0.000000}%
\pgfsetstrokecolor{textcolor}%
\pgfsetfillcolor{textcolor}%
\pgftext[x=1.607350in,y=0.352778in,,top]{\color{textcolor}\rmfamily\fontsize{10.000000}{12.000000}\selectfont \(\displaystyle {4}\)}%
\end{pgfscope}%
\begin{pgfscope}%
\pgfsetbuttcap%
\pgfsetroundjoin%
\definecolor{currentfill}{rgb}{0.000000,0.000000,0.000000}%
\pgfsetfillcolor{currentfill}%
\pgfsetlinewidth{0.803000pt}%
\definecolor{currentstroke}{rgb}{0.000000,0.000000,0.000000}%
\pgfsetstrokecolor{currentstroke}%
\pgfsetdash{}{0pt}%
\pgfsys@defobject{currentmarker}{\pgfqpoint{0.000000in}{-0.048611in}}{\pgfqpoint{0.000000in}{0.000000in}}{%
\pgfpathmoveto{\pgfqpoint{0.000000in}{0.000000in}}%
\pgfpathlineto{\pgfqpoint{0.000000in}{-0.048611in}}%
\pgfusepath{stroke,fill}%
}%
\begin{pgfscope}%
\pgfsys@transformshift{1.839458in}{0.450000in}%
\pgfsys@useobject{currentmarker}{}%
\end{pgfscope}%
\end{pgfscope}%
\begin{pgfscope}%
\definecolor{textcolor}{rgb}{0.000000,0.000000,0.000000}%
\pgfsetstrokecolor{textcolor}%
\pgfsetfillcolor{textcolor}%
\pgftext[x=1.839458in,y=0.352778in,,top]{\color{textcolor}\rmfamily\fontsize{10.000000}{12.000000}\selectfont \(\displaystyle {5}\)}%
\end{pgfscope}%
\begin{pgfscope}%
\pgfsetbuttcap%
\pgfsetroundjoin%
\definecolor{currentfill}{rgb}{0.000000,0.000000,0.000000}%
\pgfsetfillcolor{currentfill}%
\pgfsetlinewidth{0.803000pt}%
\definecolor{currentstroke}{rgb}{0.000000,0.000000,0.000000}%
\pgfsetstrokecolor{currentstroke}%
\pgfsetdash{}{0pt}%
\pgfsys@defobject{currentmarker}{\pgfqpoint{0.000000in}{-0.048611in}}{\pgfqpoint{0.000000in}{0.000000in}}{%
\pgfpathmoveto{\pgfqpoint{0.000000in}{0.000000in}}%
\pgfpathlineto{\pgfqpoint{0.000000in}{-0.048611in}}%
\pgfusepath{stroke,fill}%
}%
\begin{pgfscope}%
\pgfsys@transformshift{2.071567in}{0.450000in}%
\pgfsys@useobject{currentmarker}{}%
\end{pgfscope}%
\end{pgfscope}%
\begin{pgfscope}%
\definecolor{textcolor}{rgb}{0.000000,0.000000,0.000000}%
\pgfsetstrokecolor{textcolor}%
\pgfsetfillcolor{textcolor}%
\pgftext[x=2.071567in,y=0.352778in,,top]{\color{textcolor}\rmfamily\fontsize{10.000000}{12.000000}\selectfont \(\displaystyle {6}\)}%
\end{pgfscope}%
\begin{pgfscope}%
\pgfsetbuttcap%
\pgfsetroundjoin%
\definecolor{currentfill}{rgb}{0.000000,0.000000,0.000000}%
\pgfsetfillcolor{currentfill}%
\pgfsetlinewidth{0.803000pt}%
\definecolor{currentstroke}{rgb}{0.000000,0.000000,0.000000}%
\pgfsetstrokecolor{currentstroke}%
\pgfsetdash{}{0pt}%
\pgfsys@defobject{currentmarker}{\pgfqpoint{0.000000in}{-0.048611in}}{\pgfqpoint{0.000000in}{0.000000in}}{%
\pgfpathmoveto{\pgfqpoint{0.000000in}{0.000000in}}%
\pgfpathlineto{\pgfqpoint{0.000000in}{-0.048611in}}%
\pgfusepath{stroke,fill}%
}%
\begin{pgfscope}%
\pgfsys@transformshift{2.303675in}{0.450000in}%
\pgfsys@useobject{currentmarker}{}%
\end{pgfscope}%
\end{pgfscope}%
\begin{pgfscope}%
\definecolor{textcolor}{rgb}{0.000000,0.000000,0.000000}%
\pgfsetstrokecolor{textcolor}%
\pgfsetfillcolor{textcolor}%
\pgftext[x=2.303675in,y=0.352778in,,top]{\color{textcolor}\rmfamily\fontsize{10.000000}{12.000000}\selectfont \(\displaystyle {7}\)}%
\end{pgfscope}%
\begin{pgfscope}%
\pgfsetbuttcap%
\pgfsetroundjoin%
\definecolor{currentfill}{rgb}{0.000000,0.000000,0.000000}%
\pgfsetfillcolor{currentfill}%
\pgfsetlinewidth{0.803000pt}%
\definecolor{currentstroke}{rgb}{0.000000,0.000000,0.000000}%
\pgfsetstrokecolor{currentstroke}%
\pgfsetdash{}{0pt}%
\pgfsys@defobject{currentmarker}{\pgfqpoint{0.000000in}{-0.048611in}}{\pgfqpoint{0.000000in}{0.000000in}}{%
\pgfpathmoveto{\pgfqpoint{0.000000in}{0.000000in}}%
\pgfpathlineto{\pgfqpoint{0.000000in}{-0.048611in}}%
\pgfusepath{stroke,fill}%
}%
\begin{pgfscope}%
\pgfsys@transformshift{2.535783in}{0.450000in}%
\pgfsys@useobject{currentmarker}{}%
\end{pgfscope}%
\end{pgfscope}%
\begin{pgfscope}%
\definecolor{textcolor}{rgb}{0.000000,0.000000,0.000000}%
\pgfsetstrokecolor{textcolor}%
\pgfsetfillcolor{textcolor}%
\pgftext[x=2.535783in,y=0.352778in,,top]{\color{textcolor}\rmfamily\fontsize{10.000000}{12.000000}\selectfont \(\displaystyle {8}\)}%
\end{pgfscope}%
\begin{pgfscope}%
\pgfsetbuttcap%
\pgfsetroundjoin%
\definecolor{currentfill}{rgb}{0.000000,0.000000,0.000000}%
\pgfsetfillcolor{currentfill}%
\pgfsetlinewidth{0.803000pt}%
\definecolor{currentstroke}{rgb}{0.000000,0.000000,0.000000}%
\pgfsetstrokecolor{currentstroke}%
\pgfsetdash{}{0pt}%
\pgfsys@defobject{currentmarker}{\pgfqpoint{0.000000in}{-0.048611in}}{\pgfqpoint{0.000000in}{0.000000in}}{%
\pgfpathmoveto{\pgfqpoint{0.000000in}{0.000000in}}%
\pgfpathlineto{\pgfqpoint{0.000000in}{-0.048611in}}%
\pgfusepath{stroke,fill}%
}%
\begin{pgfscope}%
\pgfsys@transformshift{2.767892in}{0.450000in}%
\pgfsys@useobject{currentmarker}{}%
\end{pgfscope}%
\end{pgfscope}%
\begin{pgfscope}%
\definecolor{textcolor}{rgb}{0.000000,0.000000,0.000000}%
\pgfsetstrokecolor{textcolor}%
\pgfsetfillcolor{textcolor}%
\pgftext[x=2.767892in,y=0.352778in,,top]{\color{textcolor}\rmfamily\fontsize{10.000000}{12.000000}\selectfont \(\displaystyle {9}\)}%
\end{pgfscope}%
\begin{pgfscope}%
\pgfsetbuttcap%
\pgfsetroundjoin%
\definecolor{currentfill}{rgb}{0.000000,0.000000,0.000000}%
\pgfsetfillcolor{currentfill}%
\pgfsetlinewidth{0.803000pt}%
\definecolor{currentstroke}{rgb}{0.000000,0.000000,0.000000}%
\pgfsetstrokecolor{currentstroke}%
\pgfsetdash{}{0pt}%
\pgfsys@defobject{currentmarker}{\pgfqpoint{0.000000in}{-0.048611in}}{\pgfqpoint{0.000000in}{0.000000in}}{%
\pgfpathmoveto{\pgfqpoint{0.000000in}{0.000000in}}%
\pgfpathlineto{\pgfqpoint{0.000000in}{-0.048611in}}%
\pgfusepath{stroke,fill}%
}%
\begin{pgfscope}%
\pgfsys@transformshift{3.000000in}{0.450000in}%
\pgfsys@useobject{currentmarker}{}%
\end{pgfscope}%
\end{pgfscope}%
\begin{pgfscope}%
\definecolor{textcolor}{rgb}{0.000000,0.000000,0.000000}%
\pgfsetstrokecolor{textcolor}%
\pgfsetfillcolor{textcolor}%
\pgftext[x=3.000000in,y=0.352778in,,top]{\color{textcolor}\rmfamily\fontsize{10.000000}{12.000000}\selectfont \(\displaystyle {10}\)}%
\end{pgfscope}%
\begin{pgfscope}%
\pgfsetbuttcap%
\pgfsetroundjoin%
\definecolor{currentfill}{rgb}{0.000000,0.000000,0.000000}%
\pgfsetfillcolor{currentfill}%
\pgfsetlinewidth{0.803000pt}%
\definecolor{currentstroke}{rgb}{0.000000,0.000000,0.000000}%
\pgfsetstrokecolor{currentstroke}%
\pgfsetdash{}{0pt}%
\pgfsys@defobject{currentmarker}{\pgfqpoint{0.000000in}{-0.048611in}}{\pgfqpoint{0.000000in}{0.000000in}}{%
\pgfpathmoveto{\pgfqpoint{0.000000in}{0.000000in}}%
\pgfpathlineto{\pgfqpoint{0.000000in}{-0.048611in}}%
\pgfusepath{stroke,fill}%
}%
\begin{pgfscope}%
\pgfsys@transformshift{3.232108in}{0.450000in}%
\pgfsys@useobject{currentmarker}{}%
\end{pgfscope}%
\end{pgfscope}%
\begin{pgfscope}%
\definecolor{textcolor}{rgb}{0.000000,0.000000,0.000000}%
\pgfsetstrokecolor{textcolor}%
\pgfsetfillcolor{textcolor}%
\pgftext[x=3.232108in,y=0.352778in,,top]{\color{textcolor}\rmfamily\fontsize{10.000000}{12.000000}\selectfont \(\displaystyle {11}\)}%
\end{pgfscope}%
\begin{pgfscope}%
\pgfsetbuttcap%
\pgfsetroundjoin%
\definecolor{currentfill}{rgb}{0.000000,0.000000,0.000000}%
\pgfsetfillcolor{currentfill}%
\pgfsetlinewidth{0.803000pt}%
\definecolor{currentstroke}{rgb}{0.000000,0.000000,0.000000}%
\pgfsetstrokecolor{currentstroke}%
\pgfsetdash{}{0pt}%
\pgfsys@defobject{currentmarker}{\pgfqpoint{0.000000in}{-0.048611in}}{\pgfqpoint{0.000000in}{0.000000in}}{%
\pgfpathmoveto{\pgfqpoint{0.000000in}{0.000000in}}%
\pgfpathlineto{\pgfqpoint{0.000000in}{-0.048611in}}%
\pgfusepath{stroke,fill}%
}%
\begin{pgfscope}%
\pgfsys@transformshift{3.464217in}{0.450000in}%
\pgfsys@useobject{currentmarker}{}%
\end{pgfscope}%
\end{pgfscope}%
\begin{pgfscope}%
\definecolor{textcolor}{rgb}{0.000000,0.000000,0.000000}%
\pgfsetstrokecolor{textcolor}%
\pgfsetfillcolor{textcolor}%
\pgftext[x=3.464217in,y=0.352778in,,top]{\color{textcolor}\rmfamily\fontsize{10.000000}{12.000000}\selectfont \(\displaystyle {12}\)}%
\end{pgfscope}%
\begin{pgfscope}%
\pgfsetbuttcap%
\pgfsetroundjoin%
\definecolor{currentfill}{rgb}{0.000000,0.000000,0.000000}%
\pgfsetfillcolor{currentfill}%
\pgfsetlinewidth{0.803000pt}%
\definecolor{currentstroke}{rgb}{0.000000,0.000000,0.000000}%
\pgfsetstrokecolor{currentstroke}%
\pgfsetdash{}{0pt}%
\pgfsys@defobject{currentmarker}{\pgfqpoint{0.000000in}{-0.048611in}}{\pgfqpoint{0.000000in}{0.000000in}}{%
\pgfpathmoveto{\pgfqpoint{0.000000in}{0.000000in}}%
\pgfpathlineto{\pgfqpoint{0.000000in}{-0.048611in}}%
\pgfusepath{stroke,fill}%
}%
\begin{pgfscope}%
\pgfsys@transformshift{3.696325in}{0.450000in}%
\pgfsys@useobject{currentmarker}{}%
\end{pgfscope}%
\end{pgfscope}%
\begin{pgfscope}%
\definecolor{textcolor}{rgb}{0.000000,0.000000,0.000000}%
\pgfsetstrokecolor{textcolor}%
\pgfsetfillcolor{textcolor}%
\pgftext[x=3.696325in,y=0.352778in,,top]{\color{textcolor}\rmfamily\fontsize{10.000000}{12.000000}\selectfont \(\displaystyle {13}\)}%
\end{pgfscope}%
\begin{pgfscope}%
\pgfsetbuttcap%
\pgfsetroundjoin%
\definecolor{currentfill}{rgb}{0.000000,0.000000,0.000000}%
\pgfsetfillcolor{currentfill}%
\pgfsetlinewidth{0.803000pt}%
\definecolor{currentstroke}{rgb}{0.000000,0.000000,0.000000}%
\pgfsetstrokecolor{currentstroke}%
\pgfsetdash{}{0pt}%
\pgfsys@defobject{currentmarker}{\pgfqpoint{0.000000in}{-0.048611in}}{\pgfqpoint{0.000000in}{0.000000in}}{%
\pgfpathmoveto{\pgfqpoint{0.000000in}{0.000000in}}%
\pgfpathlineto{\pgfqpoint{0.000000in}{-0.048611in}}%
\pgfusepath{stroke,fill}%
}%
\begin{pgfscope}%
\pgfsys@transformshift{3.928433in}{0.450000in}%
\pgfsys@useobject{currentmarker}{}%
\end{pgfscope}%
\end{pgfscope}%
\begin{pgfscope}%
\definecolor{textcolor}{rgb}{0.000000,0.000000,0.000000}%
\pgfsetstrokecolor{textcolor}%
\pgfsetfillcolor{textcolor}%
\pgftext[x=3.928433in,y=0.352778in,,top]{\color{textcolor}\rmfamily\fontsize{10.000000}{12.000000}\selectfont \(\displaystyle {14}\)}%
\end{pgfscope}%
\begin{pgfscope}%
\pgfsetbuttcap%
\pgfsetroundjoin%
\definecolor{currentfill}{rgb}{0.000000,0.000000,0.000000}%
\pgfsetfillcolor{currentfill}%
\pgfsetlinewidth{0.803000pt}%
\definecolor{currentstroke}{rgb}{0.000000,0.000000,0.000000}%
\pgfsetstrokecolor{currentstroke}%
\pgfsetdash{}{0pt}%
\pgfsys@defobject{currentmarker}{\pgfqpoint{0.000000in}{-0.048611in}}{\pgfqpoint{0.000000in}{0.000000in}}{%
\pgfpathmoveto{\pgfqpoint{0.000000in}{0.000000in}}%
\pgfpathlineto{\pgfqpoint{0.000000in}{-0.048611in}}%
\pgfusepath{stroke,fill}%
}%
\begin{pgfscope}%
\pgfsys@transformshift{4.160542in}{0.450000in}%
\pgfsys@useobject{currentmarker}{}%
\end{pgfscope}%
\end{pgfscope}%
\begin{pgfscope}%
\definecolor{textcolor}{rgb}{0.000000,0.000000,0.000000}%
\pgfsetstrokecolor{textcolor}%
\pgfsetfillcolor{textcolor}%
\pgftext[x=4.160542in,y=0.352778in,,top]{\color{textcolor}\rmfamily\fontsize{10.000000}{12.000000}\selectfont \(\displaystyle {15}\)}%
\end{pgfscope}%
\begin{pgfscope}%
\pgfsetbuttcap%
\pgfsetroundjoin%
\definecolor{currentfill}{rgb}{0.000000,0.000000,0.000000}%
\pgfsetfillcolor{currentfill}%
\pgfsetlinewidth{0.803000pt}%
\definecolor{currentstroke}{rgb}{0.000000,0.000000,0.000000}%
\pgfsetstrokecolor{currentstroke}%
\pgfsetdash{}{0pt}%
\pgfsys@defobject{currentmarker}{\pgfqpoint{0.000000in}{-0.048611in}}{\pgfqpoint{0.000000in}{0.000000in}}{%
\pgfpathmoveto{\pgfqpoint{0.000000in}{0.000000in}}%
\pgfpathlineto{\pgfqpoint{0.000000in}{-0.048611in}}%
\pgfusepath{stroke,fill}%
}%
\begin{pgfscope}%
\pgfsys@transformshift{4.392650in}{0.450000in}%
\pgfsys@useobject{currentmarker}{}%
\end{pgfscope}%
\end{pgfscope}%
\begin{pgfscope}%
\definecolor{textcolor}{rgb}{0.000000,0.000000,0.000000}%
\pgfsetstrokecolor{textcolor}%
\pgfsetfillcolor{textcolor}%
\pgftext[x=4.392650in,y=0.352778in,,top]{\color{textcolor}\rmfamily\fontsize{10.000000}{12.000000}\selectfont \(\displaystyle {16}\)}%
\end{pgfscope}%
\begin{pgfscope}%
\pgfsetbuttcap%
\pgfsetroundjoin%
\definecolor{currentfill}{rgb}{0.000000,0.000000,0.000000}%
\pgfsetfillcolor{currentfill}%
\pgfsetlinewidth{0.803000pt}%
\definecolor{currentstroke}{rgb}{0.000000,0.000000,0.000000}%
\pgfsetstrokecolor{currentstroke}%
\pgfsetdash{}{0pt}%
\pgfsys@defobject{currentmarker}{\pgfqpoint{0.000000in}{-0.048611in}}{\pgfqpoint{0.000000in}{0.000000in}}{%
\pgfpathmoveto{\pgfqpoint{0.000000in}{0.000000in}}%
\pgfpathlineto{\pgfqpoint{0.000000in}{-0.048611in}}%
\pgfusepath{stroke,fill}%
}%
\begin{pgfscope}%
\pgfsys@transformshift{4.624758in}{0.450000in}%
\pgfsys@useobject{currentmarker}{}%
\end{pgfscope}%
\end{pgfscope}%
\begin{pgfscope}%
\definecolor{textcolor}{rgb}{0.000000,0.000000,0.000000}%
\pgfsetstrokecolor{textcolor}%
\pgfsetfillcolor{textcolor}%
\pgftext[x=4.624758in,y=0.352778in,,top]{\color{textcolor}\rmfamily\fontsize{10.000000}{12.000000}\selectfont \(\displaystyle {17}\)}%
\end{pgfscope}%
\begin{pgfscope}%
\pgfsetbuttcap%
\pgfsetroundjoin%
\definecolor{currentfill}{rgb}{0.000000,0.000000,0.000000}%
\pgfsetfillcolor{currentfill}%
\pgfsetlinewidth{0.803000pt}%
\definecolor{currentstroke}{rgb}{0.000000,0.000000,0.000000}%
\pgfsetstrokecolor{currentstroke}%
\pgfsetdash{}{0pt}%
\pgfsys@defobject{currentmarker}{\pgfqpoint{0.000000in}{-0.048611in}}{\pgfqpoint{0.000000in}{0.000000in}}{%
\pgfpathmoveto{\pgfqpoint{0.000000in}{0.000000in}}%
\pgfpathlineto{\pgfqpoint{0.000000in}{-0.048611in}}%
\pgfusepath{stroke,fill}%
}%
\begin{pgfscope}%
\pgfsys@transformshift{4.856867in}{0.450000in}%
\pgfsys@useobject{currentmarker}{}%
\end{pgfscope}%
\end{pgfscope}%
\begin{pgfscope}%
\definecolor{textcolor}{rgb}{0.000000,0.000000,0.000000}%
\pgfsetstrokecolor{textcolor}%
\pgfsetfillcolor{textcolor}%
\pgftext[x=4.856867in,y=0.352778in,,top]{\color{textcolor}\rmfamily\fontsize{10.000000}{12.000000}\selectfont \(\displaystyle {18}\)}%
\end{pgfscope}%
\begin{pgfscope}%
\pgfsetbuttcap%
\pgfsetroundjoin%
\definecolor{currentfill}{rgb}{0.000000,0.000000,0.000000}%
\pgfsetfillcolor{currentfill}%
\pgfsetlinewidth{0.803000pt}%
\definecolor{currentstroke}{rgb}{0.000000,0.000000,0.000000}%
\pgfsetstrokecolor{currentstroke}%
\pgfsetdash{}{0pt}%
\pgfsys@defobject{currentmarker}{\pgfqpoint{0.000000in}{-0.048611in}}{\pgfqpoint{0.000000in}{0.000000in}}{%
\pgfpathmoveto{\pgfqpoint{0.000000in}{0.000000in}}%
\pgfpathlineto{\pgfqpoint{0.000000in}{-0.048611in}}%
\pgfusepath{stroke,fill}%
}%
\begin{pgfscope}%
\pgfsys@transformshift{5.088975in}{0.450000in}%
\pgfsys@useobject{currentmarker}{}%
\end{pgfscope}%
\end{pgfscope}%
\begin{pgfscope}%
\definecolor{textcolor}{rgb}{0.000000,0.000000,0.000000}%
\pgfsetstrokecolor{textcolor}%
\pgfsetfillcolor{textcolor}%
\pgftext[x=5.088975in,y=0.352778in,,top]{\color{textcolor}\rmfamily\fontsize{10.000000}{12.000000}\selectfont \(\displaystyle {19}\)}%
\end{pgfscope}%
\begin{pgfscope}%
\definecolor{textcolor}{rgb}{0.000000,0.000000,0.000000}%
\pgfsetstrokecolor{textcolor}%
\pgfsetfillcolor{textcolor}%
\pgftext[x=3.000000in,y=0.173766in,,top]{\color{textcolor}\rmfamily\fontsize{10.000000}{12.000000}\selectfont Cognitive complexity}%
\end{pgfscope}%
\begin{pgfscope}%
\pgfsetbuttcap%
\pgfsetroundjoin%
\definecolor{currentfill}{rgb}{0.000000,0.000000,0.000000}%
\pgfsetfillcolor{currentfill}%
\pgfsetlinewidth{0.803000pt}%
\definecolor{currentstroke}{rgb}{0.000000,0.000000,0.000000}%
\pgfsetstrokecolor{currentstroke}%
\pgfsetdash{}{0pt}%
\pgfsys@defobject{currentmarker}{\pgfqpoint{-0.048611in}{0.000000in}}{\pgfqpoint{-0.000000in}{0.000000in}}{%
\pgfpathmoveto{\pgfqpoint{-0.000000in}{0.000000in}}%
\pgfpathlineto{\pgfqpoint{-0.048611in}{0.000000in}}%
\pgfusepath{stroke,fill}%
}%
\begin{pgfscope}%
\pgfsys@transformshift{0.600000in}{0.450000in}%
\pgfsys@useobject{currentmarker}{}%
\end{pgfscope}%
\end{pgfscope}%
\begin{pgfscope}%
\definecolor{textcolor}{rgb}{0.000000,0.000000,0.000000}%
\pgfsetstrokecolor{textcolor}%
\pgfsetfillcolor{textcolor}%
\pgftext[x=0.433333in, y=0.401775in, left, base]{\color{textcolor}\rmfamily\fontsize{10.000000}{12.000000}\selectfont \(\displaystyle {0}\)}%
\end{pgfscope}%
\begin{pgfscope}%
\pgfsetbuttcap%
\pgfsetroundjoin%
\definecolor{currentfill}{rgb}{0.000000,0.000000,0.000000}%
\pgfsetfillcolor{currentfill}%
\pgfsetlinewidth{0.803000pt}%
\definecolor{currentstroke}{rgb}{0.000000,0.000000,0.000000}%
\pgfsetstrokecolor{currentstroke}%
\pgfsetdash{}{0pt}%
\pgfsys@defobject{currentmarker}{\pgfqpoint{-0.048611in}{0.000000in}}{\pgfqpoint{-0.000000in}{0.000000in}}{%
\pgfpathmoveto{\pgfqpoint{-0.000000in}{0.000000in}}%
\pgfpathlineto{\pgfqpoint{-0.048611in}{0.000000in}}%
\pgfusepath{stroke,fill}%
}%
\begin{pgfscope}%
\pgfsys@transformshift{0.600000in}{0.796514in}%
\pgfsys@useobject{currentmarker}{}%
\end{pgfscope}%
\end{pgfscope}%
\begin{pgfscope}%
\definecolor{textcolor}{rgb}{0.000000,0.000000,0.000000}%
\pgfsetstrokecolor{textcolor}%
\pgfsetfillcolor{textcolor}%
\pgftext[x=0.363888in, y=0.748288in, left, base]{\color{textcolor}\rmfamily\fontsize{10.000000}{12.000000}\selectfont \(\displaystyle {10}\)}%
\end{pgfscope}%
\begin{pgfscope}%
\pgfsetbuttcap%
\pgfsetroundjoin%
\definecolor{currentfill}{rgb}{0.000000,0.000000,0.000000}%
\pgfsetfillcolor{currentfill}%
\pgfsetlinewidth{0.803000pt}%
\definecolor{currentstroke}{rgb}{0.000000,0.000000,0.000000}%
\pgfsetstrokecolor{currentstroke}%
\pgfsetdash{}{0pt}%
\pgfsys@defobject{currentmarker}{\pgfqpoint{-0.048611in}{0.000000in}}{\pgfqpoint{-0.000000in}{0.000000in}}{%
\pgfpathmoveto{\pgfqpoint{-0.000000in}{0.000000in}}%
\pgfpathlineto{\pgfqpoint{-0.048611in}{0.000000in}}%
\pgfusepath{stroke,fill}%
}%
\begin{pgfscope}%
\pgfsys@transformshift{0.600000in}{1.143027in}%
\pgfsys@useobject{currentmarker}{}%
\end{pgfscope}%
\end{pgfscope}%
\begin{pgfscope}%
\definecolor{textcolor}{rgb}{0.000000,0.000000,0.000000}%
\pgfsetstrokecolor{textcolor}%
\pgfsetfillcolor{textcolor}%
\pgftext[x=0.363888in, y=1.094802in, left, base]{\color{textcolor}\rmfamily\fontsize{10.000000}{12.000000}\selectfont \(\displaystyle {20}\)}%
\end{pgfscope}%
\begin{pgfscope}%
\pgfsetbuttcap%
\pgfsetroundjoin%
\definecolor{currentfill}{rgb}{0.000000,0.000000,0.000000}%
\pgfsetfillcolor{currentfill}%
\pgfsetlinewidth{0.803000pt}%
\definecolor{currentstroke}{rgb}{0.000000,0.000000,0.000000}%
\pgfsetstrokecolor{currentstroke}%
\pgfsetdash{}{0pt}%
\pgfsys@defobject{currentmarker}{\pgfqpoint{-0.048611in}{0.000000in}}{\pgfqpoint{-0.000000in}{0.000000in}}{%
\pgfpathmoveto{\pgfqpoint{-0.000000in}{0.000000in}}%
\pgfpathlineto{\pgfqpoint{-0.048611in}{0.000000in}}%
\pgfusepath{stroke,fill}%
}%
\begin{pgfscope}%
\pgfsys@transformshift{0.600000in}{1.489541in}%
\pgfsys@useobject{currentmarker}{}%
\end{pgfscope}%
\end{pgfscope}%
\begin{pgfscope}%
\definecolor{textcolor}{rgb}{0.000000,0.000000,0.000000}%
\pgfsetstrokecolor{textcolor}%
\pgfsetfillcolor{textcolor}%
\pgftext[x=0.363888in, y=1.441316in, left, base]{\color{textcolor}\rmfamily\fontsize{10.000000}{12.000000}\selectfont \(\displaystyle {30}\)}%
\end{pgfscope}%
\begin{pgfscope}%
\pgfsetbuttcap%
\pgfsetroundjoin%
\definecolor{currentfill}{rgb}{0.000000,0.000000,0.000000}%
\pgfsetfillcolor{currentfill}%
\pgfsetlinewidth{0.803000pt}%
\definecolor{currentstroke}{rgb}{0.000000,0.000000,0.000000}%
\pgfsetstrokecolor{currentstroke}%
\pgfsetdash{}{0pt}%
\pgfsys@defobject{currentmarker}{\pgfqpoint{-0.048611in}{0.000000in}}{\pgfqpoint{-0.000000in}{0.000000in}}{%
\pgfpathmoveto{\pgfqpoint{-0.000000in}{0.000000in}}%
\pgfpathlineto{\pgfqpoint{-0.048611in}{0.000000in}}%
\pgfusepath{stroke,fill}%
}%
\begin{pgfscope}%
\pgfsys@transformshift{0.600000in}{1.836054in}%
\pgfsys@useobject{currentmarker}{}%
\end{pgfscope}%
\end{pgfscope}%
\begin{pgfscope}%
\definecolor{textcolor}{rgb}{0.000000,0.000000,0.000000}%
\pgfsetstrokecolor{textcolor}%
\pgfsetfillcolor{textcolor}%
\pgftext[x=0.363888in, y=1.787829in, left, base]{\color{textcolor}\rmfamily\fontsize{10.000000}{12.000000}\selectfont \(\displaystyle {40}\)}%
\end{pgfscope}%
\begin{pgfscope}%
\pgfsetbuttcap%
\pgfsetroundjoin%
\definecolor{currentfill}{rgb}{0.000000,0.000000,0.000000}%
\pgfsetfillcolor{currentfill}%
\pgfsetlinewidth{0.803000pt}%
\definecolor{currentstroke}{rgb}{0.000000,0.000000,0.000000}%
\pgfsetstrokecolor{currentstroke}%
\pgfsetdash{}{0pt}%
\pgfsys@defobject{currentmarker}{\pgfqpoint{-0.048611in}{0.000000in}}{\pgfqpoint{-0.000000in}{0.000000in}}{%
\pgfpathmoveto{\pgfqpoint{-0.000000in}{0.000000in}}%
\pgfpathlineto{\pgfqpoint{-0.048611in}{0.000000in}}%
\pgfusepath{stroke,fill}%
}%
\begin{pgfscope}%
\pgfsys@transformshift{0.600000in}{2.182568in}%
\pgfsys@useobject{currentmarker}{}%
\end{pgfscope}%
\end{pgfscope}%
\begin{pgfscope}%
\definecolor{textcolor}{rgb}{0.000000,0.000000,0.000000}%
\pgfsetstrokecolor{textcolor}%
\pgfsetfillcolor{textcolor}%
\pgftext[x=0.363888in, y=2.134343in, left, base]{\color{textcolor}\rmfamily\fontsize{10.000000}{12.000000}\selectfont \(\displaystyle {50}\)}%
\end{pgfscope}%
\begin{pgfscope}%
\definecolor{textcolor}{rgb}{0.000000,0.000000,0.000000}%
\pgfsetstrokecolor{textcolor}%
\pgfsetfillcolor{textcolor}%
\pgftext[x=0.308333in,y=1.468750in,,bottom,rotate=90.000000]{\color{textcolor}\rmfamily\fontsize{10.000000}{12.000000}\selectfont Number of functions}%
\end{pgfscope}%
\begin{pgfscope}%
\pgfsetrectcap%
\pgfsetmiterjoin%
\pgfsetlinewidth{0.401500pt}%
\definecolor{currentstroke}{rgb}{0.000000,0.000000,0.000000}%
\pgfsetstrokecolor{currentstroke}%
\pgfsetdash{}{0pt}%
\pgfpathmoveto{\pgfqpoint{0.600000in}{0.450000in}}%
\pgfpathlineto{\pgfqpoint{0.600000in}{2.487500in}}%
\pgfusepath{stroke}%
\end{pgfscope}%
\begin{pgfscope}%
\pgfsetrectcap%
\pgfsetmiterjoin%
\pgfsetlinewidth{0.401500pt}%
\definecolor{currentstroke}{rgb}{0.000000,0.000000,0.000000}%
\pgfsetstrokecolor{currentstroke}%
\pgfsetdash{}{0pt}%
\pgfpathmoveto{\pgfqpoint{5.400000in}{0.450000in}}%
\pgfpathlineto{\pgfqpoint{5.400000in}{2.487500in}}%
\pgfusepath{stroke}%
\end{pgfscope}%
\begin{pgfscope}%
\pgfsetrectcap%
\pgfsetmiterjoin%
\pgfsetlinewidth{0.401500pt}%
\definecolor{currentstroke}{rgb}{0.000000,0.000000,0.000000}%
\pgfsetstrokecolor{currentstroke}%
\pgfsetdash{}{0pt}%
\pgfpathmoveto{\pgfqpoint{0.600000in}{0.450000in}}%
\pgfpathlineto{\pgfqpoint{5.400000in}{0.450000in}}%
\pgfusepath{stroke}%
\end{pgfscope}%
\begin{pgfscope}%
\pgfsetrectcap%
\pgfsetmiterjoin%
\pgfsetlinewidth{0.401500pt}%
\definecolor{currentstroke}{rgb}{0.000000,0.000000,0.000000}%
\pgfsetstrokecolor{currentstroke}%
\pgfsetdash{}{0pt}%
\pgfpathmoveto{\pgfqpoint{0.600000in}{2.487500in}}%
\pgfpathlineto{\pgfqpoint{5.400000in}{2.487500in}}%
\pgfusepath{stroke}%
\end{pgfscope}%
\end{pgfpicture}%
\makeatother%
\endgroup%

  \caption{Number of functions with various degrees of cognitive complexity.}
  \label{fig:cognitive-complexity}
\end{figure}

Figure~\ref{fig:cognitive-complexity} shows the bar plot of a number of
functions of a particular cognitive complexity value. Most of functions in the
project have a low cognitive complexity of around 1 or 2. Acceptable values for
cognitive complexity of functions generally are in the range between 0 and 10.
Based on this, there is one function with value above that range --- the
\texttt{lex} function. However, the complexity of the \texttt{lex} function is
deliberate --- the lexer was written with complex behavior in a single procedure
in mind, for performance reasons. The written code is simple and low-risk when
it comes to maintenance.

\newpage

\chapter{Deployment} \label{ch:deployment}

\section{Application building}

Deployment involves packaging up the web application and putting it in a
production environment that can run the app. To package, or \emph{bundle} the
web application, \thisproject{} uses Rollup --- a simple JavaScript module
bundler, described in Section~\ref{sbs:used-technologies}, which can also serve
as a \emph{build tool} for JavaScript. A bundler is a tool that recursively
follows all imports from the entry point of the application and bundles them up
into a single JavaScript file. What makes Rollup great, is its ability to keep
files small --- compared to the other tools for creating JavaScript bundles,
Rollup will almost always generate a smaller, faster bundle. This happens,
because Rollup is based on ES2015 modules, which are more efficient than
CommonJS modules, which are what other bundlers use. It's also much easier for
Rollup to remove unused code from modules using a technique called
\emph{tree-shaking}, thanks to which only the code the application actually
needs is included in the final bundle. Tree-shaking becomes important when the
application includes third-party tools or frameworks that have dozens of
functions and methods available.

Rollup uses the \texttt{rollup.config.js} configuration file, which includes
such configuration options as the entry point, the destination folder, the
output format, but most importantly it allows to include plugins, particularly
for preprocessing Svelte files, including CSS files, aliasing, static file
copying, and wasm-pack support for building Rust project to WebAssembly.

To bundle the web application, the only command the developer has to invoke is
\texttt{rollup -c}, which is aliased in the \texttt{package.json} file to the
\texttt{build} script for accessibility. The command will generate a new folder
called \texttt{dist} in the project, that contains all generated files ready for
deployment.

\section{Production environment}

GitHub Pages are public web pages for users, organizations, and repositories,
that are freely hosted on GitHub's \texttt{github.io} domain or on a custom
domain name of choice. GitHub users can design and host both personal websites
and websites related to specific GitHub projects. GitHub Pages provides each
user with one user page (\texttt{username.github.io}) and an unlimited number of
project pages (\texttt{username.github.io/projectname}). Each organization can
also have one organization site and an unlimited number of project pages.
\thisproject{}'s code is hosted on the \texttt{karolbelina/parser-parser}
repository, which means that the GitHub Pages page, hosted in the same
repository, is available on
\begin{center}
  \url{https://karolbelina.github.io/parser-parser/}
\end{center}
The developer has to decide how to organize the code so that files with ordinary
application code are separated from files necessary for deployment on GitHub
Pages. User and organization pages must be loaded from the
\texttt{main}/\texttt{master} branch, and project pages will work from any
branch, but by convention it's usually called \texttt{gh-pages}.

To deploy the bundled web application to GitHub pages, \thisproject{} uses the
gh-pages npm package, which with a single command allows to push desired files
to any git branch of the current repository. By default, the gh-pages package we
will push the file tree to the \texttt{gh-pages} branch and make one if
necessary. This makes the main branch and all other branches stay untouched
without the need to explicitly \emph{checkout} the \texttt{gh-pages} branch.

GitHub Pages and its \texttt{github.io} domains provide HTTPS, that adds a layer
of encryption that prevents others from tampering with traffic to the site. All
GitHub Pages sites, including sites that are correctly configured with a custom
domain, support HTTPS and HTTPS enforcement. HTTPS enforcement is required for
GitHub Pages sites using a \texttt{github.io} domain that were made after June
15, 2016. The user can customize the domain name of a GitHub Pages site, however
this was not necessary in the final product.

\section{Continuous integration and continuous deployment}

Continuous integration allows different developers to upload and merge code
changes in the same repository branch on a frequent basis. Once the code has
been uploaded, it's validated automatically by means of unit and integration
tests (both the uploaded code and the rest of the components of the
application). In case of an error, this can be corrected more simply.

In continuous delivery, the new code that has been introduced and that has
passed the CI process is automatically published in a production environment.
What's intended is that the repository code is always in a state that allows its
implementation in a production environment. In continuous deployment, all
changes in the code that have gone through the previous two stages are
automatically implemented in production.

\thisproject{} uses GitHub Actions --- a new tool to perform the CI/CD process,
introduced in October 2018, launched in beta in August 2019, and finally
distributed in November 2019. GitHub Actions is also paid, although it has a
free version with some limitations.

To set up a GitHub Action workflow, the developer generates an appropriate file
--- in the case of CI/CD (Continuous integration and continuous deployment),
which is the only workflow in the project, the file is named \texttt{ci.yml} and
sits in the \texttt{.github} directory at the root of the repository. The format
of the workflow file is YAML --- a human-readable data-serialization language
that is commonly used for configuration files. A part of the file can be seen in
Listing~\ref{lst:cd}. With the \texttt{on} keyword it lists the events that
invoke the workflow --- in this case, any push or pull request to the
\emph{master} branch.

Each workflow is made up of one or more jobs. The \texttt{runs-on} parameter
contains the type of virtual machine in which the code will be executed. In this
case, the workflow uses \texttt{ubuntu-latest} --- the latest supported version
of the Ubuntu operating system. The \texttt{steps} parameter is a sequence of
tasks --- in the example of CI/CD, the first step is to \emph{checkout} the
repository so the workflow can access it. Then, the workflow sets up the
toolchain: it installs Node.js, Rust, and wasm-pack, to finally install the npm
dependencies via \texttt{npm ci} (which is usually faster than \texttt{npm
install}), build the project with \texttt{npm run build}, and deploy it to
GitHub Pages using the \texttt{npm run deploy} command. The deployment is
preceded with an appropriate Git setup, where the workflow configures Git
username and e-mail to ``fake'' credentials of a GitHub Actions bot, as well as
a GitHub token to authorize the push to the \texttt{gh-pages} branch. The token
is stored in the repository via \emph{the secrets}, which allow the developers
to store sensitive information in the repository or organization, and are
available to use in GitHub Actions workflows. GitHub uses a \emph{libsodium
sealed box} to help ensure that secrets are encrypted before they reach GitHub,
and remain encrypted until the developer uses them in a workflow. Only people
with access to the repository can view the secrets.

The workflow uses third party GitHub Actions to set up the toolchain. The
\verb|setup-node@v1| action installs an appropriate version of Node.js along
with npm. The \verb|actions-rs/toolchain@v1| sets up the nightly version of
the Rust compiler and the Cargo dependency manager. Finally,
\verb|jetli/wasm-pack-action@v0.3.0| installs the latest version of wasm-pack
for the Web\-Assembly compilation.

\begin{listing}[H]
  \inputminted[fontsize=\footnotesize,frame=lines,breaklines,linenos]{yaml}
    {listings/cd.yaml}
  \caption{The deployment part of the GitHub Action workflow file related to
  CI/CD.}
  \label{lst:cd}
\end{listing}

\chapter{Software artifacts} \label{ch:software-artifacts}

Artifacts described in this chapter are by-products of the design,
implementation, and testing phases described in
Chapters~\ref{ch:design-of-the-project}, \ref{ch:implementation-of-the-project},
and \ref{ch:project-quality-study}, which are the core of the software
development part of the project.

The software artifact provides a template to build on to allow continuity
instead of starting from scratch or terminating the software --- every program
requires a constant update for continuity, lest they become redundant. The need
to continually do software updates can become a significant source of work
overload if the developer has to begin the developmental process all over again.
However, using artifacts allow developers to have a stored template to fall back
on when they need to update or upgrade their programs.

\subsection*{Design document}

This thesis serves as the design document of the project. It describes the use
cases --- descriptions of how users are meant to perform tasks on the website.
These relate directly to the function of the site, making them important
artifacts. UML diagrams are a way of visualizing and plotting out the way a
piece of software works. It works to map out links, processes, etc. Like use
cases, UML doesn't directly help software run, but it's a key step in designing
and programming a piece of software. Sequence diagrams are a way to map out the
structure of the application --- they are used to map out links and processes
that happen between clicks in a more visual way. Diagrams are a great way to
visualize the internal processes of a given program. These will be designed
throughout the coding process, particularly in the preliminary stages.

\subsection*{Source code}

The source code of the application was produced throughout the whole development
period. It encapsulates the functionalities of the application defined in the
design document, as well as test suites --- coded test to run against the
program in order to make sure a certain process is working. The project is
open-source and is available on a public GitHub repository at
\begin{center}
  \url{https://github.com/karolbelina/parser-parser}
\end{center}

The source code consists of three Rust crates and one npm package, as seen in
Section~\ref{sbs:project-structure}. Most of these work together to compose the
single web application, with the npm package providing the tool chain necessary
to build the project. The Rollup configuration provides the build tool necessary
to bundle the source code in the form of Rust projects and Svelte files into a
compiled web application.

The source code provides the testing infrastructure for the \texttt{ebnf} crate,
with unit tests defined for major components of the system, along with
integration tests which focus on the whole functionality compiled to a
WebAssembly module. With a code artifact, a software programmer can test the
program in detail and perfect things before launching the software. It defines
benchmarks for the parsing and checking functionality of the \texttt{ebnf} crate
with the use of Criterion.rs.

The definition of the GitHub Actions workflow in the \text{ci.yml} file along
with the \texttt{Dockerfile} show the build process for the web application that
may be utilized by deploying the final product in production, as well as to
ascertain the compatibility of each developed program with a specific machine.

\subsection*{Compiled applications}

The web application is the source code compiled into a usable application. This
is the final artifact, and one of the only ones a typical user will care about.
The compiled application will let the user access it on their machine, and use
it as its meant to be used. The command-line application is an auxillary
compiled artifact of the project, that may also be used by the user by
downloading the binary of the program and using it on their own machine.

\subsection*{Documentation}

The standard Rust distribution ships with a tool called \texttt{rustdoc}. Its
job is to generate documentation for Rust projects. On a fundamental level,
Rustdoc takes as an argument either a crate root or a Markdown file, and
produces HTML, CSS, and JavaScript. The documentation can be built for all
crates that are specified in the Cargo.toml by using the \texttt{cargo doc}
command. This compiles into documentation the \emph{doc comments} in functions
or modules, which are very useful for big projects that require documentation.
They are denoted by a ``\texttt{///}'', and support Markdown. The compiled documentation
is put into the generated HTML documentation in the \texttt{target/doc}
directory. For convenience, running \texttt{cargo doc --open} will build the
HTML for the current crate's documentation (as well as the documentation for
all of your crate's dependencies) and open the result in a web browser.

\chapter{User manual} \label{ch:user-manual}

\section{System requirements}

The development process gives the user much flexibility when it comes to the
choice of an operating system. Any tool in the project's toolchain is available
on all major operating systems, whether it's Windows, GNU/Linux or macOS.

To install Rust along with Cargo please follow the official installation guide
for your operating system available at
\begin{center}
  \url{https://www.rust-lang.org/tools/install}
\end{center}
Node.js and npm can be downloaded through
\begin{center}
  \url{https://www.npmjs.com/get-npm}
\end{center}
Wasm-pack's installer and installation guide can be found at
\begin{center}
  \url{https://rustwasm.github.io/wasm-pack/installer/}
\end{center}
All three tools must be installed on the machine of the user in order to run the
application locally.

\section{Installation guide}

The source code of the application can be obtained either from the CD appended
to the thesis, or from the official repository hosted on GitHub at
\begin{center}
  \url{https://github.com/karolbelina/parser-parser}
\end{center}
which can be downloaded as a ZIP file, or cloned from the command-line with
the \texttt{git clone} command (in this case, make sure that Git is installed).

After downloading, please make sure all necessary components are in place. To
list all files in the project's directory, first change the current directory to
\texttt{parser-parser} with the \texttt{cd} command and then list all files
using the \texttt{ls} command on Unix-like systems and \texttt{dir} on Windows.
Listing~\ref{lst:git-download} shows the process of downloading the repository
from GitHub and listing all files in that repository from the command-line.
\begin{listing}[H]
  \begin{minted}[fontsize=\small,frame=lines,breaklines]{text}
$ git clone https://github.com/karolbelina/parser-parser.git
$ cd parser-parser
$ ls
Dockerfile  LICENSE  README.md  app  base  cli  ebnf
  \end{minted}
  \caption{Command-line output after downloading the source code on
  Ubuntu~20.04.}
  \label{lst:git-download}
\end{listing}

To run the web application, the user must first install npm dependencies of the
\texttt{app} component of the project. To do this, one must change the current
directory to \texttt{app}, and then, by using npm, install the necessary
dependencies with the \texttt{npm install} command. This requires a connection
with the internet and will take several minutes to complete. To finally build
and run the application, the user must run the \texttt{npm run start} command,
which will start the build process of the web application. This will build all
Rust components of the application, as well as the Svelte application itself,
and can take several additional minutes. After completion, the user should see
the ``\texttt{Your application is ready}'' notification, and below, the local IP
address where the application is currently being served, as seen in
Listing~\ref{lst:npm-install-build}. To see the application, the user should
copy the address and paste it into the address bar in any supported web browser.
\begin{listing}[H]
  \begin{minted}[fontsize=\small,frame=lines,breaklines]{text}
$ cd app
$ npm install
$ npm run build

Your application is ready~!

- Local:      http://localhost:5000
- Network:    Add `--host` to expose

------------------ LOGS ------------------
  \end{minted}
  \caption{Installation of the dependencies and building the web application.}
  \label{lst:npm-install-build}
\end{listing}

\section{Usage guide}

\begin{figure}[ht]
  \centering
  \begin{tikzpicture}
    \node[inner sep=0pt] (russell) at (0,0)
      {\frame{\includegraphics[width=\textwidth]{parser_parser.png}}};
    \node[] at (-0.25\textwidth,-1.3) {\Huge\bfseries\textcolor{red}{1}};
    \node[] at (4.3,2.7) {\Huge\bfseries\textcolor{red}{2}};
    \node[] at (0.25\textwidth,1.7) {\Huge\bfseries\textcolor{red}{3}};
    \node[] at (0.25\textwidth,-1.3) {\Huge\bfseries\textcolor{red}{4}};
  \end{tikzpicture}
  \caption{Screenshot of the \thisproject{} web application with annotations in
  red.}
  \label{fig:parser-parser-screenshot}
\end{figure}

Figure~\ref{fig:parser-parser-screenshot} shows the web application served at
the local address. The application window in divided into several sections,
all of which have been annotated on Figure~\ref{fig:parser-parser-screenshot}:
\begin{enumerate}
  \item The grammar window, where the user can provide a grammar definition
  in the EBNF format.
  \item The initial production rule selection. After providing a valid grammar,
  the user can select an initial production rule from the ones provided.
  \item The test string window, where the user, after providing a valid grammar,
  can test whether an arbitrary input string belongs to the language generated
  by that grammar.
  \item The visualization of the parse tree, shown only if the user provides a
  string that belongs to the language generated by the grammar. The
  visualization is interactive --- the nodes of the parse tree can be collapsed
  and expanded.
\end{enumerate}

The screenshot presents an example use case of the application, in which the
user clicked on the grammar window (1) with the left mouse button and began to
type the definition of a desired grammar with the use of the keyboard. The
grammar, defined as
\begin{minted}[fontsize=\small,frame=lines,breaklines,linenos]{lexers/ebnf_lexer.py:EbnfLexer -x}
expression = term, { ('+' | '-'), term };
term       = factor, { ('*' | '/'), factor };
factor     = constant
           | variable
           | '(', expression, ')';
variable   = 'x' | 'y' | 'z';
constant   = digit, { digit };
digit      = '0' | '1' | '2' | '3' | '4'
           | '5' | '6' | '7' | '8' | '9';
\end{minted}
is able to generate simple arithmetic expressions made out of natural numbers
and variables $x$, $y$, and $z$. The supported operators are: addition,
subtraction, multiplication, and division, all of which keep an appropriate
operator precedence order. The expressions can be parenthesized to alter the
natural order of precedence. The precedence support for operators is built into
the definition of the grammar --- it is not a special behavior of
\thisproject{}.

After providing the grammar, the user clicked on the initial rule selection (2)
and selected the initial production rule to be the ``\texttt{expression}'' rule.
The user had the option to select any of the six rules defined in the grammar,
which are updated in real-time, but has chosen the rule related to expressions.
In the test string window (3) the user provided a test string ``\texttt{2+2*2}''
to check if it belongs to the language generated by the previously provided
grammar.

\begin{figure}[H]
  \centering
  \begin{subfigure}[b]{0.49\textwidth}
    \centering
    \begin{tikzpicture}
      \node[circle,draw](z){$+$}
        child{node[circle,draw]{$1$}
          child[missing]
          child[missing]
        }
        child{node[circle,draw]{$*$}
          child{node[circle,draw] {$2$}}
          child{node[circle,draw] {$x$}}
        };
    \end{tikzpicture}
    \caption{Expected parse tree structure.}
    \label{fig:expected-parse-tree}
  \end{subfigure}
  \hfill
  \begin{subfigure}[b]{0.49\textwidth}
    \centering
    \frame{\includegraphics[width=\textwidth]{parse_tree_1.png}}
    \caption{Parse tree in \thisproject{}.}
    \label{fig:actual-parse-tree}
  \end{subfigure}
  \caption{A parse tree of a simple arithmetic expression and its operator
  precedence.}
  \label{fig:parser-parser-parse-tree-example}
\end{figure}

As the user typed the input string, the visualization of the parse tree (4) was
being updated. Initially, the output showed ``failure'', as the initial empty
string did not belong to the language, but as the user typed more and more
characters into the test string window, the output window updated accordingly.
By the time the user typed the entire desired test string, the output window
showed a visualization of the parse tree of the ``\texttt{2+2*2}'' expression.
Figure~\ref{fig:parser-parser-parse-tree-example} shows an expected,
mathematically correct parse tree of a slightly altered $1 + 2 \cdot x$
expression, as well as the parse tree of the same expression ``\texttt{1+2*x}''
in \thisproject{}. The expected parse tree and the one generated in
\thisproject{} have the same structure, where the multiplication operation would
be evaluated first, as it is one level below the addition operator.

The user had the option to collapse or expand individual nodes of the parse tree
dynamically in case the resulting parse tree was too large to visualize as a
whole. At any point, the user had the option to change the initial production
rule of the grammar, as well as the grammar itself.

\begin{figure}[H]
  \centering
  \begin{subfigure}[b]{0.49\textwidth}
    \centering
    \frame{\includegraphics[width=\textwidth]{parser_parser_undefined_rule.png}}
    \caption{Undefined rule detection.}
    \label{fig:undefined-rule-detection}
  \end{subfigure}
  \hfill
  \begin{subfigure}[b]{0.49\textwidth}
    \centering
    \frame{\includegraphics[width=\textwidth]{parser_parser_left_recursion.png}}
    \caption{Left recursion detection and its cycle.}
    \label{fig:left-recursion-detection}
  \end{subfigure}
  \hfill
  \begin{subfigure}[b]{0.49\textwidth}
    \centering
    \frame{\includegraphics[width=\textwidth]{parser_parser_repetition_symbol_expected.png}}
    \caption{Simple syntax error.}
    \label{fig:simple-syntax-error}
  \end{subfigure}
  \caption{Three examples of errors detected by \thisproject{}.}
  \label{fig:parser-parser-error-examples}
\end{figure}

As the user provides a definition of a grammar into the grammar window, the
syntax and semantics of that grammar are being automatically validated. In case
the definition is not valid, the reason of that is automatically reported to the
user. Figure~\ref{fig:parser-parser-error-examples} shows three examples of
invalid grammars, where the exact reason and its place in the definition is
highlighted by underlining it with the red color. The user can then hover the
cursor over the highlighted area to read the error, and then correct it
accordingly.

As long as the definition of the grammar is invalid, the user cannot choose
an initial production rule or test an input string.

\chapter{Summary} \label{ch:summary}

The thesis described the process of designing and implementing a tool for
formulating and visualizing context-free grammars in the Extended Backus-Naur
Form. First, it provided a basic analysis of the advantages of the Extended
Backus-Naur Form notation over its predecessor --- Backus-Naur Form. It
transformed the official specification of the ISO/IEC 14977 standard by removing
inaccuracies and ambiguities, as well as to make it more suitable for modern
software environments by providing support for Unicode as the base character set
of the specification. A definition of tokens, as well as the grammar in the form
of an abstract syntax tree, was proposed to specify the structure of the
Extended Backus-Naur Form in the implementation phase of the project.

The design process carried out by the author was a key step in designing and
implementing the project. Planned use cases allowed the developer to map out key
features of the final product and defined the ``correct'' way of performing
tasks by the user, which related directly to the functionalities of the product.
The thesis defined various UML diagrams to facilitate the design phase ---
diagrams such as activity and sequence diagrams have been used to map out links
and processes that happen between user interactions in a more visual manner.
Sequence diagrams provided a way of visualizing the internal processes of the
application.

The implementation phase of the thesis described the process of implementing the
business logic of the application. Various components of the EBNF parser: the
lexer, the parser, and the preprocessor, were implemented in the Rust
programming language, which were then compiled to WebAssembly to allow them to
run in a web environment. The checker component, which checks if a given input
string provided by the user belongs to the language generated by a given
grammar, was produces and compiled in a similar way. The business logic was put
together in the form of a command-line application as a most basic method of
interfacing with the implemented library. To fully address the issue of
accessibility of the application, a web-based application with a graphical user
interface was implemented and deployed on a web page and can run in a browser
without the need for installation on the user's device. The web application was
implemented in the Svelte framework, which incorporated the markup of the site,
the CSS styling, as well as JavaScript scripts used for user interaction.

The phase described the process of tokenization and proposed it in the form of a
Deterministic Finite Automaton. The lexer takes into account the proper
interpretation of Unicode graphemes. A technique of combinatory parsing was then
used to combine the tokens into an abstract syntax tree. The implementation
phase included the definition of an algorithm for detecting left recursion and
undefined production rules in a grammar. Occurrences of these are reported to
the user in a readable way that highlights the exact positions of errors in the
input string. The web application provides a basic code editor for inputting the
grammar, with such features as autocompletions and syntax highlighting. The tool
is able to visualize the parse tree to the user in an interactive way. The final
product is welcoming to new users than other similar tools. Users are able to
access the website and use the tool right from their browser without needing to
carry out a complex setup process, thanks to the deployment on the GitHub Pages
hosting service for static sites.

The thesis also analyzed the quality of the project by defining various unit and
integration tests to ensure the correctness of the business logic of the
application. The project utilized property-based testing as a useful addition to
a test suite that generates hundreds of test cases automatically. Integration
tests helped to combine implemented components into a single library and compile
it to the WebAssembly target. With the help of Google Lighthouse, the web
application underwent an external review to check its quality, progress, or
adherence to plans, modern web standards, and regulations. The thesis provided a
way to analyze the business logic of the application in terms of the so-called
cognitive complexity.

The designed and implemented system gives the opportunity to extend it with
various functionalities. First of all, the tool is designed from the start with
other grammar specifications in mind. Specifications like BNF, ABNF, Wirth
syntax notation, or other EBNF variations like the EBNF notation from the W3C
Extensible Markup Language (XML) are a great fit for further expansion and a way
to extend the list of supported languages. Each language would have an
associated parser that compiles it down to a common notation. Secondly, the
functionalities of the application could be expanded by providing a
code-generation aspect to the grammars given by users. Users would be able to
``kickstart'' their parsers by generating code in various popular programming
languages that represent the provided grammar (again, not necessarily in the
EBNF notation).

\bibliography{bibliography.bib}

\listoffigures

\listoftables

\listoflistings

\begin{appendices}

\chapter{Modified specification of the ISO/IEC 14977} \label{ch:modified-spec}

Listing~\ref{lst:specification} shows the modified version of the specification
of the ISO/IEC 14977 standard. The modification is described in detail in
Section~\ref{sec:modified-specification} of the thesis.

\begin{listing}[H]
  \inputminted[fontsize=\small,frame=lines,breaklines,linenos]
    {lexers/ebnf_lexer.py:EbnfLexer -x}{listings/specification.ebnf}
  \caption{Modified version of the EBNF language specification defined in
  \cite{iso-14977}.}
  \label{lst:specification}
\end{listing}

\chapter{CD contents}

\noindent The disc included with the thesis contains:
\begin{itemize}
  \item the source code of the project presented in
  Section~\ref{sbs:project-structure} --- 320 KB,
  \item the source code of the thesis in \LaTeX --- 1.28 MB,
  \item the thesis PDF document --- 980 KB,
\end{itemize}

\end{appendices}

\end{document}
